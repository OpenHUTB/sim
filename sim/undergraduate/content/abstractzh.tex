%!TEX root = ../csuthesis_main.tex
% 设置中文摘要
\keywordscn{智能驾驶\quad 仿真场景\quad 危险场景生成\quad 自动化技术}
%\categorycn{TP391}
\begin{abstractzh}


在智能驾驶系统的发展过程中,仿真场景的生成与优化是确保其安全性和可靠性的重要手段。本项目综述了智能驾驶危险仿真场景的生成和优化技术。首先,基于自然驾驶数据,识别并提取出具有代表性的危险驾驶场景,为仿真场景的构建提供了数据基础。其次,通过多维场景自动提取和融合方法,识别出典型的行车场景,如巡线、跟车、邻车切入等,并将其与动态驾驶场景进行融合,以生成更为复杂和真实的测试场景。此外,针对现有测试场景中危险场景数量不足的问题,提出了一种基于聚类分析和重要性采样的测试用例生成和增强方法,有效提高了危险场景的测试覆盖率和测试效率。最后,开发了一种自动化仿真测试平台,实现了测试场景的快速构建、仿真软件的联合调用以及结果分析与报告生成的全自动化执行。通过这些方法,能够显著提高智能驾驶系统在仿真环境中的安全测试效率,为智能驾驶技术的进一步发展提供了有力支持。


\end{abstractzh}