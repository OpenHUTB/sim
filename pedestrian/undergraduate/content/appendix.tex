\chapter{附录代码}

\section{Carla行人控制脚本}

\begin{lstlisting}[language=Python]
# coding:utf-8 
import carla
import random

client = carla.Client("localhost", 2000)
client.set_timeout(10.0)
world = client.get_world()
world = client.load_world('Town05')
pedestrain_blueprints = world.get_blueprint_library().filter("walker.pedestrian.0001")
# 设置行人起点
pedestrain = world.try_spawn_actor(random.choice(pedestrain_blueprints),
                                   carla.Transform(carla.Location(x=19, y=9, z=2), carla.Rotation(yaw=-90)))
pedestrain_control = carla.WalkerControl()
# 设置行人速度
pedestrain_control.speed = 2.0
pedestrain_rotation = carla.Rotation(0, -90, 0)
pedestrain_control.direction = pedestrain_rotation.get_forward_vector()
pedestrain.apply_control(pedestrain_control)

while True:
    # 设置终点条件
    if (pedestrain.get_location().y < -6.8):
        control = carla.WalkerControl()
        control.direction.x = 0
        control.direction.z = 0
        control.direction.y = 0
        pedestrain.apply_control(control)
        print("finish")
        break
\end{lstlisting}

\section{Carla行人往返路径控制脚本}

\begin{lstlisting}[language=Python]
# coding:utf-8 
import carla
import random

client = carla.Client("localhost", 2000)
client.set_timeout(10.0)
world = client.get_world()
world = client.load_world('Town05')
pedestrain_blueprints = world.get_blueprint_library().filter("walker.pedestrian.0001")
# 设置行人起点
pedestrain = world.try_spawn_actor(random.choice(pedestrain_blueprints),
                                   carla.Transform(carla.Location(x=19, y=9, z=2), carla.Rotation(yaw=-90)))
revert_flag = False

while True:
    world.wait_for_tick()
    control = carla.WalkerControl()
    control.direction.x = 0
    control.direction.z = 0
    control.speed = 2.0
    # 往返路径
    if (pedestrain.get_location().y > 7.7):
        revert_flag = True
    if (pedestrain.get_location().y < -7.0):
        revert_flag = False
    if (revert_flag):
        control.direction.y = -1
    else:
        control.direction.y = 1
    pedestrain.apply_control(control)
\end{lstlisting}

\section{Carla行人控制与障碍物检测脚本}

\begin{lstlisting}[language=Python]
import carla
import pygame
import sys
import numpy as np
import time
import random
import math

# 初始化 pygame
pygame.init()
WIDTH, HEIGHT = 1280, 720
screen = pygame.display.set_mode((WIDTH, HEIGHT))
pygame.display.set_caption("CARLA 行人控制 - 完整版")
clock = pygame.time.Clock()


class ObstacleDetector:
    def __init__(self, world, parent_actor, detection_range=5.0):
        self.world = world
        self.parent = parent_actor
        self.detection_range = detection_range
        self.obstacles = []

        # 碰撞传感器
        self.collision_sensor = self._setup_collision_sensor()

    def _setup_collision_sensor(self):
        blueprint = self.world.get_blueprint_library().find('sensor.other.collision')
        transform = carla.Transform(carla.Location(x=0.5, z=0.5))
        sensor = self.world.spawn_actor(blueprint, transform, attach_to=self.parent)
        sensor.listen(self._on_collision)
        return sensor

    def _on_collision(self, event):
        print(f"! 碰撞发生: {event.other_actor.type_id}")

    def update(self):
        self.obstacles = []
        parent_location = self.parent.get_location()

        # 检测车辆
        for vehicle in self.world.get_actors().filter('vehicle.*'):
            if vehicle.id != self.parent.id:
                distance = self._calculate_distance(parent_location, vehicle.get_location())
                if distance < self.detection_range:
                    self.obstacles.append({
                        'type': '车辆',
                        'distance': distance,
                        'actor': vehicle
                    })

        # 检测行人
        for walker in self.world.get_actors().filter('walker.*'):
            if walker.id != self.parent.id:
                distance = self._calculate_distance(parent_location, walker.get_location())
                if distance < self.detection_range:
                    self.obstacles.append({
                        'type': '行人',
                        'distance': distance,
                        'actor': walker
                    })

    def _calculate_distance(self, loc1, loc2):
        dx = loc1.x - loc2.x
        dy = loc1.y - loc2.y
        dz = loc1.z - loc2.z
        return math.sqrt(dx ** 2 + dy ** 2 + dz ** 2)

    def get_nearest_obstacle(self):
        return min(self.obstacles, key=lambda x: x['distance']) if self.obstacles else None


class CarlaSimulation:
    def __init__(self):
        # 连接CARLA
        self.client = carla.Client("localhost", 2000)
        self.client.set_timeout(20.0)
        self.world = self.client.load_world('Town01')
        self.original_settings = self.world.get_settings()

        # 同步模式设置
        self.settings = self.world.get_settings()
        self.settings.synchronous_mode = True
        self.settings.fixed_delta_seconds = 0.05
        self.world.apply_settings(self.settings)

        # 初始化对象
        self.walker = None
        self.camera = None
        self.ai_walkers = []
        self.vehicles = []

        # 初始化系统
        self._init_main_walker()
        self._init_camera()
        self._spawn_ai_walkers(10)  # 减少AI行人数量
        self._spawn_vehicles(10)  # 减少车辆数量
        self.obstacle_detector = ObstacleDetector(self.world, self.walker)

        # 使用中文字体
        self.font = pygame.font.Font("SimHei.ttf", 36)  # 请确保 SimHei.ttf 字体文件存在

    def _init_main_walker(self):
        walker_bp = self.world.get_blueprint_library().find('walker.pedestrian.0001')
        spawn_point = carla.Transform(carla.Location(x=30, y=3, z=0.5))

        for _ in range(5):
            self.walker = self.world.try_spawn_actor(walker_bp, spawn_point)
            if self.walker:
                break
            time.sleep(0.5)

        if not self.walker:
            raise RuntimeError("主行人生成失败")

    def _init_camera(self):
        camera_bp = self.world.get_blueprint_library().find('sensor.camera.rgb')
        camera_bp.set_attribute('image_size_x', str(WIDTH))
        camera_bp.set_attribute('image_size_y', str(HEIGHT))

        # 修正的摄像头位置和旋转参数
        self.camera = self.world.spawn_actor(
            camera_bp,
            carla.Transform(
                carla.Location(x=-5, z=2.5),  # 位置
                carla.Rotation(pitch=-10)  # 旋转
            ),
            attach_to=self.walker
        )

        # 图像处理回调
        self.image_surface = None

        def process_image(image):
            array = np.frombuffer(image.raw_data, dtype=np.uint8)
            array = array.reshape((HEIGHT, WIDTH, 4))[:, :, :3]
            array = array[:, :, ::-1]  # BGR转RGB
            self.image_surface = pygame.surfarray.make_surface(array.swapaxes(0, 1))

        self.camera.listen(process_image)

    def _spawn_ai_walkers(self, count):
        for _ in range(count):
            actor = self._create_ai_walker()
            if actor:
                self.ai_walkers.append(actor)

    def _create_ai_walker(self):
        try:
            spawn_loc = self.world.get_random_location_from_navigation()
            if not spawn_loc:
                return None

            walker_bp = random.choice(self.world.get_blueprint_library().filter('walker.pedestrian.*'))
            walker = self.world.try_spawn_actor(walker_bp, carla.Transform(spawn_loc))
            if not walker:
                return None

            controller_bp = self.world.get_blueprint_library().find('controller.ai.walker')
            controller = self.world.spawn_actor(controller_bp, carla.Transform(), walker)
            controller.start()
            controller.go_to_location(self.world.get_random_location_from_navigation())
            return (controller, walker)
        except:
            return None

    def _spawn_vehicles(self, count):
        spawn_points = [p for p in self.world.get_map().get_spawn_points() if p.location.z > 0]
        vehicle_bps = self.world.get_blueprint_library().filter('vehicle.*')

        for _ in range(count):
            if not spawn_points:
                break
            spawn_point = random.choice(spawn_points)
            vehicle_bp = random.choice(vehicle_bps)

            vehicle = self.world.try_spawn_actor(vehicle_bp, spawn_point)
            if vehicle:
                vehicle.set_autopilot(True)
                self.vehicles.append(vehicle)
                spawn_points.remove(spawn_point)

    def run(self):
        print("控制说明:WASD移动 | 空格跳跃 | Shift加速 | ESC退出")
        try:
            while True:
                self.world.tick()

                # 处理退出事件
                for event in pygame.event.get():
                    if event.type == pygame.QUIT or (event.type == pygame.KEYDOWN and event.key == pygame.K_ESCAPE):
                        return

                # 获取控制输入
                keys = pygame.key.get_pressed()
                control = carla.WalkerControl()
                direction = carla.Vector3D()

                # 移动控制
                if keys[pygame.K_w]: direction.x += 1
                if keys[pygame.K_s]: direction.x -= 1
                if keys[pygame.K_a]: direction.y -= 1
                if keys[pygame.K_d]: direction.y += 1

                # 配置参数
                control.jump = keys[pygame.K_SPACE]
                control.speed = 3.0 if (keys[pygame.K_LSHIFT] or keys[pygame.K_RSHIFT]) else 1.5
                if direction.length() > 0:
                    control.direction = direction.make_unit_vector()

                # 障碍物检测
                self.obstacle_detector.update()
                nearest = self.obstacle_detector.get_nearest_obstacle()

                # 自动制动
                if nearest and nearest['distance'] < 2.0:
                    control.speed = 0
                    print(f"! 自动停止:检测到 {nearest['type']} ({nearest['distance']:.1f}m)")

                self.walker.apply_control(control)

                # 更新显示
                screen.fill((0, 0, 0))
                if self.image_surface:
                    screen.blit(self.image_surface, (0, 0))
                self._draw_hud()
                pygame.display.flip()

                clock.tick(30)

        finally:
            self.cleanup()

    def _draw_hud(self):
        y_pos = 20
        nearest = self.obstacle_detector.get_nearest_obstacle()

        if nearest:
            text = self.font.render(f"最近障碍物: {nearest['type']} 距离: {nearest['distance']:.2f}m", True,
                                    (255, 0, 0))
            screen.blit(text, (20, y_pos))
            y_pos += 40

            if nearest['distance'] < 3.0:
                warning = self.font.render("警告:前方有障碍物!", True, (255, 0, 0))
                screen.blit(warning, (WIDTH // 2 - 100, HEIGHT - 60))

    def cleanup(self):
        print("\n正在清理资源...")
        self.world.apply_settings(self.original_settings)

        # 清理主摄像头
        if self.camera and self.camera.is_alive:
            print("清理主摄像头...")
            self.camera.stop()
            self.camera.destroy()

        # 清理主行人
        if self.walker and self.walker.is_alive:
            print("清理主行人...")
            self.walker.destroy()

        # 清理AI行人
        for controller, walker in self.ai_walkers:
            try:
                if controller.is_alive:
                    print(f"清理AI行人控制器: {controller.id}")
                    controller.stop()
                    controller.destroy()
                if walker.is_alive:
                    print(f"清理AI行人: {walker.id}")
                    walker.destroy()
            except Exception as e:
                print(f"清理AI行人错误: {e}")

        # 清理车辆
        for vehicle in self.vehicles:
            try:
                if vehicle.is_alive:
                    print(f"清理车辆: {vehicle.id}")
                    vehicle.destroy()
            except Exception as e:
                print(f"清理车辆错误: {e}")

        # 清理传感器
        if hasattr(self.obstacle_detector, 'collision_sensor'):
            if self.obstacle_detector.collision_sensor.is_alive:
                print("清理碰撞传感器...")
                self.obstacle_detector.collision_sensor.destroy()

        # 等待资源释放
        for _ in range(10):
            self.world.tick()
            time.sleep(0.1)

        pygame.quit()


if __name__ == "__main__":
    try:
        sim = CarlaSimulation()
        sim.run()
    except Exception as e:
        print(f"运行错误: {str(e)}")
        sys.exit(1)
\end{lstlisting}

\section{PID控制代码}

以下是基于PID控制的行人环境的代码实现:

\begin{lstlisting}[language=Python]
import carla
import numpy as np
import random
import time
import threading
import cv2

class PIDController:
    """PID控制器实现"""

    def __init__(self, kp, ki, kd, max_output):
        self.kp = kp
        self.ki = ki
        self.kd = kd
        self.max_output = max_output
        self.integral = 0.0
        self.previous_error = 0.0

    def reset(self):
        self.integral = 0.0
        self.previous_error = 0.0

    def compute(self, error, dt):
        self.integral += error * dt
        derivative = (error - self.previous_error) / dt if dt > 0 else 0.0
        output = self.kp * error + self.ki * self.integral + self.kd * derivative
        output = np.clip(output, -self.max_output, self.max_output)
        self.previous_error = error
        return output


class PIDPedestrianEnv:
    """基于PID控制的行人环境"""

    def __init__(self, target_location=carla.Location(x=180, y=120, z=1)):
        # ==== 初始化关键属性 ====
        self.last_lidar_points = np.array([])  # 初始化为空数组
        self.min_obstacle_distance = 10.0  # 初始默认值更大
        self.target_location = target_location
        self.collision_occurred = False
        self.obstacle_clusters = []
        self.current_speed = 0.0
        self.sensors = []
        self.dynamic_obstacles = []
        self.last_display = time.time()
        self.img_lock = threading.Lock()
        self.lidar_lock = threading.Lock()
        # 初始化方向向量
        self.target_direction = carla.Vector3D(0, 0, 0)
        self.avoidance_direction = carla.Vector3D(0, 0, 0)
        self.combined_direction = carla.Vector3D(0, 0, 0)
        self.obstacle_detected = False

        # ==== PID控制器 ====
        self.steering_pid = PIDController(kp=0.8, ki=0.001, kd=0.2, max_output=30)
        self.speed_pid = PIDController(kp=0.5, ki=0.01, kd=0.1, max_output=3.0)

        # ==== Carla连接配置 ====
        self.client = carla.Client("localhost", 2000)
        self.client.set_timeout(30.0)
        self._connect_to_server()

        # ==== 初始化Carla组件 ====
        self._preload_assets()
        self._setup_spectator()
        self._spawn_pedestrian()
        self._attach_sensors()
        self._spawn_dynamic_obstacles(num_vehicles=2, num_walkers=1)

    def _connect_to_server(self):
        """强制使用Town01地图并设置良好的天气"""
        for retry in range(5):
            try:
                self.world = self.client.load_world("Town01")
                if "Town01" in self.world.get_map().name:
                    # 设置更好的天气
                    weather = carla.WeatherParameters(
                        cloudiness=10.0,
                        precipitation=0.0,
                        sun_altitude_angle=70.0,  # 较高的太阳角度
                        fog_density=0.0,
                        wetness=0.0
                    )
                    self.world.set_weather(weather)
                    print("成功连接Carla服务器")
                    return
            except Exception as e:
                print(f"连接失败(尝试 {retry + 1}/5): {str(e)}")
                time.sleep(2)
        raise ConnectionError("无法连接到Carla服务器")

    def _preload_assets(self):
        """预加载蓝图"""
        self.blueprint_library = self.world.get_blueprint_library()
        self.walker_bps = self.blueprint_library.filter('walker.pedestrian.*')
        self.controller_bp = self.blueprint_library.find('controller.ai.walker')
        self.vehicle_bps = self.blueprint_library.filter('vehicle.*')
        self.collision_bp = self.blueprint_library.find('sensor.other.collision')
        self.lidar_bp = self._configure_lidar()
        self.camera_bp = self._configure_camera()

    def _configure_lidar(self):
        lidar = self.blueprint_library.find('sensor.lidar.ray_cast')
        # 增加检测范围和精度
        lidar.set_attribute('range', '15.0')  # 增加检测范围
        lidar.set_attribute('points_per_second', '50000')  # 增加点云密度
        lidar.set_attribute('rotation_frequency', '20')  # 增加旋转频率
        lidar.set_attribute('upper_fov', '10')  # 上视场角
        lidar.set_attribute('lower_fov', '-30')  # 下视场角,增加对地面和低障碍物的检测
        lidar.set_attribute('channels', '64')  # 增加通道数提高分辨率
        return lidar

    def _configure_camera(self):
        camera = self.blueprint_library.find('sensor.camera.rgb')
        # 提高分辨率到1280x720(高清)
        camera.set_attribute('image_size_x', '1280')
        camera.set_attribute('image_size_y', '720')
        # 增加视野范围
        camera.set_attribute('fov', '100')
        return camera

    def _setup_spectator(self):
        try:
            spectator = self.world.get_spectator()
            transform = carla.Transform(
                carla.Location(x=160, y=138, z=50),
                carla.Rotation(pitch=-90)
            )
            spectator.set_transform(transform)
        except Exception as e:
            print(f"视角设置失败: {str(e)}")

    def _spawn_pedestrian(self):
        """生成受控行人"""
        for _ in range(3):
            try:
                spawn_point = carla.Transform(
                    carla.Location(x=160, y=138, z=1.0),
                    carla.Rotation(yaw=random.randint(0, 360))
                )
                self.pedestrian = self.world.spawn_actor(
                    random.choice(self.walker_bps),
                    spawn_point)
                break
            except Exception as e:
                print(f"行人生成失败: {str(e)}")
                time.sleep(0.5)

    def _attach_sensors(self):
        """附加传感器"""
        try:
            # 碰撞传感器
            collision_sensor = self.world.spawn_actor(
                self.collision_bp,
                carla.Transform(),
                attach_to=self.pedestrian)
            collision_sensor.listen(lambda event: self._on_collision(event))

            # 激光雷达
            lidar_transform = carla.Transform(carla.Location(z=2.5))
            lidar_sensor = self.world.spawn_actor(
                self.lidar_bp,
                lidar_transform,
                attach_to=self.pedestrian)
            lidar_sensor.listen(lambda data: self._process_lidar(data))

            # 摄像头 - 调整位置和角度
            camera_transform = carla.Transform(
                carla.Location(x=0.5, z=1.6),  # 更接近真实人眼位置
                carla.Rotation(pitch=-5)  # 稍微向下的角度,更自然
            )
            camera_sensor = self.world.spawn_actor(
                self.camera_bp,
                camera_transform,
                attach_to=self.pedestrian)
            camera_sensor.listen(lambda image: self._process_image(image))

            self.sensors = [collision_sensor, lidar_sensor, camera_sensor]
        except Exception as e:
            print(f"传感器初始化失败: {str(e)}")
            self._cleanup_actors()
            raise

    def _spawn_dynamic_obstacles(self, num_vehicles, num_walkers):
        """生成动态障碍物"""
        vehicle_spawn_points = [
            p for p in self.world.get_map().get_spawn_points()
            if p.location.distance(self.pedestrian.get_location()) > 20.0
        ]
        for _ in range(num_vehicles):
            try:
                vehicle = self.world.spawn_actor(
                    random.choice(self.vehicle_bps),
                    random.choice(vehicle_spawn_points))
                vehicle.set_autopilot(True)
                self.dynamic_obstacles.append(vehicle)
            except:
                print("生成车辆失败,继续尝试")

        for _ in range(num_walkers):
            try:
                walker = self.world.spawn_actor(
                    random.choice(self.walker_bps),
                    carla.Transform(self._random_destination()))
                controller = self.world.spawn_actor(
                    self.controller_bp,
                    carla.Transform(),
                    attach_to=walker)
                controller.start()
                controller.go_to_location(self._random_destination())
                self.dynamic_obstacles.extend([walker, controller])
            except:
                print("生成行人失败,继续尝试")

    def _random_destination(self):
        return carla.Location(
            x=random.uniform(120, 200),
            y=random.uniform(100, 160),
            z=1.0)

    def _on_collision(self, event):
        self.collision_occurred = True

\end{lstlisting}

\section{强化学习行人环境与PPO训练脚本}

\begin{lstlisting}[language=Python]
import carla
import gymnasium as gym
import numpy as np
import random
import time
import threading
import cv2
import torch
from gymnasium import spaces
from stable_baselines3 import PPO
from stable_baselines3.common.vec_env import DummyVecEnv

# 动作空间定义
ACTION_DICT = {
    0: (0.0, 0.0),  # 停止
    1: (0.0, 1.0),  # 直行
    2: (-30.0, 0.8),  # 左转
    3: (30.0, 0.8),  # 右转
    4: (0.0, 2.0)  # 奔跑
}


class EnhancedPedestrianEnv(gym.Env):
    """强化学习行人环境完整实现"""

    def __init__(self, target_location=carla.Location(x=180, y=120, z=1)):
        super().__init__()

        # ==== 初始化关键属性 ====
        self.target_location = target_location
        self.last_reward = 0.0
        self.previous_speed = 0.0
        self.current_speed = 0.0
        self.collision_occurred = False
        self.min_obstacle_distance = 5.0
        self.previous_target_distance = 0.0
        self.episode_step = 0
        self.dynamic_obstacles = []
        self.sensors = []

        # ==== 线程安全锁 ====
        self.img_lock = threading.Lock()
        self.lidar_lock = threading.Lock()
        self.last_display = time.time()

        # ==== Carla连接配置 ====
        self.client = carla.Client("localhost", 2000)
        self.client.set_timeout(30.0)
        self._connect_to_server()

        # ==== 空间定义 ====
        self.action_space = spaces.Discrete(len(ACTION_DICT))
        self.observation_space = spaces.Box(
            low=np.array([-1.0] * 8 + [0.0, -1.0]),
            high=np.array([1.0] * 8 + [3.0, 1.0]),
            dtype=np.float32
        )

        # ==== 初始化Carla组件 ====
        self._preload_assets()
        self._setup_spectator()

    def _connect_to_server(self):
        """强制使用Town01地图的连接方法"""
        for retry in range(5):
            try:
                # 强制加载Town01地图
                self.world = self.client.load_world("Town01")
                map_name = self.world.get_map().name
                if "Town01" in map_name:
                    print(f"成功加载Town01地图 (Carla v{self.client.get_server_version()})")
                    return
                else:
                    print(f"地图加载异常,获取到:{map_name}")
            except RuntimeError as e:
                print(f"地图加载失败(尝试 {retry + 1}/5):{str(e)}")
                try:
                    # 如果无法加载,尝试启动离线服务器
                    if not self.client.get_client_version():  # 检查客户端连接
                        self.client.start()
                        print("已启动离线Carla服务器")
                    time.sleep(2)
                except Exception as inner_e:
                    print(f"服务器启动失败:{str(inner_e)}")
                    time.sleep(2)
            except Exception as e:
                print(f"意外错误:{str(e)}")
                time.sleep(2)

        # 最终检查
        try:
            final_map = self.client.get_world().get_map().name
            raise ConnectionError(f"最终连接失败,当前地图:{final_map}(应使用Town01)")
        except:
            raise ConnectionError("无法连接到Carla服务器,请检查:\n1. Carla服务器是否运行\n2. 端口2000是否可用")

    def _preload_assets(self):
        """预加载蓝图资产"""
        self.blueprint_library = self.world.get_blueprint_library()
        self.walker_bps = self.blueprint_library.filter('walker.pedestrian.*')
        self.controller_bp = self.blueprint_library.find('controller.ai.walker')
        self.vehicle_bps = self.blueprint_library.filter('vehicle.*')
        self.camera_bp = self._configure_camera()
        self.lidar_bp = self._configure_lidar()
        self.collision_bp = self.blueprint_library.find('sensor.other.collision')

    def _configure_camera(self):
        """配置RGB摄像头"""
        camera_bp = self.blueprint_library.find('sensor.camera.rgb')
        camera_bp.set_attribute('image_size_x', '640')  # 降低分辨率
        camera_bp.set_attribute('image_size_y', '360')
        camera_bp.set_attribute('fov', '90')
        return camera_bp

    def _configure_lidar(self):
        """配置激光雷达"""
        lidar_bp = self.blueprint_library.find('sensor.lidar.ray_cast')
        lidar_bp.set_attribute('range', '10.0')
        lidar_bp.set_attribute('points_per_second', '10000')  # 降低点云密度
        return lidar_bp

    def _setup_spectator(self):
        """设置观察视角"""
        try:
            spectator = self.world.get_spectator()
            transform = carla.Transform(
                carla.Location(x=160, y=138, z=50),
                carla.Rotation(pitch=-90)
            )
            spectator.set_transform(transform)
        except Exception as e:
            print(f"视角设置失败: {str(e)}")

    def reset(self, **kwargs):
        """重置环境状态"""
        # 清理旧Actor
        self._cleanup_actors()
        time.sleep(0.5)  # 增加清理间隔

        # 初始化状态变量
        self.episode_step = 0
        self.collision_occurred = False
        self.last_reward = 0.0
        self.previous_speed = 0.0
        self.current_speed = 0.0
        self.min_obstacle_distance = 5.0
        self.previous_target_distance = 0.0

        # 生成新行人
        self._spawn_pedestrian()

        # 附加传感器
        self._attach_sensors()

        # 生成障碍物(初始数量较少)
        self._spawn_dynamic_obstacles(num_vehicles=2, num_walkers=1)

        return self._get_obs(), {}

    def _spawn_pedestrian(self):
        """生成受控行人(修复括号问题)"""
        for _ in range(3):
            try:
                spawn_point = carla.Transform(  # 修复此处括号
                    carla.Location(x=160, y=138, z=1.0),
                    carla.Rotation(yaw=random.randint(0, 360))
                )  # 补全括号
                self.pedestrian = self.world.spawn_actor(
                    random.choice(self.walker_bps),
                    spawn_point
                )
                break
            except Exception as e:
                print(f"行人生成失败: {str(e)}")
                time.sleep(0.5)

        # 添加AI控制器
        self.controller = self.world.spawn_actor(
            self.controller_bp,
            carla.Transform(),
            attach_to=self.pedestrian,
            attachment_type=carla.AttachmentType.Rigid
        )
        self.controller.start()

    def _attach_sensors(self):
        """附加传感器(增加错误处理)"""
        try:
            # 碰撞传感器
            collision_sensor = self.world.spawn_actor(
                self.collision_bp,
                carla.Transform(),
                attach_to=self.pedestrian
            )
            collision_sensor.listen(lambda event: self._on_collision(event))

            # 激光雷达
            lidar_transform = carla.Transform(carla.Location(z=2.5))
            lidar_sensor = self.world.spawn_actor(
                self.lidar_bp,
                lidar_transform,
                attach_to=self.pedestrian
            )
            lidar_sensor.listen(lambda data: self._process_lidar(data))

            # 摄像头(降低帧率)
            camera_transform = carla.Transform(
                carla.Location(x=0.8, z=1.7),
                carla.Rotation(pitch=-10)
            )
            camera_sensor = self.world.spawn_actor(
                self.camera_bp,
                camera_transform,
                attach_to=self.pedestrian
            )
            camera_sensor.listen(lambda image: self._process_image(image))

            self.sensors = [collision_sensor, lidar_sensor, camera_sensor]
        except Exception as e:
            print(f"传感器初始化失败: {str(e)}")
            self._cleanup_actors()
            raise

    def _spawn_dynamic_obstacles(self, num_vehicles, num_walkers):
        """生成动态障碍物(安全版本)"""
        try:
            # 生成车辆
            vehicle_spawn_points = [
                p for p in self.world.get_map().get_spawn_points()
                if p.location.distance(self.pedestrian.get_location()) > 20.0
            ]
            for _ in range(num_vehicles):
                vehicle = self.world.spawn_actor(
                    random.choice(self.vehicle_bps),
                    random.choice(vehicle_spawn_points)
                )
                vehicle.set_autopilot(True)
                self.dynamic_obstacles.append(vehicle)

                # 生成行人障碍物
                for _ in range(num_walkers):
                    walker = self.world.spawn_actor(
                        random.choice(self.walker_bps),
                        carla.Transform(self._random_destination()))


                controller = self.world.spawn_actor(
                    self.controller_bp,
                    carla.Transform(),
                    attach_to=walker,
                    attachment_type=carla.AttachmentType.Rigid)
                controller.start()
                controller.go_to_location(self._random_destination())
                self.dynamic_obstacles.extend([walker, controller])
        except Exception as e:
            print(f"障碍物生成失败: {str(e)}")

    def _random_destination(self):
        """生成随机目的地"""
        return carla.Location(
            x=random.uniform(120, 200),
            y=random.uniform(100, 160),
            z=1.0)

    def _on_collision(self, event):
        """碰撞事件处理"""
        self.collision_occurred = True

    def _process_lidar(self, data):
        """处理激光雷达数据(安全版本)"""
        try:
            points = np.frombuffer(data.raw_data, dtype=np.float32).reshape(-1, 4)
            with self.lidar_lock:
                if len(points) > 0:
                    distances = np.sqrt(points[:, 0] ** 2 + points[:, 1] ** 2)
                    self.min_obstacle_distance = np.min(distances)
                else:
                    self.min_obstacle_distance = 5.0
        except Exception as e:
            print(f"激光雷达处理错误: {str(e)}")

    def _process_image(self, image):
        """处理摄像头图像(优化性能)"""
        with self.img_lock:
            try:
                # 降低处理频率
                if time.time() - self.last_display < 0.1:  # 10FPS
                    return

                array = np.frombuffer(image.raw_data, dtype=np.uint8)
                array = array.reshape((image.height, image.width, 4))
                img_bgr = cv2.cvtColor(array[:, :, :3], cv2.COLOR_RGB2BGR)

                # 显示基本信息
                cv2.putText(img_bgr, f"Speed: {self.current_speed:.1f}m/s",
                            (10, 30), cv2.FONT_HERSHEY_SIMPLEX, 0.6, (255, 255, 255), 1)
                cv2.putText(img_bgr, f"Reward: {self.last_reward:.2f}",
                            (10, 60), cv2.FONT_HERSHEY_SIMPLEX, 0.6, (255, 255, 255), 1)

                # 显示处理后的图像
                cv2.imshow('Pedestrian View', cv2.resize(img_bgr, (960, 480)))
                cv2.waitKey(1)
                self.last_display = time.time()
            except Exception as e:
                print(f"图像处理错误: {str(e)}")

    def _get_obs(self):
        """获取观测数据"""
        try:
            transform = self.pedestrian.get_transform()
            current_loc = transform.location
            current_rot = transform.rotation

            # 计算目标方向
            target_vector = self.target_location - current_loc
            target_distance = target_vector.length()
            target_dir = target_vector.make_unit_vector() if target_distance > 0 else carla.Vector3D()

            # 转换到局部坐标系
            yaw = np.radians(current_rot.yaw)
            rotation_matrix = np.array([
                [np.cos(yaw), np.sin(yaw)],
                [-np.sin(yaw), np.cos(yaw)]
            ])
            local_target = rotation_matrix @ np.array([target_dir.x, target_dir.y])

            # 获取障碍物距离
            with self.lidar_lock:
                obstacle_dist = self.min_obstacle_distance

            return np.array([
                current_loc.x / 200.0 - 1.0,
                current_loc.y / 200.0 - 1.0,
                local_target[0],
                local_target[1],
                np.clip(obstacle_dist / 5.0, 0.0, 1.0),
                self.current_speed / 3.0,
                (self.current_speed - self.previous_speed) / 3.0,
                np.sin(yaw),
                np.cos(yaw),
                target_distance / 100.0
            ], dtype=np.float32)
        except Exception as e:
            print(f"获取观测数据失败: {str(e)}")
            return np.zeros(self.observation_space.shape)

    def step(self, action_idx):
        """执行动作(安全版本)"""
        try:
            self.episode_step += 1

            # 解析动作
            yaw_offset, speed_ratio = ACTION_DICT[action_idx]

            # 平滑转向
            current_yaw = self.pedestrian.get_transform().rotation.yaw
            target_yaw = current_yaw + yaw_offset * 0.3
            self.pedestrian.set_transform(carla.Transform(
                self.pedestrian.get_location(),
                carla.Rotation(yaw=target_yaw)))

            # 速度控制
            base_speed = 1.5 + 1.5 * speed_ratio
            safe_speed = min(base_speed, 3.0) if self.min_obstacle_distance > 2.0 else 0.8
            self.previous_speed = self.current_speed
            self.current_speed = safe_speed

            # 应用控制
            control = carla.WalkerControl(
                direction=carla.Vector3D(1, 0, 0),
                speed=safe_speed)
            self.pedestrian.apply_control(control)

            # 推进仿真
            self.world.tick()

            # 计算奖励
            new_obs = self._get_obs()
            current_target_dist = new_obs[-1] * 100.0
            progress = self.previous_target_distance - current_target_dist
            speed_reward = min(safe_speed / 3.0, 0.5)
            collision_penalty = 10.0 if self.collision_occurred else 0.0
            time_penalty = 0.05
            reward = progress * 2.0 + speed_reward - time_penalty - collision_penalty
            self.last_reward = reward
            self.previous_target_distance = current_target_dist

            # 终止条件
            done = False
            if self.collision_occurred:
                done = True
                reward -= 20.0
            elif current_target_dist < 2.0:
                done = True
                reward += 20.0

            return new_obs, reward, done, False, {}
        except Exception as e:
            print(f"执行步骤时发生错误: {str(e)}")
            return np.zeros(self.observation_space.shape), 0, True, False, {}

    def _cleanup_actors(self):
        """增强版资源清理"""
        destroy_list = []

        # 主行人及控制器
        if hasattr(self, 'pedestrian') and self.pedestrian.is_alive:
            destroy_list.append(self.pedestrian)
        if hasattr(self, 'controller') and self.controller.is_alive:
            destroy_list.append(self.controller)

        # 传感器
        for sensor in self.sensors:
            if sensor.is_alive:
                destroy_list.append(sensor)

        # 动态障碍物
        for obstacle in self.dynamic_obstacles:
            if obstacle.is_alive:
                destroy_list.append(obstacle)

        # 批量销毁
        if destroy_list:
            try:
                self.client.apply_batch([carla.command.DestroyActor(x) for x in destroy_list])
                time.sleep(0.5)  # 等待销毁完成
            except Exception as e:
                print(f"销毁Actor时发生错误: {str(e)}")

        self.sensors = []
        self.dynamic_obstacles = []

    def close(self):
        """关闭环境"""
        self._cleanup_actors()
        cv2.destroyAllWindows()


class TrainingWrapper(gym.Wrapper):
    """训练包装器(稳定版)"""

    def __init__(self, env):
        super().__init__(env)
        self.episode_count = 0
        self.success_history = []
        self.model = None

    def step(self, action):
        try:
            obs, reward, terminated, truncated, info = self.env.step(action)
            info['episode'] = {'r': reward, 'l': self.episode_count}
            return obs, reward, terminated, truncated, info
        except Exception as e:
            print(f"训练步骤错误: {str(e)}")
            return np.zeros(self.env.observation_space.shape), 0, True, False, {}

    def reset(self, **kwargs):
        self.episode_count += 1
        if self.episode_count % 50 == 0:  # 每50回合保存检查点
            self.save_checkpoint()
        return self.env.reset(**kwargs)

    def save_checkpoint(self):
        """安全保存检查点"""
        if self.model is not None:
            try:
                timestamp = time.strftime("%Y%m%d-%H%M%S")
                self.model.save(f"pedestrian_model_{timestamp}")
                print(f"检查点已保存:pedestrian_model_{timestamp}")
            except Exception as e:
                print(f"保存检查点失败: {str(e)}")


if __name__ == "__main__":
    # 初始化环境
    env = EnhancedPedestrianEnv()
    wrapped_env = TrainingWrapper(env)
    vec_env = DummyVecEnv([lambda: wrapped_env])

    # 配置PPO模型(优化参数)
    model = PPO(
        "MlpPolicy",
        vec_env,
        verbose=1,
        learning_rate=2e-4,
        n_steps=2048,
        batch_size=128,
        gamma=0.99,
        policy_kwargs={
            "net_arch": dict(pi=[256, 256], vf=[256, 256]),
            "activation_fn": torch.nn.ReLU
        },
        device='auto'
    )
    wrapped_env.model = model

    try:
        # 分阶段训练
        print("开始第一阶段训练(10k steps)...")
        model.learn(total_timesteps=10000)

        print("开始第二阶段训练(50k steps)...")
        model.learn(total_timesteps=50000, reset_num_timesteps=False)

    finally:
        model.save("pedestrian_ppo")
        vec_env.close()
        print("训练完成,最终模型已保存。")
\end{lstlisting}

\section{基于强化学习PPO算法的行人导航代码}

\begin{lstlisting}[language=Python]
import carla
import gymnasium as gym
import numpy as np
import random
import time
import threading
import torch
import gc
from gymnasium import spaces
from stable_baselines3 import PPO
from stable_baselines3.common.vec_env import DummyVecEnv

ACTION_DICT = {
    0: (0.0, 0.0),  # 停止
    1: (0.0, 1.0),  # 直行
    2: (-30.0, 0.8),  # 左转
    3: (30.0, 0.8),  # 右转
    4: (0.0, 2.0)  # 奔跑
}

def reset_environment(env):
    """重置环境,确保每次运行时环境被清理并初始化"""
    print("正在重置环境...")
    try:
        env.close()  # 确保关闭上次的环境
        env.reset()  # 重置环境
        print("环境已重置")
    except Exception as e:
        print(f"重置环境时发生错误: {str(e)}")

class EnhancedPedestrianEnv(gym.Env):
    def __init__(self, target_location=carla.Location(x=202, y=65, z=1)):
        super().__init__()

        # 初始化关键属性
        self.target_location = target_location
        self.last_reward = 0.0
        self.previous_speed = 0.0
        self.current_speed = 0.0
        self.collision_occurred = False
        self.min_obstacle_distance = 5.0
        self.previous_target_distance = 0.0
        self.episode_step = 0
        self.sensors = []
        self.target_actor = None
        self.cleanup_lock = threading.Lock()

        # Carla连接配置
        self.client = carla.Client("localhost", 2000)
        self.client.set_timeout(30.0)
        self._connect_to_server()

        # 空间定义
        self.action_space = spaces.Discrete(len(ACTION_DICT))
        self.observation_space = spaces.Box(
            low=np.array([-1.0] * 8 + [0.0, -1.0]),
            high=np.array([1.0] * 8 + [3.0, 1.0]),
            dtype=np.float32
        )

        # 初始化组件
        self._preload_assets()
        self._setup_spectator()

    def _connect_to_server(self):
        """连接Carla服务器"""
        for retry in range(5):
            try:
                self.world = self.client.load_world("Town01")
                settings = self.world.get_settings()
                settings.synchronous_mode = True
                settings.fixed_delta_seconds = 0.05
                self.world.apply_settings(settings)

                if "Town01" in self.world.get_map().name:
                    print(f"成功加载Town01地图 (Carla v{self.client.get_server_version()})")
                    return
            except Exception as e:
                print(f"连接失败(尝试 {retry + 1}/5):{str(e)}")
                time.sleep(2)
        raise ConnectionError("无法连接到Carla服务器")

    def _preload_assets(self):
        """预加载蓝图资产"""
        self.blueprint_library = self.world.get_blueprint_library()
        self.walker_bps = self.blueprint_library.filter('walker.pedestrian.*')
        self.controller_bp = self.blueprint_library.find('controller.ai.walker')
        self.vehicle_bps = self.blueprint_library.filter('vehicle.*')
        self.lidar_bp = self._configure_lidar()
        self.collision_bp = self.blueprint_library.find('sensor.other.collision')
        self.target_marker_bp = self.blueprint_library.find('static.prop.streetbarrier')

    def _configure_lidar(self):
        """配置激光雷达"""
        lidar_bp = self.blueprint_library.find('sensor.lidar.ray_cast')
        lidar_bp.set_attribute('range', '10.0')
        lidar_bp.set_attribute('points_per_second', '10000')
        return lidar_bp

    def _setup_spectator(self):
        """初始化观察视角"""
        self.spectator = self.world.get_spectator()
        self._update_spectator_view()

    def _update_spectator_view(self):
        """更新俯视视角"""
        try:
            if hasattr(self, 'pedestrian') and self.pedestrian.is_alive:
                ped_loc = self.pedestrian.get_transform().location
                self.spectator.set_transform(carla.Transform(
                    carla.Location(x=ped_loc.x, y=ped_loc.y, z=20),
                    carla.Rotation(pitch=-90)
                ))
        except Exception as e:
            print(f"视角更新失败: {str(e)}")

    def _spawn_target_marker(self):
        """生成目标点标记"""
        if self.target_actor and self.target_actor.is_alive:
            self.target_actor.destroy()
        self.target_actor = self.world.spawn_actor(
            self.target_marker_bp,
            carla.Transform(self.target_location, carla.Rotation())
        )

    def reset(self, **kwargs):
        """重置环境"""
        with self.cleanup_lock:
            self._cleanup_actors()
            time.sleep(0.5)

            # 显式停止控制器(防止残留控制信号)
            if hasattr(self, 'controller') and self.controller.is_alive:
                self.controller.stop()

            # 重置状态变量
            self.episode_step = 0
            self.collision_occurred = False
            self.last_reward = 0.0
            self.previous_speed = 0.0
            self.current_speed = 0.0
            self.min_obstacle_distance = 5.0
            self.previous_target_distance = 0.0

            # 生成新实例
            self._spawn_pedestrian()
            self._attach_sensors()
            self._spawn_target_marker()
            self._update_spectator_view()

            return self._get_obs(), {}

    def _spawn_pedestrian(self):
        """生成受控行人"""
        for _ in range(3):
            try:
                # 设置行人生成位置
                spawn_point = carla.Transform(
                    carla.Location(x=160, y=138, z=1.0),
                    carla.Rotation(yaw=random.randint(0, 360))
                )
                self.pedestrian = self.world.spawn_actor(
                    random.choice(self.walker_bps),
                    spawn_point
                )
                break
            except Exception as e:
                print(f"行人生成失败: {str(e)}")
                time.sleep(0.5)

        # 添加控制器
        self.controller = self.world.spawn_actor(
            self.controller_bp,
            carla.Transform(),
            attach_to=self.pedestrian,
            attachment_type=carla.AttachmentType.Rigid
        )
        self.controller.start()

    def _attach_sensors(self):
        """附加传感器"""
        try:
            # 碰撞传感器
            collision_sensor = self.world.spawn_actor(
                self.collision_bp,
                carla.Transform(),
                attach_to=self.pedestrian
            )
            collision_sensor.listen(lambda e: self._on_collision(e))

            # 激光雷达
            lidar = self.world.spawn_actor(
                self.lidar_bp,
                carla.Transform(carla.Location(z=2.5)),
                attach_to=self.pedestrian
            )
            lidar.listen(lambda d: self._process_lidar(d))

            self.sensors = [collision_sensor, lidar]
        except Exception as e:
            print(f"传感器初始化失败: {str(e)}")
            self._cleanup_actors()
            raise

    def _on_collision(self, event):
        """碰撞处理"""
        self.collision_occurred = True

    def _process_lidar(self, data):
        """处理激光雷达数据"""
        try:
            points = np.frombuffer(data.raw_data, dtype=np.float32).reshape(-1, 4)
            with self.cleanup_lock:
                if len(points) > 0 and hasattr(self, 'min_obstacle_distance'):
                    distances = np.sqrt(points[:, 0] ** 2 + points[:, 1] ** 2)
                    self.min_obstacle_distance = np.min(distances)
                else:
                    self.min_obstacle_distance = 5.0
        except Exception as e:
            print(f"激光雷达处理错误: {str(e)}")

    def _get_obs(self):
        """获取观测数据"""
        try:
            transform = self.pedestrian.get_transform()
            current_loc = transform.location
            current_rot = transform.rotation

            # 计算目标方向
            target_vector = self.target_location - current_loc
            target_distance = target_vector.length()
            target_dir = target_vector.make_unit_vector() if target_distance > 0 else carla.Vector3D()

            # 转换到局部坐标系
            yaw = np.radians(current_rot.yaw)
            rotation_matrix = np.array([
                [np.cos(yaw), np.sin(yaw), 0],
                [-np.sin(yaw), np.cos(yaw), 0],
                [0, 0, 1]
            ])
            local_target = rotation_matrix @ np.array([target_dir.x, target_dir.y, target_dir.z])

            return np.array([
                current_loc.x / 200 - 1,
                current_loc.y / 200 - 1,
                local_target[0],
                local_target[1],
                np.clip(self.min_obstacle_distance / 5, 0, 1),
                self.current_speed / 3,
                (self.current_speed - self.previous_speed) / 3,
                np.sin(yaw),
                np.cos(yaw),
                target_distance / 100
            ], dtype=np.float32)
        except Exception as e:
            print(f"观测获取失败: {str(e)}")
            return np.zeros(self.observation_space.shape)

\end{lstlisting}
