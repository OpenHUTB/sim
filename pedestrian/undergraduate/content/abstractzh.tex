%!TEX root = ../csuthesis_main.tex
% 设置中文摘要
\keywordscn{行人导航\quad 强化学习\quad 虚拟仿真\quad Carla平台\quad 路径规划\quad 避障}
%\categorycn{TP391}
\begin{abstractzh}

城市交通系统智能化发展下,行人导航与控制研究对交通管理和智慧城市建设至关重要。传统方法基于规则模型或简单路径规划,难以适应现代交通需求。强化学习,一种自适应学习最优策略的人工智能方法,在动态环境决策中显示出潜力。本研究提出基于虚幻引擎和Carla平台的行人导航系统,通过强化学习优化路径规划与避障,适应复杂环境。实验表明,该系统在动态环境中提升避障能力和路径规划效率,拓展了强化学习在智能交通系统中的应用,为智慧城市建设提供技术支持。城市化快速进程中,交通系统智能化转型关键在于提升效率和安全性。行人导航技术虽解决基本需求,但在复杂多变环境中表现不足。开发适应性强的行人导航系统变得迫切。强化学习在处理不确定性和动态环境决策中具优势。本研究结合虚拟仿真技术,开发创新行人导航系统,利用虚幻引擎和Carla平台构建逼真导航环境,实时优化路径规划和避障策略。为验证系统有效性,研究在多种仿真场景中进行实验,包括导航准确性、避障能力和路径规划效率评估。结果显示,基于强化学习的系统在动态环境中具有显著优势,提升导航体验和交通效率。研究成果为强化学习在智能交通系统中应用开辟新领域,为智慧城市建设提供技术支持。随着技术进步和交通需求增长,基于强化学习的行人导航系统将在城市交通管理中扮演重要角色。

\end{abstractzh}
