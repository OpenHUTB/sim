%!TEX root = ../csuthesis_main.tex
% 设置中文摘要
\keywordscn{行人导航\quad 强化学习\quad 虚拟仿真\quad Carla平台\quad 路径规划\quad 避障}
%\categorycn{TP391}
\begin{abstractzh}

智慧城市环境下交通环境对行人导航的要求更高,传统的基于规划路径的方式已经无法满足动态、复杂环境下行人的变化。为了提升行人导航的自适应能力和路径规划效率,本文在虚幻引擎与Carla中模拟一个真实的步行环境,构建一个具备强学习能力的行人控制和行人导航系统,动态仿真真实城市的交通环境,为行人强化学习模型提供测试、训练环境。本文将DQN、PPO等强化学习算法引入行人控制和行人避障,设计多维度奖励函数提升智能体对目标、避障、路径的理解能力,从而提升复杂交通环境下行人的决策能力。对行人导航、避障和路径多场景仿真评估,验证了系统在动态、复杂环境下的鲁棒性。验证了虚拟仿真和强化学习的融合可以拓展智慧交通应用领域,对多智能体协同智能交通未来发展提供可行性方案与路径支撑。

\end{abstractzh}
