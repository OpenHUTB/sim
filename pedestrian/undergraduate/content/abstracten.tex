%!TEX root = ../csuthesis_main.tex
\keywordsen{Pedestrian navigation, Reinforcement learning, Virtual simulation, Carla platform, Path planning, Obstacle avoidance}
\begin{abstracten}

As urban traffic systems evolve towards greater intelligence, the study and development of pedestrian navigation and control systems have become increasingly significant in the realm of traffic management and the advancement of smart city infrastructures. Traditional methods of guiding pedestrians, which frequently rely on rule-based models or rudimentary path planning algorithms, are often found to be inadequate when it comes to adapting to the ever-changing and intricate nature of urban environments. This shortcoming poses a challenge in satisfying the real-time and intelligent demands of contemporary traffic systems. In the recent past, reinforcement learning, an artificial intelligence technique that excels in adaptively learning the most effective strategies, has emerged as a powerful tool in addressing decision-making challenges within dynamic settings. This particular research endeavor introduces a novel pedestrian navigation system that is built upon the robust foundations of Unreal Engine and the Carla platform, effectively merging virtual simulation technology with the principles of reinforcement learning. The system is designed to refine pedestrian path planning and obstacle avoidance functionalities by leveraging the adaptive learning capabilities of reinforcement learning, thus allowing for the dynamic adjustment of walking routes in response to the complexities of the surrounding environment. To thoroughly assess and contrast its performance against conventional path planning algorithms, the system underwent a series of experiments within diverse simulation environments. The outcomes of these experiments have indicated that the pedestrian navigation system, which is underpinned by reinforcement learning, notably enhances the efficiency of pedestrian obstacle avoidance and path planning within dynamic urban landscapes. This research not only broadens the scope of reinforcement learning applications within intelligent traffic systems but also offers innovative technical solutions that could be pivotal in the ongoing development of smart cities in the future.

\end{abstracten}
