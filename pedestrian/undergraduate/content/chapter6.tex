\chapter{结果与展望}

\section{实验结果}

本文以CARLA模拟平台为实验环境,采用先进的基于增强学习算法训练行人路径规划模型的方法,构建可规避障碍物从而提升行人步行效率和步行安全的深度学习行人路径规划系统,通过多种天气、多种时间、多种交通密度等多种情况下的训练和实验,确保了其在各种复杂情况下的表现,取得了可供借鉴的实验效果,为行人路径规划研究提供参考意义。

1.\textbf{训练过程与效果:}

本项目基于 CARLA 仿真平台设计实现行人导航系统,运用 PPO、DQN、PID 等多种强化学习算法训练行人移动并开展不同策略对比实验,结果显示基于 PPO 和 DQN 的模型在避障与目标导航任务中性能优异,不同训练阶段行人导航的成功率、路径效率和速度等指标有效提升,PPO 算法使行人在复杂环境中较稳定避障并快速找到目标位置,DQN 模型目标导向性和避障灵活性好但在复杂障碍物环境中响应迟缓,PID 控制策略稳定性高但面对多变障碍物灵活性不足。实验核心是用强化学习算法控制行人在模拟环境行进并避障,基于 PPO 算法训练行人模型并设置多场景模拟现实情况,训练中模型逐渐适应环境变化且多数情况下成功完成起点到目标的导航,PPO 算法适应能力良好,训练中模型提升对障碍物的感知与避让能力和行进效率,数万步训练后行人模型能在给定环境高效避开车辆、行人等动态障碍物到达目标位置。

2.\textbf{模型评估与性能分析:}

通过测试在训练后模型在实际场景中的表现来验证模型的关键性指标,模型在复杂、动态的场景中能迅速规划出有效路径并规避无效绕行停顿,在可接受的时间内到达目标点;不受行人、车辆等其他动态障碍物的干扰有良好的避障能力,通过多次训练来适应各种障碍物所处位置和运动趋势而规划出正确路径避免碰撞。模型采用 PPO 算法在训练过程中稳定高效,训练通过几步就靠近最终目标,一般可以无碰撞在多个实验环境中达成目标。为了充分验证模型性能,评估了从简单到复杂多个难度等级的多种环境,训练后模型行人模型在复杂城市街道场景和简单开放场景中的成功率都很高,模型验证了复杂度不高的鲁棒性。

3.\textbf{实时控制效果:}

模型训练后,在真实场景下仿真测试,检测实时控制效果。实时路径控制通过实时环境信息调整行人行走的速度和方向,确保行人方向可以正确避开障碍物并朝目标方向行进。模型在多种实验条件下均很好适应环境信息的变化,在复杂多变的障碍物场景中或环境信息切换的情况下,均能很好适应环境信息变化。

\section{未来工作与展望}

虽然本论文完成了基于强化学习的行人导航控制系统,并在多个仿真环境中得到了比较满意的实验结果,但是仍有很多方面需要继续完善和改进。未来的研究工作可以从以下几个方向进行深入探索和扩展:

1.\textbf{优化和增强算法:}本次研究中的 PPO 算法控制较好,但是也存在局限性,面对更难的交通场景,在完成训练并收敛需要耗费较长时间,后续研究可以将其与深度 Q 网络、基于模型的强化学习等算法结合,提高训练效率,加速收敛。深度 Q 网络、基于模型的强化学习等算法可继续研究,将其和 PPO 结合,可以提升算法在复杧行为环境中的算法性能,或者将混合强化学习的算法进行结合更多的场景,更为复杂的环境。

2.\textbf{多智能体协作与调度:}目前学术界和工业界研究重点多集中于单一行人导航控制技术,而随着智慧城市与智能交通系统的发展进步,未来研究方向有望拓展至多智能体系统。在多智能体系统中多个行人、车辆及其他智能体间的协作愈发关键,因此研究多人环境中行人与行人、行人与车辆的协作协调机制十分重要。采用多智能体强化学习(MARL)方法模拟不同智能体相互作用并优化其路径选择以提高系统运行效率是一个潜在研究方向。

3.\textbf{传感器环境认知能力提升:}已有的研究多使用激光雷达、摄像头这两种常用传感器进行感知认知,在后续的研究中可以引入更多种类的传感器来增强行人智能体对环境的感知能力,例如深度摄像头、红外传感器等。这些高层次的传感器能够提供低光、雨雾雪情况下更丰富的环境数据,感知能力增强后可以辅助行人智能体实现更好的避障、减少碰撞、获得更丰富更准确的环境信息,而环境信息又对智能体的路径规划决策起重要作用,辅助智能体更智能的规划出最优路径、提高效率、保障安全。

4.\textbf{复杂场景的应用与测试:}当前实验包括不同环境、不同障碍物等情况,由于仿真的限制,无法对真实场景中复杂的真实影响因素进行考验,今后工作中将系统投入到真实世界,在城市道路、繁忙地点、各种公共场所等复杂交通环境对实际障碍物更加复杂的情况进行考验,同时考虑交通环境中天气、光线等复杂影响因素对系统的影响。行人控制系统在真实世界的实时性和稳定性有更多考验,如何在实时高效率寻路的同时对复杂的外界变化进行处理是今后工作中的重点。

5.\textbf{集成与系统优化:}智能城市与自动驾驶技术的进步普及对行人导航控制系统提出多系统集成需求,需与自动驾驶、交通管理等系统协同以确保行人在自动驾驶环境中与车辆及其他元素协调共存,这成为未来研究的重要焦点。硬件部署、通信延迟、传感器精度等技术细节深刻影响系统实际应用效果,有效开展系统集成与优化以保障行人安全和交通流畅是未来研究不可忽视的关键课题。

\section{结论}

本研究应用强化学习领域的 PPO 算法,基于 CARLA 平台开发行人路径规划与障碍物规避控制系统,通过模型训练及测试验证了系统在复杂环境中的稳定性与避障能力。未来研究将围绕算法改进、传感器算法优化、多智能体合作及实际场景测试展开,以拓展系统在实际应用中的广度与深度,适应更广泛环境。随着技术进步与多学科交叉融合,基于智能算法的行人控制系统在智能交通和智慧城市建设中的重要性愈发凸显,将为城市交通智能化与安全化发展贡献重要价值。​