%!TEX root = ../../csuthesis_main.tex
\chapter{结果与展望}

\section{实验结果总结}
本研究的主要目标是设计并实现一个能够在动态环境中自主避障并完成导航任务的行人控制系统。该系统的核心功能包括路径规划、障碍物避让、控制决策及反馈优化等多个关键模块。为了验证系统的有效性,研究者在Carla仿真平台上进行了多轮详尽的实验。实验的主要结果如下:

在路径规划方面,实验结果清晰表明,系统能够高效计算从起点到终点的最优路径,并有效避开各种静态障碍物。系统采用了如A*算法和Dijkstra算法等先进的路径规划方法,这些算法使系统能够在不同环境下迅速生成既高效又安全的路径。

关于动态避障能力,实验表明,系统在面对动态障碍物(如其他行人和车辆等)时展现出了卓越的避障能力。通过集成传感器(如Lidar和摄像头),系统能够实时获取环境信息,成功检测障碍物并及时调整行进路线,从而确保行人在避开潜在碰撞风险的同时,保持安全行进。

在实时性和适应性方面,系统能够快速响应环境变化,并根据障碍物的动态位置实时调整路径。在多次测试中,系统展现出极高的实时性,能够在不断变化的环境中迅速做出决策,确保行人及时、安全地到达目的地。

在控制精度与稳定性方面,控制决策模块在速度和方向控制上表现出了极高的稳定性。该模块能够根据实时反馈调整行人的运动状态,避免剧烈震荡或偏离目标路线,确保行人的舒适性与安全性。

综合实验结果,本研究得出结论,行人导航系统在复杂多变的环境条件下表现出其有效性、稳定性和适应性。系统不仅能够在不同场景下成功执行导航任务,还能有效避开障碍物,展现出卓越的实时决策能力。

\section{存在的问题与不足}
尽管系统在实验中表现出了令人满意的结果,但经过深入分析,本研究发现仍然存在一些潜在的改进与优化空间。首先,当前使用的路径规划算法,如A*和Dijkstra算法,虽然在大多数情况下能够有效地计算路径,但在处理大规模或复杂环境时,计算速度成为制约系统性能的一个重要因素。随着障碍物数量的增多以及环境复杂度的提升,路径规划所需的时间可能会显著增长,这直接影响了系统的实时响应能力。此外,尽管系统能够有效避开静态和动态障碍物,在某些特别复杂的环境中,避障决策仍然可能受到限制。例如,在障碍物密集或环境极为复杂的情况下,现有的避障策略可能导致路径效率下降,进而影响任务的完成速度和效率。在强化学习的训练过程中,本研究也遇到了一些挑战,尤其是在收敛速度较慢和训练过程中的不稳定性方面。尽管系统经过多次迭代训练能够逐步优化策略,但在一些动态和复杂环境中,训练效果仍有进一步提升的空间。另外,由于计算资源有限,无法获得最佳的参数设置,这对训练效果也产生了一定影响。传感器数据的准确性同样是影响系统性能的关键因素。尽管Carla仿真平台提供了相对精确的传感器数据,但在实际应用中,传感器的误差和噪声可能导致障碍物检测的精度降低,这直接影响了避障决策的效果,进而影响整个系统的性能和可靠性。

\section{未来展望}
尽管本研究已取得一些初步成果,并揭示了一些有趣的现象和潜在规律,但展望未来,本研究仍然意识到许多方面有待改进和深入探索,以期达到更高的学术水平和实际应用价值。在算法优化方面,未来的研究可以着重探索更加高效的路径规划算法,诸如基于深度学习的强化学习路径规划方法,或者采用混合算法,将深度强化学习与传统规划算法结合,从而提升路径规划的效率和精确度。此外,优化避障算法,减少计算时间并增强系统的实时响应能力,也将是未来研究的重要方向。

在多模态传感器融合方面,目前系统主要依赖Lidar和摄像头传感器进行障碍物检测。为了进一步提升系统的障碍物检测能力和精度,未来可以考虑引入更多类型的传感器,例如超声波传感器、雷达等,并结合多模态传感器数据融合技术,以实现更全面、精确的环境感知。另一个值得关注的研究方向是增强现实与仿真环境的结合。尽管本研究已使用Carla仿真平台进行测试,未来可以考虑将增强现实(AR)技术与仿真环境结合,将该系统应用于真实世界的行人导航中。通过更复杂的实际环境和实时数据,可以进一步验证系统在实际应用中的效果和可靠性。

自适应避障与策略优化也是一个值得深入研究的领域。为了更好地应对更加复杂和多变的环境,未来的研究可以专注于自适应避障策略,通过持续的学习和实时反馈,使系统能够根据不同环境条件自动优化避障路径,并提升避障效率。此外,行人行为分析与决策的进一步研究和整合,将是提升行人导航系统智能化水平的关键。通过进一步研究和整合行人行为模型,智能体将能够根据不同环境和目标做出更合理、更智能的行为决策。例如,模拟行人在不同交通规则或社会环境下的行为,将有助于提升行人导航系统的智能化水平。

\section{总结}
本研究精心设计并成功实现了一种先进的行人导航系统,该系统集成了多个关键模块,包括路径规划、障碍物避让、控制决策以及反馈优化等。这些模块协同工作,使得系统能够在各种复杂多变的环境中实现自主避障和精确的目标导航。通过一系列的实验验证,我们发现该系统能够高效地执行路径规划任务,并且在遇到障碍物时能够迅速做出准确的避障反应,确保行人的安全和导航的准确性。

尽管当前的行人导航系统已经展示出良好的性能,但在计算效率、避障策略的多样性和训练过程的稳定性等方面,系统仍然存在一些挑战和局限性。为了应对这些挑战,未来的研究工作将集中于对现有算法进行深度优化,探索更高效的传感器融合技术,以及开发更为智能的自适应避障策略。这些努力将有助于进一步提升系统的整体性能和环境适应能力。随着相关技术的不断进步和成熟,我们有理由相信,行人导航系统在未来将能够在智能城市规划、自动驾驶辅助以及公共安全等多个领域发挥更加重要的作用,为人们的生活带来更多的便利和安全保障。

