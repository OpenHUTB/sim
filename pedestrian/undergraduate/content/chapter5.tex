\chapter{结果与展望}

\section{实验结果}

本研究以 CARLA 模拟平台为实验基础环境,采用先进的强化学习机器学习方法精心训练行人控制模型,旨在实现基于深度学习技术的行人路径规划系统,该系统可有效避让障碍物以提高行人行走安全性与效率。研究在包含多种天气条件、时间段及交通密度等变量的不同环境设置下开展广泛训练和测试工作,确保模型在复杂情况下表现出色。最终获得一系列具有参考价值的实验结果,既验证了方法的有效性,也为未来行人路径规划领域研究提供了宝贵数据支持。

1.\textbf{训练过程与效果:}

本项目基于 CARLA 仿真平台设计实现行人导航系统,运用 PPO、DQN、PID 等多种强化学习算法训练行人移动并开展不同策略对比实验,结果显示基于 PPO 和 DQN 的模型在避障与目标导航任务中性能优异,不同训练阶段行人导航的成功率、路径效率和速度等指标有效提升,PPO 算法使行人在复杂环境中较稳定避障并快速找到目标位置,DQN 模型目标导向性和避障灵活性好但在复杂障碍物环境中响应迟缓,PID 控制策略稳定性高但面对多变障碍物灵活性不足。实验核心是用强化学习算法控制行人在模拟环境行进并避障,基于 PPO 算法训练行人模型并设置多场景模拟现实情况,训练中模型逐渐适应环境变化且多数情况下成功完成起点到目标的导航,PPO 算法适应能力良好,训练中模型提升对障碍物的感知与避让能力和行进效率,数万步训练后行人模型能在给定环境高效避开车辆、行人等动态障碍物到达目标位置。

2.\textbf{模型评估与性能分析:}

评估训练后模型在实际环境中的表现时关注几个关键性能指标:模型需在复杂动态变化环境中迅速规划合理路径并避免不必要绕行停顿,无论障碍物布局简单或复杂均应在合理时间内准确到达目标点;遇行人、车辆等动态障碍物时须展现卓越避障能力,通过持续学习适应依据障碍物具体位置和移动趋势调整行进路线以有效避免碰撞。训练采用 PPO(近端策略优化)算法,其在训练中稳定性高、收敛速度快,几千步训练后模型表现接近最终目标,多数情况下能成功避免碰撞且多次实验均稳定完成既定任务。为全面评估性能设计覆盖简单到复杂各种难度级别的多个测试场景,经实验评估发现训练后的行人模型在复杂城市街道环境和相对简单的开放区域内均保持较高成功率,证明了模型在不同难度下的适应性和鲁棒性。

3.\textbf{实时控制效果:}

在训练好的模型实际仿真测试中验证实时控制效果,实时路径控制的系统依据实时环境反馈信息调整行人行进速度与方向,确保其有效避开各类障碍物并朝既定目标顺利前进。不同实验设置下模型能很好适应环境变化,无论障碍物分布复杂多变还是动态场景频繁切换均保持稳定性能表现。

\section{未来工作与展望}

尽管本研究已经成功地实现了基于强化学习的行人导航控制系统,并且在多个仿真场景中取得了令人满意的实验结果,但仍然存在一些方面需要进一步的完善和提升。未来的研究工作可以从以下几个方向进行深入探索和扩展:

1.\textbf{算法优化与提升:}本项研究中 PPO算法控制效果良好但存在明显局限性,处理复杂交通场景时完成训练并收敛至稳定状态所需时间较长。未来研究可考虑将其与深度 Q 网络、基于模型的强化学习等其他先进强化学习算法结合,以提高训练效率并加快收敛速度。深度 Q 网络和基于模型的强化学习等领域值得探索,这种结合不仅能提升算法在复杂环境中的表现,还可进一步探索混合强化学习算法的可能性以适应更多样化和复杂的场景需求。

2.\textbf{多智能体协作与控制:}目前学术界和工业界研究重点多集中于单一行人导航控制技术,而随着智慧城市与智能交通系统的发展进步,未来研究方向有望拓展至多智能体系统。在多智能体系统中多个行人、车辆及其他智能体间的协作愈发关键,因此研究多人环境中行人与行人、行人与车辆的协作协调机制十分重要。采用多智能体强化学习(MARL)方法模拟不同智能体相互作用并优化其路径选择以提高系统运行效率是一个潜在研究方向。

3.\textbf{传感器与环境感知能力增强:}目前多数研究主要依靠激光雷达和摄像头两种基础传感器进行环境感知与识别,未来研究可考虑引入深度摄像头、红外传感器等更多种类传感器以提升行人智能体在复杂环境中的感知能力。这类先进传感器可在低光照、雨雾雪等恶劣天气条件下提供更准确丰富的环境信息,增强感知能力后行人智能体既能更有效避障、降低碰撞风险,又能获取更精确的环境数据,而这些数据对优化智能体路径规划与决策过程至关重要,可助力其更智能地选择最佳路径,提升整体运行效率与安全性。

4.\textbf{复杂场景下的应用与测试:}当前实验虽覆盖多种环境和障碍物设置但限于仿真场景,未充分测试现实世界复杂多变因素。未来研究致力于将系统应用于现实世界,在城市街道、交通繁忙交叉口及各种公共场所等复杂交通场景对实际存在的障碍物进行详尽测试,并考虑天气条件、照明变化等外部因素对系统性能的影响。实际部署中行人控制系统的实时性和稳定性面临更多挑战,故如何在确保高效导航的同时应对这些复杂外部变化是未来工作的关键。

5.\textbf{集成与系统优化:}随着智能城市与自动驾驶技术的进步普及使得行人导航控制系统需与自动驾驶、交通管理等多种系统集成,这种集成可确保行人在自动驾驶环境中与车辆及其他元素协调共存并成为未来研究的重要焦点。硬件部署、通信延迟、传感器精度等技术细节对系统实际应用效果有深远影响,有效进行系统集成与优化以确保行人安全和交通流畅是未来研究不可忽视的重要课题。

\section{结论}

本项研究工作主要利用了强化学习领域内的一种先进算法,即PPO算法,通过在CARLA模拟平台上进行行人路径规划以及障碍避让控制系统的开发。通过精心设计的模型训练过程和详尽的测试环节,验证了该系统在处理复杂场景时的稳定性和避障能力。展望未来,研究的进一步方向将聚焦于算法的持续优化、传感器性能的进一步提升、多智能体之间的协作机制研究,以及在现实世界场景中的应用测试。这些研究方向的深入探索,旨在进一步增强系统的实际应用能力和普适性,使其能够更好地适应多样化的环境需求。随着技术的不断进步以及跨学科领域的深入合作,基于智能算法的行人控制系统在智能交通系统和智慧城市建设中的重要性将会日益凸显,为城市交通的智能化和安全化贡献巨大的力量。

