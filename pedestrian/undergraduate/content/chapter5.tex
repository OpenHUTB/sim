\chapter{结果与展望}

\section{实验结果}

在本研究中,我们使用了CARLA模拟平台,采用强化学习方法训练行人控制模型,以实现基于深度学习的行人路径规划与障碍避让功能。通过在多种环境设置下进行训练和测试,我们得到了以下几方面的实验结果。

1.\textbf{训练过程与效果:}

本项目基于 CARLA 仿真平台设计实现行人导航系统,运用 PPO、DQN、PID 等多种强化学习算法训练行人移动并开展不同策略对比实验,结果显示基于 PPO 和 DQN 的模型在避障与目标导航任务中性能优异,不同训练阶段行人导航的成功率、路径效率和速度等指标有效提升,PPO 算法使行人在复杂环境中较稳定避障并快速找到目标位置,DQN 模型目标导向性和避障灵活性好但在复杂障碍物环境中响应迟缓,PID 控制策略稳定性高但面对多变障碍物灵活性不足。实验核心是用强化学习算法控制行人在模拟环境行进并避障,基于 PPO 算法训练行人模型并设置多场景模拟现实情况,训练中模型逐渐适应环境变化且多数情况下成功完成起点到目标的导航,PPO 算法适应能力良好,训练中模型提升对障碍物的感知与避让能力和行进效率,数万步训练后行人模型能在给定环境高效避开车辆、行人等动态障碍物到达目标位置。

2.\textbf{模型评估与性能分析:}

训练后模型在实际环境中的表现可通过以下关键指标评估:模型能在复杂动态环境中快速规划路径并避免不必要绕行和停顿,在不同障碍物布局及复杂度下均能于合理时间内到达目标点;面对行人和车辆等动态障碍物时展现强大避障能力,通过持续训练逐渐学会依据障碍物位置和移动方向调整行进路线以避免碰撞;PPO 算法在训练中表现出较高稳定性和较快收敛速度,模型经几千步训练后表现接近最终目标,多数情况下能有效避免碰撞并在多次实验中稳定完成任务。多个测试场景实验评估了模型在不同难度下的表现,训练后的行人模型在复杂城市街道环境和相对简单的开放区域内均保持较高成功率。

3.\textbf{实时控制效果:}

将训练模型应用于实际仿真测试实时控制效果,实时对行人进行路径控制,控制系统根据实时环境反馈调整行进速度和方向,确保行人避开障碍物并朝目标前进,不同实验设置表明模型能适应环境变化,在不同障碍物分布和动态场景中保持稳定表现。

\section{未来工作与展望}

尽管本研究已经实现了基于强化学习的行人导航控制系统,并在多个仿真场景中取得了较好的实验结果,但仍有一些方面需要进一步改进。未来的研究可以从以下几个方向展开:

1.\textbf{算法优化与提升:}PPO 算法在本研究中控制效果较好但存在局限性,其在复杂交通场景中训练收敛时间较长,未来可结合深度 Q 网络(DQN)、基于模型的强化学习(Model-Based RL)等更多先进强化学习算法以提高训练效率和收敛速度,还可探索混合强化学习算法以适应更多复杂场景需求。

2.\textbf{多智能体协作与控制:}当前研究聚焦单一行人导航控制,未来可扩展至多智能体系统,随着智慧城市和智能交通系统发展,多个行人、车辆及其他智能体间协作愈发重要,研究多人环境中行人之间、行人与车辆的协作协调机制是重要方向,例如利用多智能体强化学习(MARL)方法研究不同智能体相互作用并优化路径选择以提升系统整体效率。

3.\textbf{传感器与环境感知能力增强:}现有研究主要依靠激光雷达和摄像头等基本传感器进行环境感知,未来可引入深度摄像头、红外传感器等更多种类传感器,以提升行人智能体在低光照或复杂天气条件下的环境感知能力,增强的感知能力不仅能提升避障效果,还可为行人智能体提供更精确环境数据以优化路径规划和决策。

4.\textbf{复杂场景下的应用与测试:}尽管实验涵盖多种环境和障碍物设置,但仍限于仿真场景,缺乏对现实环境中复杂因素的测试,未来研究致力于将系统应用于现实世界,在城市街道、交叉口、公共场所等各种复杂交通场景中对实际障碍物进行测试,同时考虑天气、照明等外部因素影响,且行人控制系统的实时性和稳定性在实际部署中面临更多挑战,如何在保证高效导航的同时应对复杂外部变化是未来工作关键。

5.\textbf{集成与系统优化:}随着智能城市和自动驾驶技术发展,行人导航控制系统可能需与自动驾驶系统、交通管理系统等其他系统集成,保证行人在自动驾驶系统中的协调共存是未来研究重点之一,此外系统的硬件部署、通信延迟、传感器精度等因素对实际应用效果有重要影响,故系统集成与优化是未来重要研究课题。

\section{结论}

本研究借助强化学习中的 PPO 算法在 CARLA 模拟平台实现行人路径规划与障碍避让控制系统,通过模型训练与测试验证其在复杂场景下的稳定性和避障能力。未来研究将围绕算法优化、传感器性能提升、多智能体协作机制及现实场景应用测试展开,以增强系统实际应用能力和普适性。随着技术进步与跨领域协作,基于智能算法的行人控制系统在智能交通和智慧城市建设中的作用将愈发重要。

