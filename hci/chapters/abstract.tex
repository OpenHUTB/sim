随着人机交互技术向自然化、精准化方向演进,传统交互方式已难以满足复杂场景下的操控需求。肌肉驱动作为直接映射人体生物力学信号的交互范式,为实现本能化人机交互提供了新路径。本文围绕"基于肌肉驱动的生物力学人机交互系统"展开研究,核心目标是构建肌肉驱动与视觉引导协同的方向盘精准控制系统。

首先,本文基于MuJoCo生物力学仿真引擎,设计并实现了包含肩关节、肘关节的人体双臂肌肉骨骼模型,以及具备物理属性的方向盘仿真模型,通过Actuator模块完成肌肉驱动与方向盘关节的精准耦合。其次,提出了肌肉激活度与方向盘转角的非线性映射算法,引入死区过滤与饱和限制机制,平衡控制灵敏度与稳定性;集成视觉引导模块,构建动态目标角度生成与实时反馈机制,提升交互直观性。最后,开发了精度测试系统,通过平均绝对误差(MAE)、均方根误差(RMSE)与控制成功率等指标量化评估系统性能。

实验结果表明,所设计的系统可实现肌肉驱动的方向盘稳定控制,视觉引导反馈响应及时。系统平均绝对误差(MAE)为2.22°,均方根误差(RMSE)为2.88°,控制成功率(误差<5°)达93.3\%,满足精准交互需求。该系统为驾驶模拟、工业控制等场景提供了新的交互方案,相关建模方法与精度评估体系可为同类生物力学人机交互系统的开发提供参考。