\subsection{研究背景及意义}

\subsubsection{选题背景}
在人机交互技术从"工具适配人"向"人适配本能"的演进过程中,传统依赖机械按键、触控屏的交互方式逐渐暴露局限性:一方面,复杂场景(如驾驶模拟、工业远程操控)中,机械交互的延迟与操作门槛难以满足精准、实时的控制需求;另一方面,现有基于肌电信号(EMG)的肌肉驱动系统,普遍依赖物理传感器采集信号,存在成本高、易受环境干扰、适配性差等缺陷。

生物力学人机交互作为自然交互的核心方向,通过模拟人体肌肉收缩与关节运动的生理机制,可实现"本能化"的设备操控。当前,MuJoCo等生物力学仿真引擎的成熟,为构建高精度肌肉骨骼模型提供了技术支撑——其支持多自由度关节建模、肌肉-力矩耦合仿真,可在虚拟环境中复现人体运动的生物力学特性。但现有研究存在三个核心痛点:一是肌肉驱动信号与实际设备(如方向盘)的耦合精度不足;二是缺乏视觉引导与肌肉驱动的协同控制机制;三是控制性能的量化评估体系不完善。

在此背景下,针对方向盘这一典型操控设备,构建"肌肉驱动+视觉引导"的协同交互系统,既符合人机交互自然化的发展趋势,也能为驾驶辅助、康复医疗等领域提供低成本、高适配的交互方案。

\subsubsection{选题意义}
本研究的价值在于,通过构建基于MuJoCo的肌肉骨骼-方向盘耦合仿真系统,实现了肌肉激活度向设备操控量的精准映射,同时集成视觉引导模块提升交互直观性,既丰富了生物力学人机交互的技术路径,也为复杂场景下的精准操控提供了新的实现方案;所设计的MAE、RMSE与控制成功率量化评估方法,为同类系统的性能测试提供了标准化参考;而基于开源工具链的无硬件依赖实现,既降低了肌肉驱动交互的应用成本,也可为高校相关课程提供可复现的实验案例,兼具技术探索、应用落地与教学实践的多重价值。

\subsection{国内外研究现状}
肌肉驱动人机交互技术作为自然交互领域的前沿方向,其核心是通过映射人体肌肉的生物力学信号实现设备操控,目前国内外研究已在模型构建、算法设计、场景应用等维度取得阶段性进展,但在"模型-设备耦合""多模态协同""性能量化"等方面仍存在技术缺口。

\subsubsection{国外研究现状}
国外在肌肉驱动人机交互领域的研究起步于21世纪初,目前已形成"仿真建模-算法映射-系统落地"的完整技术链:

在肌肉骨骼仿真建模层面,以MIT的Todorov团队为代表,其开发的MuJoCo生物力学引擎突破了传统物理引擎在肌肉-关节耦合模拟上的局限,支持多自由度人体模型的精准构建——该团队2012年提出的"肌肉激活度-关节力矩映射模型",可复现人体上肢90\%以上的运动姿态,已成为当前生物力学仿真的主流框架;斯坦福大学后续基于MuJoCo扩展了肌肉疲劳、力反馈等生理特性模拟,进一步提升了模型的真实性。

在核心算法与系统落地层面,User-in-the-Box实验室的研究最具代表性:其2020年推出的"肌肉驱动远程操控框架",通过将人体肌肉信号映射为机械臂的关节控制量,实现了毫米级精度的远程装配作业;2022年该团队拓展了驾驶模拟场景,尝试通过肌肉信号控制方向盘,但该系统依赖专用肌电采集设备,硬件成本超过5万元,且未集成视觉引导机制,普通用户的操控误差超过15°。

此外,德国慕尼黑工业大学聚焦于"肌肉驱动与虚拟环境的融合",开发了基于VR的肌肉驱动训练系统,但该系统仅用于运动康复训练,未涉及实际设备的操控。

整体而言,国外研究的优势在于模型精度与算法成熟度,但存在硬件依赖强、场景适配性弱的问题,针对低成本、高普适性的民用场景(如普通驾驶模拟)的研究较少。

\subsubsection{国内研究现状}
国内对肌肉驱动人机交互的研究始于2010年后,主要聚焦于工业、医疗等细分领域的应用落地,研究方向呈现"场景导向型"特征:

在工业领域,哈尔滨工业大学机器人研究所2018年开发了基于肌电信号的机械臂肌肉驱动系统,通过采集前臂肌肉的EMG信号,实现了机械臂的6自由度操控,控制精度可达±2mm,但该系统需要在用户手臂粘贴4个以上的传感器,且易受环境电磁干扰,在工业现场的稳定性不足;东南大学2021年对该技术进行优化,采用无线传感器降低了布线成本,但信号延迟仍超过100ms,难以满足实时控制需求。

在医疗领域,北京航空航天大学康复工程研究所2019年推出了上肢康复外骨骼的肌肉驱动系统,通过识别患者的肌肉收缩意图,辅助完成肘关节屈伸训练,临床实验显示该系统可提升康复效率30\%,但系统仅支持预设动作的辅助,不具备自由操控能力;上海交通大学2022年拓展了该系统的功能,加入了视觉反馈模块,但反馈界面仅显示关节角度数据,未实现"视觉引导与肌肉驱动的协同控制"。

此外,国内高校在肌肉驱动的仿真建模层面,多依赖国外的MuJoCo引擎,自主开发的模型框架较少,且研究集中于单一技术环节(如信号采集、算法映射),缺乏完整的系统集成。

\subsubsection{研究现状总结}
综合国内外研究进展可见:当前肌肉驱动人机交互技术已实现从"理论建模"到"场景落地"的突破,但仍存在三个核心技术缺口:

一是模型与设备的耦合精度不足:现有系统多聚焦于肌肉信号到关节运动的映射,缺乏针对方向盘等特定设备的定制化耦合设计,操控误差普遍超过10°;

二是多模态协同机制缺失:多数系统仅依赖肌肉信号或视觉反馈单一模态,未实现两者的协同优化,用户操控的学习门槛较高;

三是性能量化体系不完善:现有研究的评估指标多为"控制精度""响应延迟"等单一维度,缺乏涵盖"精度-稳定性-易用性"的综合量化体系。

\subsection{研究内容与技术路线}

\subsubsection{研究内容}
本研究围绕"基于肌肉驱动的生物力学人机交互系统"的核心目标,聚焦方向盘精准控制场景,从仿真建模、算法设计、系统集成、性能测试四个维度展开,具体研究内容如下:

(1)肌肉骨骼与方向盘耦合仿真模型构建

基于MuJoCo生物力学仿真引擎,构建高精度人体双臂肌肉骨骼模型与方向盘物理模型。其中,人体模型需包含躯干、肩关节、肘关节等关键结构,配置符合人体生理特性的肌肉参数(如肌肉长度、收缩力、阻尼系数)与关节约束(如转动范围、力矩限制);方向盘模型需还原真实物理属性(如半径、重量、转动惯量),通过Actuator模块实现肌肉驱动与方向盘关节的刚性耦合,确保肌肉收缩信号可直接转化为方向盘的转角运动,为后续控制算法提供高保真仿真环境。

(2)肌肉驱动与方向盘转角的映射算法设计

针对肌肉激活度与方向盘转角的非线性关系,设计自适应映射算法。首先,通过正弦/余弦函数模拟人体双臂肌肉的动态激活度(左臂对应左转、右臂对应右转),还原真实发力逻辑;其次,引入增益系数、死区过滤与饱和限制机制增益系数用于调节方向盘转动灵敏度,死区过滤用于抑制信号噪声导致的抖动,饱和限制用于约束最大转角(±90°),平衡控制精度与稳定性;最后,通过迭代实验优化算法参数,实现肌肉信号到方向盘转角的精准、平滑映射。

(3)视觉引导与实时反馈模块开发

构建"动态目标-视觉引导-实时反馈"的协同机制。设计动态目标角度生成逻辑,通过周期性调整目标转角(范围±90°),模拟真实场景下的方向盘操控需求;开发可视化引导界面,在仿真窗口中实时显示目标角度、实际角度、误差值等关键信息,为用户提供直观的视觉参考;优化反馈响应机制,确保视觉信息更新频率与仿真步长(0.005s)同步,降低交互延迟,提升操控的沉浸感与准确性。

(4)系统集成与多维度性能测试

基于Python完成仿真模型、映射算法、视觉引导模块的全流程集成,开发可独立运行的人机交互系统。设计多维度性能测试方案:功能测试验证模型运动、算法映射、视觉反馈的完整性;精度测试通过平均绝对误差(MAE)、均方根误差(RMSE)量化转角控制精度;稳定性测试在连续运行1小时场景下评估系统无故障运行能力;易用性测试通过操控学习曲线验证视觉引导对降低操作门槛的作用。同时,基于测试结果迭代优化模型参数与算法逻辑,提升系统综合性能。

\subsubsection{技术路线}
本研究遵循"需求分析-方案设计-开发实现-测试优化"的技术流程,具体路线如下:

(1)第一阶段:需求分析与基础准备(第1-4周)

明确系统功能需求(肌肉驱动控制、视觉引导、精度测试)与性能指标(MAE≤3°、响应延迟≤10ms);

学习MuJoCo仿真引擎的建模原理、Python编程及相关库(NumPy、Pandas、Matplotlib)的使用;

查阅生物力学人机交互、视觉引导控制相关文献,确定模型参数与算法设计依据。

(2)第二阶段:仿真模型与核心算法开发(第5-9周)

基于MuJoCo XML语法,编写人体双臂肌肉骨骼模型与方向盘物理模型的配置文件,完成关节、肌肉、几何形状等参数的初始化;

实现肌肉激活度模拟生成函数,设计包含增益调节、死区过滤、饱和限制的非线性映射算法;

搭建基础仿真环境,验证模型运动与算法映射的正确性,初步优化参数(如增益系数、死区阈值)。

(3)第三阶段:视觉引导模块开发与系统集成(第10-12周)

开发动态目标角度生成模块,通过周期性调整目标值模拟真实操控场景;

基于MuJoCo Viewer API,设计可视化引导界面,实时显示目标角度、实际角度、误差曲线等信息;

完成仿真模型、映射算法、视觉引导模块的集成,解决模块间的数据传输与同步问题,确保系统流畅运行。

(4)第四阶段:系统测试与迭代优化(第13-15周)

设计功能测试用例,验证模型运动、算法映射、视觉反馈等核心功能的完整性;

开展精度测试、稳定性测试、易用性测试,记录实验数据(时间戳、目标角度、实际角度);

基于测试结果,优化模型参数(如肌肉阻尼系数、方向盘转动惯量)与算法逻辑(如增益系数动态调整),提升系统性能;

整理实验数据,计算MAE、RMSE、控制成功率等核心指标,形成测试报告。

(5)第五阶段:成果整理与论文撰写(第16-18周)

整理系统源代码、仿真模型配置文件、实验数据文件,形成完整的技术成果包;

撰写毕业论文,涵盖绪论、相关技术基础、系统设计、实现过程、测试结果等章节;

优化论文结构与内容,补充图表(如系统架构图、仿真界面截图、精度测试曲线),准备答辩。

\subsection{论文结构与章节安排}
本文共分为六章,具体结构安排如下:

第一章为绪论,主要介绍研究背景、意义、国内外研究现状以及研究内容与技术路线。

第二章为相关技术基础,详细介绍MuJoCo生物力学仿真引擎、Python相关技术库以及人机交互控制理论基础。

第三章为系统总体设计,包括需求分析、架构设计以及关键技术方案。

第四章为系统详细实现,具体阐述仿真模型、核心算法、视觉引导模块以及数据处理模块的实现过程。

第五章为系统测试与结果分析,通过功能测试和性能测试验证系统有效性,并进行结果分析。

第六章为总结与展望,总结研究成果,指出创新点与不足,展望未来研究方向。