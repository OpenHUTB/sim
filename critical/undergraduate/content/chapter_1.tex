\section{绪论}

\subsection{研究背景与意义}
随着人工智能与自动驾驶技术的飞速发展,自动驾驶系统正逐步从封闭测试环境走向复杂多变的真实道路场景。自动驾驶技术的核心目标是实现安全、高效的出行,其可靠性与鲁棒性直接关系到驾乘人员的生命安全及交通系统的稳定运行。然而,真实道路环境中存在大量低概率、高风险的极端驾驶场景(又称"边缘案例"),这类场景往往因环境干扰、交通参与者行为突变等因素导致驾驶风险陡增,成为制约自动驾驶系统落地应用的关键瓶颈。

现有自动驾驶测试主要依赖真实道路测试、场地测试及仿真测试三种方式。真实道路测试虽能反映实际场景,但存在测试周期长、成本高、危险场景复现难度大等问题;场地测试受限于场地规模与场景覆盖度,难以穷尽各类极端情况;而仿真测试凭借可重复性强、测试效率高、成本可控等优势,已成为自动驾驶系统研发与验证的核心手段。CARLA(Car Learning to Act)作为一款开源的高级自动驾驶仿真平台,具备高保真的环境建模、灵活的场景配置及完善的API接口,能够为极端驾驶场景的构建与测试提供可靠的技术支撑。

极端驾驶场景的生成与测试是验证自动驾驶系统应急处理能力的核心环节。当前,仿真测试中普遍存在场景覆盖不全面、危险场景生成缺乏系统性、场景真实性与多样性不足等问题,导致自动驾驶系统在面对突发危险时易出现决策失误或控制失效。因此,基于CARLA仿真平台设计科学合理的危险场景分类体系,研发高效的场景生成算法,实现对极端驾驶场景的系统性构建与量化评估,不仅能够填补现有仿真测试中危险场景覆盖的空白,为自动驾驶系统提供全面、严苛的测试环境,还能推动自动驾驶技术在复杂危险场景下的性能优化,加速其商业化落地进程,具有重要的理论研究价值与工程应用意义。

\subsection{国内外研究现状}

\subsubsection{自动驾驶仿真平台发展现状}
国外在自动驾驶仿真平台领域起步较早,已形成一批功能成熟、应用广泛的产品与开源项目。除CARLA外,百度Apollo的Apollo Simulator、谷歌的Waymo Simulator、德国大陆集团的VTD(Virtual Test Drive)等均具备高保真的仿真能力。CARLA凭借其开放的架构设计、支持多传感器模拟(如激光雷达、摄像头、雷达等)及灵活的场景编辑功能,被广泛应用于自动驾驶算法训练与测试。其客户端-服务器架构支持多智能体协同仿真,能够模拟复杂的交通流与道路环境,为危险场景的构建提供了良好的基础。此外,国外研究多聚焦于仿真平台的真实性提升,如通过高精度地图建模、物理引擎优化等方式增强场景与真实世界的一致性。

国内自动驾驶仿真平台的研发近年来也取得了显著进展,除百度Apollo Simulator外,华为的ADS Simulator、商汤科技的 SenseSim等平台均已具备一定的行业应用规模。国内研究更注重结合中国道路法规与交通场景特点,如针对混合交通流(行人、非机动车与机动车混行)、特殊道路标志标线等场景的仿真建模。但总体而言,现有仿真平台在危险场景的系统性设计、动态生成能力及场景量化评估方面仍存在不足,难以满足自动驾驶系统对极端场景测试的需求。

\subsubsection{危险场景定义与分类方法}
目前,国内外关于危险驾驶场景的定义尚未形成统一标准,主流观点认为危险场景是指可能导致交通事故发生、对驾驶安全构成严重威胁的场景,其核心特征包括突发性、不确定性及高风险关联性。在分类方法上,现有研究主要从不同维度展开:基于场景触发因素,可分为环境因素主导型(如恶劣天气、道路施工)、交通参与者行为主导型(如前车急刹、行人鬼探头)及设备故障主导型(如车辆故障停车);基于道路类型,可分为城市道路危险场景、高速公路危险场景、乡村道路危险场景等;基于风险等级,可分为一般危险场景、严重危险场景与极端危险场景。

国外研究在危险场景分类上更注重场景的量化定义,如通过碰撞概率、制动距离等参数界定场景危险程度。例如,欧盟的ADASIS(Advanced Driver Assistance Systems Interface Specification)标准中,对前车紧急制动、车道偏离等危险场景的参数阈值进行了明确规定。国内研究则更侧重结合中国道路交通的特殊性,如针对路口抢行、非机动车违规穿行等典型场景的分类与定义。但现有分类体系普遍存在场景覆盖不全面、分类维度单一、缺乏对场景要素的系统性解构等问题,难以支撑极端驾驶场景的规范化生成与测试。

\subsubsection{场景生成技术研究进展}
自动驾驶仿真场景的生成技术主要分为三类:手动生成方法、规则驱动生成方法与数据驱动生成方法。手动生成方法通过仿真平台的可视化编辑工具或API接口,手动配置场景要素(如车辆位置、速度、环境参数等),该方法操作简单、场景可控性强,但效率低下,难以生成大规模、多样化的危险场景,适用于典型场景的验证与演示。

规则驱动生成方法基于预设的场景规则与参数约束,通过编写脚本自动生成场景,例如基于交通流理论设计车辆跟驰、换道等行为规则。该方法能够批量生成场景,但规则的设计依赖领域专家经验,灵活性不足,难以覆盖复杂多变的极端场景。

数据驱动生成方法是近年来的研究热点,主要包括基于强化学习、生成对抗网络(GAN)等深度学习技术的场景生成。其中,强化学习通过构建智能体与环境的交互机制,使智能体在探索过程中生成具有对抗性的危险场景,能够有效提升场景的多样性与严苛性。例如,部分研究将自动驾驶车辆作为被测试方,通过强化学习训练对抗智能体(如其他车辆、行人),使其产生能够挑战被测试方性能的行为(如突然加塞、逆行等)。但现有数据驱动生成方法仍存在场景真实性不足、训练过程复杂、生成效率偏低等问题,且缺乏对生成场景的系统性量化评估。

\subsection{研究目标与主要内容}

\subsubsection{研究目标}
本研究以CARLA仿真平台为基础,围绕极端驾驶场景的分类、生成与量化评估展开研究,旨在实现以下目标:

构建一套全面、系统的危险驾驶场景分类体系,覆盖不同道路类型、环境条件与交通参与者行为,明确各类场景的核心要素与参数特征;

研发基于手动控制与强化学习的极端驾驶场景生成算法,实现对至少10种典型危险场景的高效生成,兼顾场景的真实性、多样性与严苛性;

设计科学合理的量化评估指标体系,对生成场景的安全性、质量与生成效率进行全面评估,验证所提方法的有效性与实用性。

\subsubsection{主要内容}
为实现上述研究目标,本研究的主要内容包括:

危险驾驶场景分类体系构建:结合现有研究成果与中国道路交通特点,从主体行为、环境干扰、交通要素三个维度设计分类框架,定义前车紧急制动、路口抢行冲突、恶劣天气超车等10种典型危险场景的具体特征与参数范围,并建立场景的形式化表示模型;

极端驾驶场景生成算法设计与实现:
\begin{itemize}
    \item 基于CARLA的Python API,设计手动控制与规则驱动的场景生成方法,通过编写脚本配置场景要素,实现典型危险场景的快速生成与复现;
    \item 提出基于强化学习(DDPG算法)的对抗性场景生成方法,构建状态空间、动作空间与奖励函数,训练对抗智能体生成具有挑战性的极端驾驶场景,提升场景的多样性与严苛性;
\end{itemize}

生成场景量化评估:设计安全指标(如碰撞风险系数、制动距离达标率)、场景质量指标(如场景真实性得分、多样性指数)与生成效率指标(如场景生成耗时、资源占用率),通过对比实验分析不同生成方法的优劣,验证生成场景的有效性与实用性。

\subsection{论文组织结构}
本论文共分为六章,各章节的组织结构如下:

第一章绪论:阐述研究背景与意义,综述国内外在自动驾驶仿真平台、危险场景分类与场景生成技术方面的研究现状,明确研究目标、主要内容与论文组织结构;

第二章相关理论与技术基础:介绍CARLA仿真平台的架构、传感器模型与API接口,界定极端驾驶场景的理论定义与场景要素解构方法,阐述强化学习的基础理论及其在驾驶场景生成中的适用性;

第三章基于CARLA的极端驾驶场景分类体系构建:详细介绍分类维度设计,定义10种典型危险场景的具体特征,建立场景的参数化描述框架与关键参数取值范围;

第四章极端驾驶场景生成算法设计与实现:设计整体生成框架,分别实现基于手动控制与规则的场景生成方法和基于强化学习的对抗场景生成算法,说明软件模块划分、CARLA环境配置与代码组织结构;

第五章实验结果分析与量化评估:介绍实验的硬件与软件环境、基准测试方案及测试用自动驾驶模型,展示生成场景的效果,基于设计的评估指标进行量化分析与对比讨论;

第六章总结与展望:总结本研究的主要工作与成果,分析研究过程中存在的局限性,对未来的研究方向进行展望。