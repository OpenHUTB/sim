%!TEX root = ../csuthesis_main.tex
% 设置中文摘要
\keywordscn{智能驾驶\quad 仿真场景\quad 危险场景生成\quad NSGA-II 多目标优化}
%\categorycn{TP391}
\begin{abstractzh}

在智能驾驶系统的发展过程中,仿真场景的生成与优化是保障其安全性和可靠性的关键技术手段。本文首先基于自然驾驶数据提取具有代表性的危险场景,构建真实有效的仿真数据基础;随后,采用多维场景融合方法识别典型行车行为(如变道、跟车、邻车切入等),并与动态交通要素结合,生成更加复杂和真实的测试场景。针对现有测试环境中高风险场景覆盖率不足的问题,本文重点引入 NSGA-II 多目标优化算法,通过非支配排序与拥挤度计算,在“最小安全距离”和“碰撞风险”两个目标之间实现 Pareto 最优平衡,有效筛选出多样且具有代表性的高危场景。实验结果表明,与传统随机搜索方法相比,NSGA-II 可将高风险场景覆盖率提高 30\% 以上。最后,设计并实现了一套自动化仿真测试平台,集成场景生成、仿真执行、数据采集与结果分析功能,实现测试流程的自动化和标准化。本文方法显著提升了智能驾驶系统在仿真环境中的安全性验证能力,为未来自动驾驶系统的开发与测试提供了有力技术支撑。


\end{abstractzh}