With the commercialization of autonomous driving technology, extreme driving scenario simulation testing has become a key link in verifying system safety and robustness. Existing simulation tests face issues such as incomplete coverage of dangerous scenarios, lack of systematic generation, and insufficient authenticity, which restrict the iterative optimization of autonomous driving technology. This study, based on the CARLA simulation platform, focuses on the classification, generation, and quantitative evaluation of extreme driving scenarios.

First, a dangerous scenario classification system with three dimensions (subject behavior, environmental interference, and traffic elements) was constructed, defining the characteristics and parameter ranges of 10 typical extreme driving scenarios such as sudden braking of preceding vehicles and intersection conflicts. Second, a dual-mode scenario generation algorithm was designed: typical scenarios are quickly reproduced through manual control and rule-driven methods, while complex adversarial dangerous scenarios are generated using the Deep Deterministic Policy Gradient (DDPG) reinforcement learning algorithm. Finally, a quantitative evaluation system covering safety indicators, scenario quality indicators, and generation efficiency indicators was established to comprehensively verify the generated scenarios.

Experimental results show that the proposed classification system has good systematicness and coverage, the generation algorithm can efficiently produce diverse and highly authentic extreme driving scenarios, and the quantitative evaluation results can effectively support the judgment of scenario quality and algorithm performance. This study provides standardized scenario resources and efficient generation methods for extreme scenario testing of autonomous driving systems, which is of great significance for promoting the safe implementation of autonomous driving technology.

\textbf{Keywords:} Autonomous driving; CARLA simulation platform; Extreme driving scenarios; Scenario generation; Reinforcement learning; Quantitative evaluation