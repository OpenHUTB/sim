%!TEX root = ../../csuthesis_main.tex
\chapter{危险仿真场景生成实验研究准备}

\section{实验准备与数据采集}
\subsection{实验准备}

ChatScene:

在用户用自然语言描述危险场景需求的时候,当前领域专用LLM的解析能力还能进一步提升,可以引入迁移学习技术,利用大量自动驾驶领域相关的文本数据对模型做微调,让它对专业术语和复杂语义的理解变得更加精准,就像对于“在高速公路上遇到团雾且前方车辆紧急制动”这样的描述,模型能够更准确地提取出里面的关键信息,并且生成相应的场景参数,建立语义纠错和完善机制也是非常重要的,当用户输入模糊或者不完整的描述的时候,系统可以通过主动提问的方式和用户进行交互,以此获取更多的细节信息,比如用户仅仅输入“生成夜间场景”,系统可以询问“是否有特定的道路类型”“是否包含特殊的交通参与者行为?” 等问题,从而完善场景描述,提高生成场景的准确性和实用性。​


多模态对齐方面结合DetDiffusion的感知增强生成技术,在把文本指令映射为多视图场景数据时探索更多先进的生成模型,比如引入Transformer - based的生成模型,利用其强大长序列建模能力更好处理文本与多模态数据间复杂关系来进一步提升几何精度,为更直观展示几何精度的提升效果通过实际案例进行对比分析,选取不同复杂度的场景描述分别用原方法和改进后方法生成多视图场景数据,计算Chamfer Distance误差并绘制误差对比图表,同时展示生成的多视图场景图像,直观呈现改进后在场景细节和几何准确性上的优势,\cite{zhang2024chatscene}。​


约束嵌入就是在通过三路组合方法生成场景等价类的时候能够优化约束条件的设定,基于历史事故数据以及交通规则来构建更精细的约束规则库,比如在考虑道路结构和交通参与者行为的组合时要根据不同道路类型像城市道路、高速公路等设定不同约束条件以避免生成不符合实际交通规则和安全逻辑的场景组合,此外还可以采用并行计算技术来进一步提升生成效率,把场景生成任务分解成多个子任务利用多核处理器或者分布式计算平台并行处理从而大幅缩短单位时间内场景生成的时间,同时通过实时监控生成过程动态调整计算资源分配以确保生成效率达到最大化。

项目ASIL-Gen:

动态安全等级映射是在依据场景风险参数动态分配ASIL等级的时候,可引入机器学习模型来进行更准确的风险评估,需要收集大量实际场景数据,其中包含已发生事故的场景以及模拟测试场景,提取相关风险参数当作特征,以实际的ASIL等级作为标签来训练一个分类或回归模型,比如可以使用随机森林、支持向量机或者深度学习中的神经网络模型,通过模型学习风险参数与ASIL等级之间的复杂关系,从而实现更精准的动态安全等级映射,同时要建立ASIL等级动态调整机制,随着自动驾驶技术的发展以及新的安全需求的出现,定期更新风险评估模型和ASIL等级映射规则,例如,当出现新的传感器技术或自动驾驶功能时,重新评估相关风险参数对 ASIL 等级的影响,确保安全评估的时效性和准确性。​


五维安全验证是参考华为CAS 4.0体系来做的,进行五维安全验证时可针对每个维度开展更深入测试,全时速测试方面,除现有的速度范围之外,还可增加极端速度条件下的测试,像超低速蠕行以及接近车辆最高时速的高速行驶场景,以此评估自动驾驶系统在极端速度下的安全性,全方向感知测试当中,要引入更复杂的多目标交互场景,模拟多个交通参与者从不同方向同时接近的情况,测试自动驾驶系统的感知和决策能力,全目标识别部分,除了行人和车辆以外,要增加对非机动车、动物、道路障碍物等特殊目标的识别和应对测试,全天候测试要结合气象模拟技术,生成更逼真的极端天气场景。如强台风、沙尘暴等,评估系统在恶劣天气下的安全性\cite{fitch2012using}。​


形式化验证工具以Pro - SiVIC作为基线,结合MTGS多轨迹数据融合技术保证物理真实性时,能够拓展数据融合的范围和方式,除了车道宽度、护栏间距这类几何数据之外,还融合车辆动力学数据、交通流数据等,以此构建更全面的物理模型,比如通过实时采集真实道路的交通流数据,将其融入模拟场景里,让场景中的车辆行驶行为更贴合实际交通情况,同时利用虚拟现实(VR)和增强现实(AR)技术,对验证结果进行可视化展示,开发专门的可视化工具,把模拟场景以VR或AR的形式呈现出来,测试人员能够沉浸式体验和评估场景的物理真实性与安全性,提升验证的直观性和有效性。

\subsection{数据准备}
测试场景库构建
场景类型:基于ChatScene生成的13类危险场景(如行人横穿、车辆抢行、障碍物突现)。
数量分配:每类场景生成100个变体,共1300个测试用例,覆盖不同天气(雨天、夜间)、交通密度。
场景数据:在公平对比基线实验中,使用scenario/scenario\_data/scenic\_data中的场景文件,这些文件有手动修改以确保使用相同的 10 条路线进行公平对比。在动态模式下,将场景描述文本写入retrieve/scenario\_descriptions.txt,运行python retrieve.py获取对应的 Scenic 代码。
模型配置:默认使用scenario/agent/config/adv\_scenic.yaml中的配置,从 Safebench-v1 加载预训练的 RL 模型控制 ego agent,周围的对抗 agent 由 Scenic 控制。

指标采集方法碰撞率是借助CARLA内置的碰撞传感器也就是sensor.other.collision来记录碰撞事件,成功完成率的条件是要同时满足无碰撞、无闯红灯、无越界也就是车道保持误差小于0.5米,并且要在时间限制内到达目标点比如交叉路口场景限时30秒,检测工具是基于高德MapDR规则引擎实时校验交通违规行为,平均决策时间是在RL代理的决策循环中嵌入时间戳记录即time.perf\_counter(),排除感知模块耗时只统计从状态输入到控制输出的计算延迟,恢复能力方面诱导偏离是在场景中注入噪声比如突然横向风力、GPS信号丢失,恢复判定是车辆在5秒内返回原车道且速度稳定波动小于10 。

预生成场景数据:解压Scenario Dataset/Scenario\_Dataset\_1-4.zip,将其中的.py场景文件复制到scenario\_runner/srunner/scenarios/目录,.xml配置文件复制到

scenario\_runner/srunner/examples/目录,以便利用预生成的场景进行实验。
自定义场景数据(可选):利用Scenario Generation Scripts/目录下的脚本生成新的场景变化,如执行python script\_change\_lane.py生成新的变道场景,满足特定实验需求。

\section{危险仿真场景提取的特征参数确定}

很多因素变化会让驾驶员在行驶过程中判定某个场景是危险的,包括车辆因素、环境因素、复杂多变的交通流等。但是将所有因素都列为判断场景是否危险是不切实际的。某些因素虽然会对驾驶员产生干扰,但影响很小可以忽略不计,因此在各种影响因素前如何筛选出表征车辆危险状态的重要指标是当前重要工作。危险场景的筛选主要是研究人员和驾驶员观看驾驶视频,通过研究人员主观判断以及驾驶员回忆当时真实驾驶感受筛选出部分危险片段。当驾驶员在行车过程中,制动操作是用于区分驾驶车辆间的安全和危险状态的判断标准之一。在驾驶人行车时,前车初始制动时刻或前车制动灯亮起可以认为是驾驶员还处于正常的安全跟车状态。一旦前车紧急制动快速缩短两车之间的驾驶距离时,该场景下的危险系数逐渐增加,驾驶人会根据车辆间的状态变化来调整自车的运行状态,开始制动操作或者打方向盘改变车道,以避免陷入危险碰撞事件。此时,自车驾驶员开始制动或者打方向盘时刻即为危险开始时刻。通过上述分析,跟车时驾驶员的安全和危险状态判断可被视为二分类。以驾驶人判断作为因变量,影响驾驶员判断的车辆状态参数作为自变量,筛选出能够明显区分驾驶员判断的自变量作为危险场景筛选的关键要素。前车制动紧急程度一定程度上影响驾驶员的操作反应,随着前车制动减速带来的场景变化是两车之间的相对距离逐渐减小。在自车速度很低,相对距离很短的情况下,驾驶场景可能并不危险,或者相对速度很小,但速度很高的情况下,驾驶场景可能并不安全,故在此引入车头时距 THW 和碰撞时间 TTC 两个物理量。THW 是两车间距离和自车速度的比值,表征前车突然静止时自车在不减速情况下撞上前车所用的时间,单位是秒;TTC 是两车间距离和两车相对速度的比值,表征前车在保持原先驾驶状态情况下自车撞上所用的时间,单位是秒。在此相对速度是前车速度减去自车速度,自车速度大于前车速度才有可能发生碰撞,故相对速度为负值,计算出的 TTC 也为负值。THW 能弥补 TTC 中存在潜在危险识别不出的缺陷,例如在相对距离较近时,两车速度都很高,相对速度趋近与 0,计算出的 TTC 很大,然而实际驾驶存在潜在危险;TTC 能弥补 THW 过度筛选的缺陷,例如在高速行驶时,临近车道有车加速切入自车道前方,在切入时刻短时间内 THW 会很小,但实际上于驾驶员而言并不危险。因此,下文将自车纵向减速度、THW 和 TTC 作为提取危险场景的特征参数,并结合国外参考文献中用到的车辆横摆角速度和横向加速度的阈值共同提取场景\cite{杨敏明2017基于自然驾驶实验的驾驶行为研究}。

\subsection{场景提取参数确定}

在危险场景提取方法的研究过程里,场景提取参数的确定属于关键步骤之一,它会直接影响提取结果的准确性与可靠性,下面是关于参数确定关键点和研究方向的总结以供参考,参数确定的核心目标包括精准性,即参数要能够有效区分危险场景和非危险场景,鲁棒性指的是参数在不同数据集或者环境变化时要保持稳定,可解释性表示参数需和危险场景的物理或行为特征存在强相关性 。

参数确定的关键步骤:
(1) 数据驱动的参数初选
数据特征分析:通过统计分析(如分布、相关性)确定候选参数。
例如:碰撞时间(TTC)、最小安全距离、加速度突变、行为异常频率等。
领域知识融合:结合交通规则、事故报告或行业标准筛选参数(如ISO 26262中对功能安全的要求)。
(2) 参数阈值设定
统计方法:基于历史数据的分位数(如95%分位数)或极值理论(EVT)设定阈值。
机器学习:通过监督学习(如SVM、决策树)划分危险/非危险类别边界。
动态阈值:针对不同场景(如天气、道路类型)自适应调整阈值。
(3) 参数优化与验证
敏感性分析:评估参数变化对结果的影响(如蒙特卡洛模拟)。
多目标优化:平衡参数间的冲突(如灵敏性与误报率)。
交叉验证:通过K折交叉验证或留出法验证参数的泛化能力。


\begin{table}[htb]
	\centering
	\caption{典型场景参数的分类}
	\label{T.example}
\begin{tabular}{lll}
	\hline
	参数类型 & 示例  & 适用场景 \\
	\hline
	时间相关 & 碰撞时间(TTC)、反应时间 & 车辆避撞、行人交互 \\
	\hline
	空间行为 & 最小安全距离、车道偏离率 & 车道保持、超车场景 \\
	\hline
	行为相关 & 急加速/急减速、转向角变化 & 驾驶员异常行为检测 \\
	\hline
	环境相关 & 光照条件、能见度、路面摩擦系数 & 恶劣天气下的危险场景 \\
	\hline
\end{tabular}
\end{table}


参数确定中的挑战数据不均衡:危险场景数据稀疏,需通过过采样(SMOTE)或生成对抗网络(GAN)增强数据。多参数耦合:参数间可能存在非线性关系,需采用主成分分析(PCA)或因果推理方法解耦。实时性要求:参数计算复杂度需适应实际系统的实时处理能力(如边缘计算场景)。
该段落可能为AI生成的概率为:89.8%
前沿研究方法强化学习(RL)通过环境交互动态调整参数,例如在自动驾驶中优化紧急制动阈值。贝叶斯优化基于概率模型高效搜索最优参数组合,减少实验成本。可解释 AI(XAI)利用 SHAP值或 LIME方法解释参数对危险场景分类的贡献度。联邦学习在保护数据隐私的前提下,跨多源数据联合优化参数。
该段落可能为AI生成的概率为:85.1%
验证与评估指标定量指标:准确率(Accuracy)、召回率(Recall)、F1-Score、AUC-ROC曲线。定性分析:通过案例研究(Case Study)验证参数合理性,如对比人工标注结果。工程指标:计算延迟、内存占用等硬件兼容性指标。\cite{neale2002100}


\subsection{评估指标}

评估指标
碰撞率:通过CARLA内置的碰撞传感器(sensor.other.collision)记录碰撞事件。

成功完成率

条件:同时满足以下两点:
无碰撞、无闯红灯、无越界(车道保持误差 < 0.5米)。
在时间限制内到达目标点(如交叉路口场景限时30秒)。
检测工具:基于高德MapDR规则引擎实时校验交通违规行为。

平均决策时间

在RL代理的决策循环中嵌入时间戳记录(time.perf\_counter())。
排除感知模块耗时(仅统计从状态输入到控制输出的计算延迟)。

恢复能力

诱导偏离:在场景中注入噪声(如突然横向风力、GPS信号丢失)。
恢复判定:车辆在5秒内返回原车道且速度稳定(波动 < 10\%)。


\begin{table}[htb]
	\centering
	\caption{指标对比}
	\label{T.example}
\begin{tabular}{llllll}
	\hline
	指标& ChatScene  & 随机搜索  & LiDARGen  & 提升率   \\
	\hline
	碰撞率 & 8.2\%  & 15.7\% & 22.4\% 	&  -47.8\%	\\
	\hline
	成功完成率 & 89.5\% & 76.3\% & 68.1\% 	& 	 +17.2\%	\\
	\hline
	平均决策时间 & 86 ms & 120 ms & 145 ms&	-28.3\%	\\
	\hline
	恢复成功率 & 92\% & 78\% & 65\%	&  +18.0\%	\\
	
	\hline
\end{tabular}
\end{table}


关键结论:


碰撞率与安全等级关联:ASIL-D级场景的碰撞率(23.5%)显著高于ASIL-A级(4.1%),验证了ASIL分类的有效性。
决策时间影响:当决策时间 > 150ms时,碰撞率上升至18.3%(vs. <100ms时的7.1%),凸显实时性对安全性的重要性。
恢复能力与场景复杂度:在动态障碍物交叉场景中,恢复成功率降至83%(vs. 变道场景的95%),表明需针对复杂场景优化控制策略。



\section{讨论与改进方向}
ChatScene 项目:


拓展场景生成能力方面,当前虽然能够基于文本描述生成场景,不过场景多样性或许会受到限制,可引入更多类型知识,像交通规则细则以及不同地区驾驶习惯数据,让生成的场景更贴合复杂现实情况,在依据文本描述生成场景时,结合真实事故案例数据,生成更具针对性和危险性的场景,以此提升自动驾驶系统测试的全面性。优化模型性能方面,训练场景选择和优化过程的计算成本可能比较高,要优化参数优化算法,减少不必要的计算步骤,提高场景选择效率,采用更高效的采样策略,在减少采样数量的同时保证所选场景质量,加快训练速度并降低实验成本。加强与其他技术融合方面,目前在与GPT - 4o集成还处于beta阶段\cite{uslu2014underground},后续应当持续优化集成效果,充分利用大语言模型强大的语言理解和生成能力,生成更自然、复杂的场景描述,探索与传感器模拟技术相结合,使生成的场景能够直接对接传感器数据模拟,为自动驾驶系统感知模块提供更真实的测试数据。


ASIL - Gen 项目:


要对场景类型和参数进行丰富处理,现有 13 种场景类型可进一步增添特殊场景,比如极端天气下的驾驶场景以及道路施工场景等,针对每个场景要细化相关参数,就像在跟车场景里增加前车不同加减速模式参数,以此让场景更具多样性和真实性,进而更全面地评估自动驾驶系统安全性。要对优化算法进行改进完善,在 NSGA - II 和随机搜索算法基础上探索结合其他智能优化算法,例如遗传算法、粒子群优化算法的优点,改进场景选择策略以提高选择效率和准确性,开发自适应算法依据场景特点和实验需求自动调整搜索策略,提升算法通用性。要完善 ASIL 分类标准,当前 ASIL 分类脚本计算方式或许存在一定局限性,结合更多实际因素如车辆动力学特性、环境干扰因素等来完善分类标准,使计算出的 ASIL 等级更准确反映场景真实安全风险,建立动态 ASIL 分类机制随着场景实时变化调整分类结果并结合两个项目 。
改进方向是统一场景评估体系,ChatScene重点在于场景生成,ASIL - Gen主要侧重于场景评估,要把两者的场景评估指标进行统一,从而让ChatScene生成的场景能够直接运用ASIL - Gen的ASIL分类标准来评估安全性,以此提高场景生成和评估的连贯性与效率,要相互补充场景资源,ChatScene动态生成场景的能力比较强,ASIL - Gen具备丰富的预生成场景以及详细的分类\cite{sayer2009integrated},ChatScene可以将生成的场景补充到ASIL - Gen场景库中,ASIL - Gen的预生成场景也能够为ChatScene提供训练和对比数据,进而拓展双方的场景资源,要协同优化实验流程,在实验流程方面进行协同,ChatScene生成的场景经过ASIL - Gen筛选和分类,确定高风险场景之后再运用ChatScene训练和评估自动驾驶模型,形成完整且高效的实验流程,以此推动自动驾驶系统安全性能的提升。

\newpage
