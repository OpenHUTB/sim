随着自动驾驶技术向商业化落地推进,极端驾驶场景的仿真测试成为验证系统安全性与鲁棒性的关键环节。现有仿真测试存在危险场景覆盖不全、生成缺乏系统性、真实性不足等问题,制约了自动驾驶技术的迭代优化。本研究以CARLA仿真平台为基础,围绕极端驾驶场景的分类、生成与量化评估展开研究。

首先,构建了包含主体行为、环境干扰、交通要素三个维度的危险场景分类体系,明确了前车紧急制动、路口抢行冲突等10种典型极端驾驶场景的特征与参数范围;其次,设计了双模式场景生成算法,通过手动控制与规则驱动实现典型场景的快速复现,基于深度确定性策略梯度(DDPG)的强化学习算法生成具有对抗性的复杂危险场景;最后,建立了涵盖安全指标、场景质量指标与生成效率指标的量化评估体系,对生成场景进行全面验证。

实验结果表明,所提分类体系具有良好的系统性与覆盖性,生成算法能够高效产出多样化、高真实性的极端驾驶场景,量化评估结果可有效支撑场景质量与算法性能的判定。本研究为自动驾驶系统的极端场景测试提供了标准化的场景资源与高效的生成方法,对推动自动驾驶技术的安全落地具有重要意义。

\textbf{关键词:} 自动驾驶;CARLA仿真平台;极端驾驶场景;场景生成;强化学习;量化评估