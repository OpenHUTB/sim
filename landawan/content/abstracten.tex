%!TEX root = ../csuthesis_main.tex
\keywordsen{Intelligent Driving \ \ Simulation Scenarios \ \ Deep learning \ \ Data synthesis}
\begin{abstracten}

Autonomous driving technology has developed rapidly in recent years, and the visual perception system is one of the keys to ensure the safety and efficiency of autonomous driving. The autonomous driving visual perception system relies on algorithms such as deep learning to extract environmental information from images and sensor data acquired from sensors such as cameras, lidar, and ultrasonics. However, the amount of data required to train these algorithms is huge, and data collection and annotation are costly.
Traditional autonomous driving data collection methods often rely on real-world road test data, which is difficult to cover a variety of complex roads, weather, traffic, and unexpected situations. The application of digital twin technology can efficiently generate training data by creating various driving scenarios in a virtual environment. This not only reduces the cost of data collection, but also expands the adaptability and robustness of the algorithm, especially when faced with rare or dangerous situations.
Therefore, the research goal of this project is to build a training data collection and evaluation system for autonomous driving visual perception algorithm based on digital twin technology, aiming to provide an effective data source through digital twin simulation and generation of diversified and high-quality driving data, and at the same time to verify and evaluate the algorithm, so as to improve the safety and reliability of autonomous driving technology.



\end{abstracten}