%!TEX root = ../csuthesis_main.tex
% 设置中文摘要
\keywordscn{智能驾驶\quad 仿真场景\quad 深度学习\quad 数据合成}
%\categorycn{TP391}
\begin{abstractzh}


自动驾驶技术近年来取得了迅猛发展,而其中的视觉感知系统是保证自动驾驶安全与效率的关键之一。自动驾驶视觉感知系统依赖于深度学习等算法,从摄像头、激光雷达、超声波等传感器中获取的图像和传感器数据中提取环境信息。然而,训练这些算法所需的数据量巨大,且数据收集和标注的成本高昂。
传统的自动驾驶数据收集方法通常依赖于真实世界的路测数据,难以覆盖各种复杂的道路、天气、交通和突发情况。数字孪生技术的应用可以通过在虚拟环境中创建各种驾驶场景,从而高效地生成训练数据。这不仅能降低数据收集成本,还能扩展算法的适应性和鲁棒性,尤其是当面对罕见或危险情况时。
因此,本课题的研究目标是构建一个基于数字孪生技术的自动驾驶视觉感知算法训练数据收集与评测系统,旨在通过数字孪生模拟和生成多样化、高质量的驾驶数据,提供一个有效的数据源,同时进行算法的验证与评测,提升自动驾驶技术的安全性与可靠性。


\end{abstractzh}