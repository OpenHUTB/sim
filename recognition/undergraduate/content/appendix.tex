%!TEX root = ../csuthesis_main.tex
% \begin{appendixs} % 无章节编号
\chapter{附录代码}

\section{CORnet-Z+SE算法Python描述}

% \begin{minted}[linenos]{c}
\begin{lstlisting}
from collections import OrderedDict
from torch import nn
    
HASH = 'cornet-z-se'
    
class SEBlock(nn.Module):
    def __init__(self, channel, reduction=8): # 减小reduction比例
    	super().__init__()
    	self.avg_pool = nn.AdaptiveAvgPool2d(1)
    	self.fc = nn.Sequential(
            nn.Linear(channel, channel // reduction),
            nn.ReLU(inplace=True),
            nn.Linear(channel // reduction, channel),
            nn.Sigmoid()
    	)
    	# 新增初始化方法
    	for m in self.modules():
            if isinstance(m, nn.Linear):
            nn.init.kaiming_normal_(m.weight, mode='fan_out', nonlinearity='relu')
            nn.init.constant_(m.bias, 0)
    
    def forward(self, x):
    	b, c, _, _ = x.size()
    	y = self.avg_pool(x).view(b, c)
    	y = self.fc(y).view(b, c, 1, 1)
    	return x * y
    
class Flatten(nn.Module):
    def forward(self, x):
        return x.view(x.size(0), -1)
    
class Identity(nn.Module):
    def forward(self, x):
        return x
    
class CORblock_Z(nn.Module):
    # 修改初始化函数
    def __init__(self, in_channels, out_channels, kernel_size=3, stride=1, use_se=True):
        super().__init__()
        self.conv = nn.Conv2d(in_channels, out_channels, kernel_size=kernel_size,stride=stride, padding=kernel_size // 2)
    
        # 添加SE模块
        self.se = SEBlock(out_channels) if use_se else nn.Identity()
    
        self.nonlin = nn.ReLU(inplace=True)
        self.pool = nn.MaxPool2d(kernel_size=3, stride=2, padding=1)
        self.output = Identity()
    
    def forward(self, inp):
    	x = self.conv(inp)
    	x = self.nonlin(x)    # 先激活
    	x = self.se(x)        # 后接SE
    	x = self.pool(x)
    	x = self.output(x)
    	return x
    
def CORnet_Z():
    model = nn.Sequential(OrderedDict([
    	# 为每个模块启用SE
    	('V1', CORblock_Z(3, 64, kernel_size=7, stride=2, use_se=True)),
    	('V2', CORblock_Z(64, 128, use_se=True)),
    	('V4', CORblock_Z(128, 256, use_se=True)),
    	('IT', CORblock_Z(256, 512, use_se=True)),
    	('decoder', nn.Sequential(OrderedDict([
            ('avgpool', nn.AdaptiveAvgPool2d(1)),
            ('flatten', Flatten()),
            ('linear', nn.Linear(512, 1000)),
            ('output', Identity())
    	])))
    ]))
    # 修改初始化部分
    for m in model.modules():
    	if isinstance(m, (nn.Conv2d, nn.Linear)):
            nn.init.xavier_uniform_(m.weight)
            if m.bias is not None:
                nn.init.constant_(m.bias, 0)
    	elif isinstance(m, nn.BatchNorm2d):
            m.weight.data.fill_(1)
            m.bias.data.zero_()
    	# 新增SE模块的初始化
    	elif isinstance(m, SEBlock):
            nn.init.kaiming_normal_(m.fc[0].weight, mode='fan_out')
            nn.init.constant_(m.fc[0].bias, 0)
            nn.init.normal_(m.fc[2].weight, mean=0, std=0.01)
            nn.init.constant_(m.fc[2].bias, 0)
    return model
\end{lstlisting}
% \end{minted}

\section{CORnet-Z+SE模型的Brain-Score接口封装代码}
\begin{lstlisting}
import torch
import os
import numpy as np
from PIL import Image
from torchvision import transforms
from brainscore_vision.model_helpers.activations.pytorch import PytorchWrapper  # brainscore提供的PyTorch模型包装器
from brainscore_vision.model_interface import BrainModel  # brainscore定义的模型接口
from brainio.assemblies import DataAssembly  # brainscore用于构造神经响应结果结构的类
from tqdm import tqdm  # 进度条库
from cornet_z_se import CORnet_Z  # 导入定义的模型

# 图像预处理函数:将输入图像转为模型接受的格式
def imagenet_preprocess(img: Image.Image):
    transform = transforms.Compose([
        transforms.Resize(224),  # 缩放至短边为224
        transforms.CenterCrop(224),  # 中心裁剪为224×224
        transforms.ToTensor(),  # 转为tensor
        transforms.Normalize(mean=[0.485, 0.456, 0.406],  # 标准化
                            std=[0.229, 0.224, 0.225])
    ])
    return transform(img)

# 构造 brainscore 标准接口模型类,继承自 BrainModel
class CORnetZZBrainModel(BrainModel):
    def __init__(self):
        model = CORnet_Z()  # 加载 CORnet-Z+SE 模型
        model.eval()  # 进入评估模式

        # 用 brainscore 的包装器封装模型和预处理
        self._wrapper = PytorchWrapper(
            identifier='cornet-zz',
            model=model,
            preprocessing=imagenet_preprocess
        )

        # 定义层名称到大脑区域的映射关系
        self._region_layer_map = {
	        'V1': 'V1.output',
	        'V2': 'V2.output',
        	'V4': 'V4.output',
        	'IT': 'IT.output',
        }

        self._recording_layers = None  # 当前待提取的层
        self._image_path = os.path.expanduser("~/.brainio/image_dicarlo_hvm-public")  # 图像根目录路径
        self._preprocess = imagenet_preprocess  # 图像预处理函数

    # 指定要记录的层(如 V4.output)
    def start_recording(self, target, *args, **kwargs):
        self._recording_layers = [self._region_layer_map[target]]

    # 处理输入图像并提取激活,构造成 DataAssembly 返回
    def look_at(self, stimuli, number_of_trials=1):
        filenames = stimuli['filename'].values  # 读取stimuli中图像文件名
        full_paths = [os.path.join(self._image_path, name) for name in filenames]  # 生成完整路径
        images = [self._preprocess(Image.open(path).convert('RGB')) for path in tqdm(full_paths, desc="Preprocessing")]
        images = torch.stack(images)  # 转为一个batch的tensor

        # 提取指定层的激活
        activations_dict = self._wrapper.get_activations(images, layer_names=self._recording_layers)
        layer_name = self._recording_layers[0]
        activations = activations_dict[layer_name]

        # 将Tensor转为numpy数组
        if isinstance(activations, torch.Tensor):
            activations = activations.detach().cpu().numpy()

        # 如果是二维 [batch_size, features]
        if isinstance(activations, np.ndarray) and activations.ndim == 2:
            activation_array = activations

        # 如果是四维 [batch_size, channels, H, W],做全局平均池化
        elif isinstance(activations, np.ndarray) and activations.ndim == 4:
            activation_array = activations.mean(axis=(2, 3))

        # 其他格式抛出异常
        else:
            raise ValueError(f"Unexpected activation shape or type: {type(activations)}, shape: {getattr(activations, 'shape', None)}")

        # 构造神经元和刺激的标签
        neuroid_ids = list(range(activation_array.shape[1]))
        stimulus_ids = stimuli['stimulus_id'].values

        # 构建 DataAssembly:用于 brainscore 后续评估
        assembly = DataAssembly(
            data=activation_array,
            coords={
                'stimulus_id': ('presentation', stimulus_ids),
                'presentation': np.arange(len(stimulus_ids)),
                'neuroid': ('neuroid', neuroid_ids),
                'layer': ('neuroid', self._recording_layers * activation_array.shape[1]),
                'region': ('neuroid', [self._region_from_layer(l) for l in self._recording_layers] * activation_array.shape[1]),
                'object_name': ('presentation', list(stimuli['object_name'])),
            },
            dims=['presentation', 'neuroid']
        )
        assembly.name = layer_name
        return assembly

    # 根据层名推断脑区名
    def _region_from_layer(self, layer):
        return layer.split('.')[0]

    # 提取指定层的所有中间激活(非 look_at 格式)
    def extract_layers(self, image):
        return self._wrapper.extract_layers(image, layers=self._recording_layers)

    # 图像在视觉角度下的大小(单位 degree,Brain-Score 用)
    @property
    def visual_degrees(self):
        return 8

    # 模型标识符
    @property
    def identifier(self):
        return self._wrapper.identifier

\end{lstlisting}

% \end{appendixs}
