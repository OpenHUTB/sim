%!TEX root = ../csuthesis_main.tex
\keywordsen{Brain-inspired vision\ \ CORnet model\ \ Channel attention\ \ Brain-Score evaluation\ \ Neural response similarity}
\begin{abstracten}

Brain-inspired visual models aim to simulate the information processing mechanisms of the human visual cortex and play a significant role in the interdisciplinary research between computer vision and neuroscience. This study is based on the CORnet model family and focuses on two main aspects: recognition performance optimization and brain-likeness analysis. First, the CORnet-S model was reproduced on the Tiny-ImageNet dataset to verify its image recognition ability under a lightweight architecture. Then, a modified model named CORnet-Z+SE was constructed by introducing a Squeeze-and-Excitation (SE) channel attention mechanism into the CORnet-Z model. Experimental results show that both Top-1 and Top-5 accuracy improved to a certain extent.

To evaluate the brain-likeness of the models, this study employed the Brain-Score framework and used the publicly available MajajHong2015 neural dataset to assess representational similarity in the V4 and IT brain regions. Results indicate that although the introduction of attention mechanisms improved classification accuracy, it led to a decline in brain similarity scores, suggesting a trade-off between discriminative performance and neural alignment. Visualization of activation maps further showed that the SE module enhanced the model’s response to local target regions, but may have reduced the breadth of overall activation, potentially affecting its biological plausibility.

\end{abstracten}
