%!TEX root = ../../csuthesis_main.tex
\chapter{总结与展望}

\section{研究总结}

本研究围绕类脑视觉建模与评估展开,重点探讨了基于CORnet模型的复现、优化与神经对齐能力分析。首先,在Tiny-ImageNet数据集上成功复现了CORnet-S模型,实验结果显示该模型具备较强的图像识别能力,验证集和测试集的 Top-1 准确率分别达到了60.34\%和63.67\%,性能接近原论文所述水平。

在此基础上,本文以CORnet-Z模型为框架,引入通道注意力机制(Squeeze-and-Excitation,SE)以提升模型对判别性特征的感知能力。通过对比实验发现,引入SE模块后模型在验证集和测试集上的Top-1与Top-5准确率均有所提升,说明注意力机制在一定程度上改善了特征提取效果,提升了模型在复杂图像场景中的识别性能。

此外,本文还基于Brain-Score提供的MajajHong2015数据集,对原始CORnet-Z、改进后的CORnet-Z+SE以及其他模型(如ResNet-18、AlexNet、CBAM版本、VOneBlock版本)进行了类脑相似性评估。从结果来看,原始CORnet-Z模型在IT层的神经预测得分较高,而加入SE、CBAM或VOneBlock等结构后,虽然提升了图像分类准确率,但在类脑得分方面出现了一定程度的下降,表明部分优化方式在类脑性和准确性之间存在权衡。

为了进一步理解模型的内部机制,本文还对CORnet-Z及其优化版本在各层的激活图进行了可视化分析。结果显示,注意力机制引导模型对局部目标区域(如车轮、车身)产生更集中的响应,表现出较强的局部结构识别能力。但同时也可能削弱了模型对整体空间结构的编码能力,这在一定程度上解释了类脑相似性指标下降的原因。

\section{未来展望}

尽管本文在模型复现、优化与类脑评估方面取得了一定成果,但仍存在以下几个值得进一步探索的方向:

引入跨层反馈机制。当前CORnet系列模型仍以前馈结构为主,未来可尝试引入类脑的跨层反馈连接,模拟V1至IT之间的动态信息流,以提高模型的时间动态建模能力与视觉稳健性。

优化类脑结构与准确性之间的权衡。如何在提升识别性能的同时保持或增强模型与大脑皮层的相似性,是构建高质量类脑模型的关键问题。未来可以考虑融合稀疏编码机制、神经解剖图谱约束等方式进行优化。

扩展评估场景与指标体系。目前的Brain-Score评估主要基于标准化静态图像与神经响应数据,后续可以结合更复杂的任务(如视频理解、主动视觉)与多模态脑数据(如fMRI与MEG)进行更全面的类脑分析。

提升硬件适配性。在保持轻量化的基础上,进一步探索神经形态硬件平台上的部署策略(如Loihi、寒武纪MLU),使类脑模型具备更强的边缘计算能力和实时性。

总的来看,类脑视觉模型是人工智能与神经科学交叉的重要方向,本文的探索虽然初步,但为后续更高层次的类脑建模研究提供了一定的基础和参考。
