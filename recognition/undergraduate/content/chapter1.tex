%!TEX root = ../../csuthesis_main.tex

\chapter{绪论}

\section{研究背景及意义}

\subsection{研究背景}

近年来,深度卷积神经网络(CNN)迅速发展,从AlexNet的简单八层前馈架构发展到NASNet等极其
深层和分支的架构,展现出了越来越好的物体分类性能和对神经和行为反应的解释能力,并被认为
是最有可能模拟灵长类动物人脑视觉系统处理视觉信息的模型。然而,从神经科学的角度来看,这
些非常深的架构与人脑视觉系统仍然存在较大差异。

人类的视觉系统是一个高度复杂和精密的器官,它能够从简单的光线感知中提取出丰富的信息,使我
们能够识别物体、理解场景并做出相应的反应。这种复杂的视觉能力背后,是大脑视觉系统结构的复
杂性。人脑分为多个视觉区域,每个视觉区域之间又存在着复杂的神经网络连接,而这些神经网络连
接又分为横向、纵向和反馈连接,这些丰富的神经网络连接形成一个层次化的视觉处理系统,使得大
脑视觉系统能够进行灵活和高效的视觉信息处理。除此之外,大脑视觉系统还能够处理动态的视觉信
息。例如,我们能够识别物体的运动轨迹,并预测物体的未来位置。

针对这些深度卷积神经网络模型与真实大脑视觉系统之间的差异性,人们就提出了CORnet模型。CORnet
模型是一个基于深度学习的人工神经网络模型,它旨在模拟人类大脑视觉系统的核心功能——物体识别。

\subsection{研究意义}

本研究的意义主要体现在两个方面:

\subsubsection{理解大脑视觉机制}

CORnet模型通过模拟大脑视觉处理流的层级结构和关键机制,为理解大脑如何处理视觉信息提供了新的
视角。该模型将输入图像信息逐步传递到不同的视觉区域,并进行特征提取和整合,模拟了大脑如何通
过不同层次的视觉处理来识别物体。此外,CORnet模型还引入了反馈连接和时间动态等机制,进一步模
拟了大脑视觉处理的复杂性。通过构建类脑模型,可以更深入地理解大脑如何处理视觉信息,包括物体
识别、场景理解等复杂任务,并验证神经科学的假设,推动神经科学的发展。

\subsubsection{提高视觉模型性能}

CORnet模型不仅在理解大脑视觉处理方面发挥了重要作用,还为视觉模型的发展提供了新的思路。类脑
模型可以借鉴大脑的强项,例如处理动态场景、识别复杂物体和解释抽象概念等,从而提高视觉模型的
性能。CORnet模型的设计理念为开发新的深度学习模型提供了灵感。此外,CORnet模型的可解释性也有
助于提高模型的可信度和可靠性,推动深度学习模型在现实世界场景中的应用,例如自动驾驶、医疗诊断等。

除此之外,CORnet等类脑视觉模型在促进神经科学和机器学习的交叉融合方面也具有重要意义。CORnet
模型为我们提供了一个探索大脑工作原理和构建更强大机器学习模型的新视角,为跨学科研究和交流搭建
了桥梁,并推动了神经科学和机器学习的交叉融合,为解决人类面临的挑战提供了新的思路和方法。



\section{文献综述}

\subsection{国内研究现状}

在类脑视觉模型与深度学习的国内研究方面,曾国叙等针对现有机器学习算法在图像识别时因微小
变化易失败的问题,提出基于NCC特征匹配的类脑视觉识别模型,利用网格细胞进行坐标编码,经
实验验证在小样本测试下识别率较高,但测试对象增多时识别率下降且速度较慢,对灰度变化有较
好鲁棒性。杜长德等从视觉信息编解码角度探讨了深度学习类脑机制,指出其在视觉认知研究中的
关键作用,并提出基于深度多视图生成式模型的编解码研究,虽有成果但存在如自然图像重建效果
待提升等局限性。杨曦等回顾视觉目标识别相关工作,总结基于脑启发的目标识别模型,分析深度
神经网络与视觉皮层相似性,介绍实验条件和方法,指出当前模型虽有进展但与人类水平仍有差距
,未来需在多方面改进。刘建立则聚焦于视觉审美体验的类脑模型构建,提出了不同框架下的模型
比较,但该模型的应用范围和实际效果尚需进一步验证。这些研究表明国内在类脑视觉模型与深度
学习领域积极探索,在模型构建、机制研究等方面取得一定成果,但仍面临模型的适应性、实用性
和与生物机制的深度融合等挑战。

\subsection{国外研究现状}

在国外类脑视觉模型研究方面,Schrimpf等提出Brain-Score来评估人工神经网络与大脑视觉处理机
制的相似性,通过对多种深度神经网络在神经和行为基准上的测试,发现 DenseNet169、CORnet-S
和ResNet-101是当前最类似大脑的模型,但也指出ImageNet性能与Brain-Score存在潜在脱节,且当
前模型仍有改进空间,如可增加数据、改进度量方式等。Cadena等利用卷积神经网络(CNN)对恒河
猴初级视觉皮层(V1)对自然图像的反应进行建模,比较了目标驱动和数据驱动两种方法,发现多层
CNN能显著提高预测性能,且基于 VGG-19的目标驱动模型在使用较少训练数据时表现更好,为理解V1
神经元的非线性计算提供了新视角。Nair等开发了基于ResNet 50的CORNet模型,用于从肺部CT扫描
图像中检测COVID-19,在包含大量患者CT图像的数据集上进行训练和测试,结果显示该模型在敏感性、
特异性和AUC等指标上表现良好,优于一些现有算法,为医学图像分析中的疾病诊断提供了新的方法。
Lu等基于CORnet模型提出ReAlnet-fMRI模型,通过多层编码对齐框架利用人类fMRI数据进行优化,在
多个模态和不同受试者的评估中,该模型比CORnet和对照模型在与人类大脑的对齐上表现更优,且在对
不同物体维度的编码上与CORnet存在差异,表明整合人类神经数据可增强视觉模型的大脑相似性。
Yamins和DiCarlo回顾了基于目标驱动的分层卷积神经网络(HCNNs)在感觉皮层建模方面的进展,指出
其在预测神经反应方面的优势,以及硬件加速、自动学习架构参数、大数据集等因素对其发展的推动作
用,同时也探讨了该模型在架构、目标和训练集、学习规则等方面的改进方向及未来研究的潜力与局限。


\subsection{未来发展方向}

综上所述,国内外在类脑视觉模型研究方面均成果颇丰。国外通过多种方式评估和改进模型,如利用
Brain-Score筛选模型、用CNN建模脑区响应、开发疾病检测模型及整合神经数据优化模型等,并指出
了发展方向。国内也积极探索,提出了基于NCC特征匹配等模型。未来,在类脑视觉模型领域有望在以
下几个方面实现突破。
多模态数据融合:当前研究主要集中在单一模态(如视觉图像)与大脑数据的关联,未来有望融合多
种模态数据,如视觉与听觉、触觉等信息,更全面地模拟大脑的感知和认知过程。例如在真实的环境
交互中,人类大脑会同时处理多种感官输入,类脑视觉模型的进一步发展需要向此方向靠拢,以实现
更强大的智能。
模型结构优化:尽管已有研究在探索不同的神经网络架构,如CORnet模型等,但仍需更深入地挖掘与
大脑结构和功能更契合的模型结构。这可能包括对现有卷积神经网络等架构的改进,或者开发全新的
架构,使其在处理视觉信息时能更好地模拟大脑的层次化、并行化处理机制以及神经元之间的复杂连
接关系。
生物合理性增强:目前的模型在某些方面与大脑仍存在差距,未来研究将致力于提高模型的生物合理
性。从神经元的微观行为到大脑区域间的宏观连接,都需要更精确的建模,比如进一步研究神经元的
脉冲时间依赖可塑性等特性在模型中的应用,以及如何更好地模拟大脑的反馈连接、侧向连接等机制。
跨领域应用拓展:类脑视觉模型不仅在计算机视觉领域有重要应用,在医疗(如疾病诊断)、机器人
(如视觉导航与操作)、交通(如自动驾驶中的视觉感知)等众多领域都有巨大的应用潜力。随着模
型的不断发展,其应用范围将进一步拓展,解决更多实际问题。


\section{研究方法与思路}

\subsubsection{复现CORnet模型}
仔细阅读论文,理解CORnet模型的架构、参数设置和训练过程。使用论文中提供的代码和权重文件,
或使用PyTorch等深度学习框架重新实现模型。在ImageNet数据集上训练模型,并评估其在ImageNet
分类任务上的性能。使用Brain-Score基准测试评估模型对大脑和行为的预测能力。
\subsubsection{改进类脑视觉网络结构}
增加反馈连接:研究论文中提到,CORnet模型缺乏反馈连接,而灵长类视觉通路中存在大量的反馈连接。
可以尝试在CORnet模型中引入反馈连接,并评估其对性能的影响。
引入注意力机制:注意力机制可以帮助模型聚焦于图像中的重要特征,从而提高识别精度。可以尝试将
注意力机制集成到CORnet模型中,并评估其对性能的影响。
探索新的卷积结构:可以尝试使用更复杂的卷积结构,例如可变形卷积或动态卷积,来更好地模拟大
的视觉处理机制。
使用自监督学习:自监督学习可以帮助模型在没有标注数据的情况下学习图像特征,从而提高模型的
泛化能力。可以尝试使用自监督学习方法训练CORnet模型,并评估其对性能的影响。
\subsubsection{实验设计}
比较不同改进方案的性能:对比不同改进方案在ImageNet分类任务和Brain-Score基准测试上的性能,
找出最优的改进方案。
分析模型内部表征:使用可视化技术分析模型内部表征,例如特征图或神经元活动,以理解模型如何
进行特征提取和分类。
进行消融实验:通过移除或修改模型中的某些部分,来研究不同部分对模型性能的影响。



