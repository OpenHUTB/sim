%!TEX root = ../csuthesis_main.tex
% 设置中文摘要
\keywordscn{智能驾驶\quad 测试场景\quad 大语言模型\quad 场景描述语言\quad Scenic}
%\categorycn{TP391}
\begin{abstractzh}
	
	智能驾驶测试场景构建效率低下是当前行业面临的重要问题。为此,本文提出了一种融合大语言模型(LLM)与形式化场景描述语言的新型生成框架,旨在高效地将自然语言描述转化为高保真三维测试场景。该框架包含三大核心模块:一是领域知识增强的指令解析模块,能够精准理解自然语言描述中的测试场景需求,为后续生成工作奠定基础;二是语法 - 语义双验证代码生成机制,确保生成的Scenic代码不仅语法正确,还符合实际场景的语义逻辑,从而提高代码的可用性和准确性;三是物理规则驱动的场景合成引擎,依据物理规则合理生成三维场景,保证场景的真实性和可靠性。在CARLA仿真平台的实验验证中,该系统展现出优异的性能:生成的Scenic代码准确率高达92.3%,场景物理合规率超过85%,且生成时延低于3秒/场景。这一成果显著提升了智能驾驶测试场景的构建效率,为智能驾驶技术的测试与优化提供了高效、准确的工具支持,有望推动智能驾驶领域相关测试工作的高效开展,促进智能驾驶技术的进一步发展与完善,具有重要的应用价值和广阔的发展前景。
	
\end{abstractzh}