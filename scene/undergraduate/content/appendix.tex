\chapter{附录:系统模块源代码}

\subsection*{自然语言理解与场景生成模块}
\begin{lstlisting}
	retrieve.py:该文件是系统的主文件,负责从自然语言描述中生成场景代码。它集成了自然语言处理和检索增强模块,通过调用大语言模型生成符合输入语义的Scenic场景脚本。

		import os
		import setGPU
		import csv
		import pickle
		import re
		from sentence_transformers import SentenceTransformer, models
		from os import path as osp
		from tqdm import tqdm
		import argparse
		from architecture import LLMChat
		from utils import load_file, retrieve_topk, generate_code_snippet, save_scenic_code
		
		
		# no need for faiss currently
		# import faiss
		
		# Argument parsing
		parser = argparse.ArgumentParser(description="Set up configurations for your script.")
		parser.add_argument('--port_ip', type=int, default=2000, help='Port IP address (default: 2000)')
		parser.add_argument('--topk', type=int, default=3, help='Top K value (default: 3) for retrieval')
		parser.add_argument('--model', type=str, default='gpt-4o', help="Model name (default: 'gpt-4o'), also support transformers model")
		parser.add_argument('--use_llm', action='store_true', help='if use llm for generating new snippets')
		args = parser.parse_args()
		
		# Configuration
		port_ip = args.port_ip
		topk = args.topk
		use_llm = args.use_llm
		
		# LLM model initialization
		llm_model = LLMChat(args.model)
		local_path = osp.abspath(osp.dirname(osp.dirname(osp.realpath(__file__))))
		extraction_prompt = load_file(osp.join(local_path, 'retrieve', 'prompts', 'extraction.txt'))
		behavior_prompt = load_file(osp.join(local_path, 'retrieve', 'prompts', 'behavior.txt'))
		geometry_prompt = load_file(osp.join(local_path, 'retrieve', 'prompts', 'geometry.txt'))
		spawn_prompt = load_file(osp.join(local_path, 'retrieve', 'prompts', 'spawn.txt'))
		scenario_descriptions = load_file(osp.join(local_path, 'retrieve', 'scenario_descriptions.txt')).split('\n')
		
		# 🔥 修改开始:本地加载 sentence-t5-large 模型
		model_dir = r"D:\sceneMain\chatScene\models\sentence-t5-large"
		if not os.path.exists(model_dir):
		raise FileNotFoundError(f"本地模型路径不存在:{model_dir}")
		
		required_files = ["config.json", "pytorch_model.bin"]
		for filename in required_files:
		if not os.path.exists(os.path.join(model_dir, filename)):
		raise FileNotFoundError(f"缺少必要的文件: {filename} 在 {model_dir} 中")
		
		word_embedding_model = models.Transformer(model_dir, max_seq_length=512)
		pooling_model = models.Pooling(
		word_embedding_model.get_word_embedding_dimension(),
		pooling_mode='mean'
		)
		encoder = SentenceTransformer(modules=[word_embedding_model, pooling_model], device='cuda')
		print("✅ 成功加载本地 sentence-t5-large 模型!")
		# 🔥 修改结束
		
		# Load the database
		with open(osp.join(local_path, 'retrieve/database_v1.pkl'), 'rb') as file:
		database = pickle.load(file)
		
		behavior_descriptions = database['behavior']['description']
		geometry_descriptions = database['geometry']['description']
		spawn_descriptions = database['spawn']['description']
		behavior_snippets = database['behavior']['snippet']
		geometry_snippets = database['geometry']['snippet']
		spawn_snippets = database['spawn']['snippet']
		
		behavior_embeddings = encoder.encode(behavior_descriptions, device='cuda', convert_to_tensor=True)
		geometry_embeddings = encoder.encode(geometry_descriptions, device='cuda', convert_to_tensor=True)
		spawn_embeddings = encoder.encode(spawn_descriptions, device='cuda', convert_to_tensor=True)
		
		# This is the head for scenic file, you can modify the carla map or ego model here
		head = '''param map = localPath(f'../maps/{Town}.xodr') 
		param carla_map = Town
		model scenic.simulators.carla.model
		EGO_MODEL = "vehicle.lincoln.mkz_2017"
		'''
		
		log_file_path = osp.join(local_path, 'safebench', 'scenario', 'scenario_data', 'scenic_data', 'dynamic_scenario', 'dynamic_log.csv')
		
		# Write log results
		with open(log_file_path, mode='w', newline='') as file:
		log_writer = csv.writer(file)
		log_writer.writerow(['Scenario', 'AdvObject', 'Behavior Description', 'Behavior Snippet', 'Geometry Description', 'Geometry Snippet', 'Spawn Description', 'Spawn Snippet', 'Success'])
		
		# Process each scenario description
		for q, current_scenario in tqdm(enumerate(scenario_descriptions)):
		messages = [
		{"role": "system", "content": "You are a helpful assistant."},
		{"role": "user", "content": extraction_prompt.format(scenario=current_scenario)},
		]
		
		response = llm_model.generate(messages)
		
		try:
		match = re.search(r"Adversarial Object:(.*?)Behavior:(.*?)Geometry:(.*?)Spawn Position:(.*)", response, re.DOTALL)
		if not match:
		raise ValueError("Failed to extract components from the response")
		
		current_adv_object, current_behavior, current_geometry, current_spawn = [s.strip() for s in match.groups()]
		
		# Retrieve the top K relevant snippets
		top_behavior_descriptions, top_behavior_snippets = retrieve_topk(encoder, topk, behavior_descriptions, behavior_snippets, behavior_embeddings, current_behavior)
		top_geometry_descriptions, top_geometry_snippets = retrieve_topk(encoder, topk, geometry_descriptions, geometry_snippets, geometry_embeddings, current_geometry)
		top_spawn_descriptions, top_spawn_snippets = retrieve_topk(encoder, topk, spawn_descriptions, spawn_snippets, spawn_embeddings, current_spawn)
		
		# Generate code snippets using the LLM
		generated_behavior_code = generate_code_snippet(
		llm_model, behavior_prompt, top_behavior_descriptions, top_behavior_snippets, current_behavior, topk, use_llm
		)
		
		generated_geometry_code = generate_code_snippet(
		llm_model, geometry_prompt, top_geometry_descriptions, top_geometry_snippets, current_geometry, topk, use_llm
		)
		
		generated_spawn_code = generate_code_snippet(
		llm_model, spawn_prompt, top_spawn_descriptions, top_spawn_snippets, current_spawn, topk, use_llm
		)
		
		# Log the results
		log_writer.writerow([current_scenario, current_adv_object, current_behavior, generated_behavior_code, current_geometry, generated_geometry_code, current_spawn, generated_spawn_code, 1])
		
		Town, generated_geometry_code = generated_geometry_code.split('\n', 1)
		scenic_code = '\n'.join([f"'''{current_scenario}'''", Town, head, generated_behavior_code, generated_geometry_code, generated_spawn_code.format(AdvObject=current_adv_object)])
		save_scenic_code(local_path, port_ip, scenic_code, q)
		
		except Exception as e:
		log_writer.writerow([current_scenario, '', '', '', '', '', '', '', 0])
		print(f"Failure for scenario: {current_scenario} - Error: {e}")
		
		
\end{lstlisting}

\subsection*{场景合成与仿真模块}
\begin{lstlisting}
	run\_train\_dynamic.py:作用:用于在动态生成的场景上训练代理。该文件使用dynamic\_scenic.yaml进行配置,并运行训练过程,优化代理的行为。


		import setGPU
		import traceback
		import os
		import os.path as osp
		
		import torch 
		from safebench.util.run_util import load_config
		from safebench.util.torch_util import set_seed, set_torch_variable
		from safebench.carla_runner import CarlaRunner
		from safebench.scenic_runner_dynamic import ScenicRunner
		
		if __name__ == '__main__':
		import argparse
		parser = argparse.ArgumentParser()
		parser.add_argument('--exp_name', type=str, default='exp')
		parser.add_argument('--output_dir', type=str, default='log')
		parser.add_argument('--ROOT_DIR', type=str, default=osp.abspath(osp.dirname(osp.dirname(osp.realpath(__file__)))))
		
		parser.add_argument('--max_episode_step', type=int, default=300)
		parser.add_argument('--auto_ego', action='store_true')
		parser.add_argument('--mode', '-m', type=str, default='eval', choices=['train_scenario', 'train_agent', 'eval'])
		parser.add_argument('--agent_cfg', nargs='*', type=str, default=['adv_scenic.yaml'])
		parser.add_argument('--scenario_cfg', nargs='*', type=str, default=['dynamic_scenic.yaml'])
		parser.add_argument('--continue_agent_training', '-cat', type=bool, default=False)
		parser.add_argument('--continue_scenario_training', '-cst', type=bool, default=False)
		
		parser.add_argument('--seed', '-s', type=int, default=0)
		parser.add_argument('--threads', type=int, default=4)
		parser.add_argument('--device', type=str, default='cuda:0' if torch.cuda.is_available() else 'cpu')   
		
		parser.add_argument('--num_scenario', '-ns', type=int, default=1, help='num of scenarios we run in one episode')
		parser.add_argument('--save_video', action='store_true')
		parser.add_argument('--render', type=bool, default=True)
		parser.add_argument('--frame_skip', '-fs', type=int, default=1, help='skip of frame in each step')
		parser.add_argument('--port', type=int, default=2002, help='port to communicate with carla')
		parser.add_argument('--tm_port', type=int, default=8002, help='traffic manager port')
		parser.add_argument('--fixed_delta_seconds', type=float, default=0.1)
		args = parser.parse_args()
		args_dict = vars(args)
		
		err_list = []
		for agent_cfg in args.agent_cfg:
		for scenario_cfg in args.scenario_cfg:
		# set global parameters
		set_torch_variable(args.device)
		torch.set_num_threads(args.threads)
		set_seed(args.seed)
		
		# load agent config
		agent_config_path = osp.join(args.ROOT_DIR, 'safebench/agent/config', agent_cfg)
		agent_config = load_config(agent_config_path)
		
		# load scenario config
		scenario_config_path = osp.join(args.ROOT_DIR, 'safebench/scenario/config', scenario_cfg)
		scenario_config = load_config(scenario_config_path)
		
		agent_config['load_dir'] = osp.join(agent_config['load_dir'], 'dynamic_scenario')
		# Check if the directory exists; if not, create it
		if not osp.exists(agent_config['load_dir']):
		os.makedirs(agent_config['load_dir'])        
		
		# main entry with a selected mode
		agent_config.update(args_dict)
		args_dict['output_dir'] = osp.join('log', 'adv_train', args.mode, agent_config['policy_name'], f"{agent_cfg.split('.')[0]}", "dynamic_scenario")
		scenario_config.update(args_dict)
		scenario_config['num_scenario'] = 1 ### 'the num_scenario can only be one for scenic'
		runner = ScenicRunner(agent_config, scenario_config)
		
		
		# start running
		runner.run()
		
		for err in err_list:
		print(err[0], err[1], 'failed!')
		print(err[2])
		
		
		
	\end{lstlisting}
\begin{lstlisting}
	dynamic\_scenic.yaml:作用:该文件是代理的配置文件,包含了代理的训练设置,包括其行为模型、对抗性行为、场景属性等。它与其他 YAML 文件结合使用来指定代理在不同场景中的行为。

		policy_type: 'scenic'
		scenario_category: 'scenic'
		
		route_dir: 'safebench/scenario/scenario_data/route'
		scenic_dir: 'safebench/scenario/scenario_data/scenic_data/'
		sample_num: 50
		opt_step: 10
		select_num: 2
		
		method: 'scenic'
		scenario_id: null
		route_id: [0,1,2,3,4,5,6,7]
		
		ego_action_dim: 2
		ego_state_dim: 4
		ego_action_limit: 1.0
		
		
\end{lstlisting}

\subsection*{评估与展示模块}

	
\begin{lstlisting}
		evaluate\_scene\_quality.py:该文件负责对生成的场景进行量化评估,输出语义保真度、多样性与驾驶性能相关指标。

		import os
		import json
		import numpy as np
		import matplotlib.pyplot as plt
		from sklearn.metrics import pairwise_distances_argmin_min
		import cv2
		
		# 文件路径
		DESCRIPTION_FILE = 'D:/sceneMain/chatScene/retrieve/scenario_descriptions.txt'
		HISTORY_FILE = 'D:/sceneMain/chatScene/retrieve/scenario_history.txt'
		SCENE_IMAGE_DIR = 'D:/sceneMain/chatScene/outputs/'
		
		# 加载最新的场景描述(只读取文件的第一行)
		def load_latest_description(path):
		"""只读取描述文件中的第一行"""
		with open(path, 'r', encoding='utf-8') as f:
		first_line = f.readline().strip()  # 读取第一行
		return first_line
		
		# 将新的场景描述追加到历史记录文件
		def append_to_history(new_description, history_path):
		"""将新的场景描述追加到历史文件"""
		with open(history_path, 'a', encoding='utf-8') as f:
		f.write(new_description + '\n')
		
		# 计算图像相似度(使用结构相似度)
		def calculate_image_similarity(image1, image2):
		"""计算两张图像之间的相似度"""
		gray1 = cv2.cvtColor(image1, cv2.COLOR_BGR2GRAY)
		gray2 = cv2.cvtColor(image2, cv2.COLOR_BGR2GRAY)
		score, _ = cv2.quality.QualitySSIM_compute(gray1, gray2)
		return score
		
		# 计算场景的多样性(使用生成图像之间的距离)
		def calculate_scene_diversity(image_dir):
		"""计算所有图像之间的多样性"""
		images = []
		for filename in os.listdir(image_dir):
		if filename.endswith('.png'):
		img = cv2.imread(os.path.join(image_dir, filename))
		images.append(img)
		
		# 转换为数组(每个图像的特征)
		image_features = [np.reshape(img, (-1, 3)) for img in images]
		image_features = np.concatenate(image_features, axis=0)
		
		# 计算每对图像的最小距离
		distances = pairwise_distances_argmin_min(image_features, image_features)
		avg_distance = np.mean(distances[1])  # 平均最小距离
		return avg_distance
		
		# 评估场景质量:语义一致性,图像质量,多样性
		def evaluate_scene_quality(image_dir):
		"""评估场景的质量"""
		
		# 语义一致性(假设为手动指定或从其他方法中获得)
		semantic_consistency = 0.9  # 假设的值,通常需要根据具体情况进行计算
		
		# 图像质量:假设使用已有的参考图像进行评估(此处为一个示例)
		reference_image = cv2.imread('D:/sceneMain/chatScene/reference_image.png')  # 参考图像
		image_files = [f for f in os.listdir(image_dir) if f.endswith('.png')]
		avg_image_quality = 0
		for image_file in image_files:
		img = cv2.imread(os.path.join(image_dir, image_file))
		similarity = calculate_image_similarity(reference_image, img)
		avg_image_quality += similarity
		avg_image_quality /= len(image_files)
		
		# 多样性
		diversity = calculate_scene_diversity(image_dir)
		
		# 打印评估结果
		print(f"语义一致性: {semantic_consistency}")
		print(f"平均图像质量: {avg_image_quality}")
		print(f"场景多样性: {diversity}")
		
		return semantic_consistency, avg_image_quality, diversity
		
		# 主函数
		def main():
		# 读取最新的场景描述
		latest_description = load_latest_description(DESCRIPTION_FILE)
		
		# 将描述追加到历史记录
		append_to_history(latest_description, HISTORY_FILE)
		
		# 打印最新描述
		print(f"最新的场景描述: {latest_description}")
		
		# 评估生成的场景质量
		semantic_consistency, avg_image_quality, diversity = evaluate_scene_quality(SCENE_IMAGE_DIR)
		
		# 可以根据需要将评估结果保存为JSON或其他格式
		evaluation_results = {
			"semantic_consistency": semantic_consistency,
			"avg_image_quality": avg_image_quality,
			"diversity": diversity
		}
		
		# 保存评估结果到文件
		with open('D:/sceneMain/chatScene/outputs/evaluation_results.json', 'w') as f:
		json.dump(evaluation_results, f, indent=4)
		
		if __name__ == "__main__":
		main()
		
		
\end{lstlisting}
	
	
\begin{lstlisting}
		run\_eval\_dynamic.py:该文件用于运行评估流程,调用evaluate\_scene\_quality.py 对生成的场景进行评估。

		import setGPU
		import traceback
		import os.path as osp
		
		import torch 
		
		from safebench.util.run_util import load_config
		from safebench.util.torch_util import set_seed, set_torch_variable
		from safebench.carla_runner import CarlaRunner
		from safebench.scenic_runner_dynamic import ScenicRunner
		
		if __name__ == '__main__':
		import argparse
		parser = argparse.ArgumentParser()
		parser.add_argument('--exp_name', type=str, default='exp')
		parser.add_argument('--output_dir', type=str, default='log')
		parser.add_argument('--ROOT_DIR', type=str, default=osp.abspath(osp.dirname(osp.dirname(osp.realpath(__file__)))))
		
		parser.add_argument('--max_episode_step', type=int, default=300)
		parser.add_argument('--auto_ego', action='store_true')
		parser.add_argument('--mode', '-m', type=str, default='eval', choices=['train_agent', 'train_scenario', 'eval'])
		parser.add_argument('--agent_cfg', nargs='*', type=str, default=['adv_scenic.yaml'])
		parser.add_argument('--scenario_cfg', nargs='*', type=str, default=['dynamic_scenic.yaml'])
		parser.add_argument('--continue_agent_training', '-cat', type=bool, default=False)
		parser.add_argument('--continue_scenario_training', '-cst', type=bool, default=False)
		
		parser.add_argument('--seed', '-s', type=int, default=0)
		parser.add_argument('--threads', type=int, default=4)
		parser.add_argument('--device', type=str, default='cuda:0' if torch.cuda.is_available() else 'cpu')   
		
		parser.add_argument('--num_scenario', '-ns', type=int, default=2, help='num of scenarios we run in one episode')
		parser.add_argument('--save_video', action='store_true')
		parser.add_argument('--render', type=bool, default=True)
		parser.add_argument('--frame_skip', '-fs', type=int, default=1, help='skip of frame in each step')
		parser.add_argument('--port', type=int, default=2002, help='port to communicate with carla')
		parser.add_argument('--tm_port', type=int, default=8002, help='traffic manager port')
		parser.add_argument('--fixed_delta_seconds', type=float, default=0.1)
		parser.add_argument('--test_policy', type=str, default='sac')
		parser.add_argument('--test_epoch', type=int, default=None)
		args = parser.parse_args()
		
		err_list = []
		for agent_cfg in args.agent_cfg:
		for scenario_cfg in args.scenario_cfg:
		# set global parameters
		set_torch_variable(args.device)
		torch.set_num_threads(args.threads)
		set_seed(args.seed)
		
		# load agent config
		agent_config_path = osp.join(args.ROOT_DIR, 'safebench/agent/config', agent_cfg)
		agent_config = load_config(agent_config_path)
		agent_config['policy_name'] = args.test_policy
		
		## load the corresponding model ##
		agent_config['load_dir'] = osp.join(agent_config['load_dir'], "dynamic_scenario")
		
		# load scenario config
		scenario_config_path = osp.join(args.ROOT_DIR, 'safebench/scenario/config', scenario_cfg)
		scenario_config = load_config(scenario_config_path)
		
		args.output_dir = osp.join('log', 'adv_train', args.mode, agent_config['policy_name'], f"{agent_cfg.split('.')[0]}_epoch{args.test_epoch}")
		args.exp_name =  "dynamic_scenario"
		args_dict = vars(args)
		# main entry with a selected mode
		agent_config.update(args_dict)
		print(agent_config['load_dir'])
		scenario_config.update(args_dict)
		
		scenario_config['num_scenario'] = 1 # 'the num_scenario can only be one for scenic'
		runner = ScenicRunner(agent_config, scenario_config)
		
		# start running
		try:
		runner.run(args.test_epoch)
		except:
		runner.close()
		traceback.print_exc()
		err_list.append([agent_cfg, scenario_cfg, traceback.format_exc()])
		
		for err in err_list:
		print(err[0], err[1], 'failed!')
		print(err[2])
		
		run\_eval.py
			import setGPU
		import traceback
		import os.path as osp
		
		import torch 
		
		from safebench.util.run_util import load_config
		from safebench.util.torch_util import set_seed, set_torch_variable
		from safebench.carla_runner import CarlaRunner
		
		
		if __name__ == '__main__':
		import argparse
		parser = argparse.ArgumentParser()
		parser.add_argument('--exp_name', type=str, default='exp')
		parser.add_argument('--output_dir', type=str, default='log')
		parser.add_argument('--ROOT_DIR', type=str, default=osp.abspath(osp.dirname(osp.dirname(osp.realpath(__file__)))))
		
		parser.add_argument('--max_episode_step', type=int, default=300)
		parser.add_argument('--auto_ego', action='store_true')
		parser.add_argument('--mode', '-m', type=str, default='eval', choices=['train_agent', 'train_scenario', 'eval'])
		parser.add_argument('--agent_cfg', nargs='*', type=str, default=['adv_scenic.yaml'])
		parser.add_argument('--scenario_cfg', nargs='*', type=str, default=['eval_scenic.yaml'])
		parser.add_argument('--continue_agent_training', '-cat', type=bool, default=False)
		parser.add_argument('--continue_scenario_training', '-cst', type=bool, default=False)
		
		parser.add_argument('--seed', '-s', type=int, default=0)
		parser.add_argument('--threads', type=int, default=4)
		parser.add_argument('--device', type=str, default='cuda:0' if torch.cuda.is_available() else 'cpu')   
		
		parser.add_argument('--num_scenario', '-ns', type=int, default=2, help='num of scenarios we run in one episode')
		parser.add_argument('--save_video', action='store_true')
		parser.add_argument('--render', type=bool, default=True)
		parser.add_argument('--frame_skip', '-fs', type=int, default=1, help='skip of frame in each step')
		parser.add_argument('--port', type=int, default=2002, help='port to communicate with carla')
		parser.add_argument('--tm_port', type=int, default=8002, help='traffic manager port')
		parser.add_argument('--fixed_delta_seconds', type=float, default=0.1)
		parser.add_argument('--test_policy', type=str, default='sac')
		parser.add_argument('--route_id', type=int, default=0)
		parser.add_argument('--scenario_id', type=int, default=0)
		parser.add_argument('--test_epoch', type=int, default=None)
		args = parser.parse_args()
		
		err_list = []
		for agent_cfg in args.agent_cfg:
		for scenario_cfg in args.scenario_cfg:
		# set global parameters
		set_torch_variable(args.device)
		torch.set_num_threads(args.threads)
		set_seed(args.seed)
		
		# load agent config
		agent_config_path = osp.join(args.ROOT_DIR, 'safebench/agent/config', agent_cfg)
		agent_config = load_config(agent_config_path)
		agent_config['policy_name'] = args.test_policy
		
		## load the corresponding model ##
		agent_config['load_dir'] = osp.join(agent_config['load_dir'], f'scenario_{args.scenario_id}')
		
		# load scenario config
		scenario_config_path = osp.join(args.ROOT_DIR, 'safebench/scenario/config', scenario_cfg)
		scenario_config = load_config(scenario_config_path)
		scenario_config['scenario_id'] = args.scenario_id
		
		args.output_dir = osp.join('log', 'adv_train', args.mode, agent_config['policy_name'], f"{agent_cfg.split('.')[0]}_epoch{args.test_epoch}", f"{scenario_cfg.split('.')[0]}")
		args.exp_name = 'scenario_' + str(scenario_config['scenario_id'])
		args_dict = vars(args)
		# main entry with a selected mode
		agent_config.update(args_dict)
		print(agent_config['load_dir'])
		scenario_config.update(args_dict)
		if scenario_config['policy_type'] == 'scenic':
		from safebench.scenic_runner import ScenicRunner
		scenario_config['num_scenario'] = 1 # 'the num_scenario can only be one for scenic'
		scenario_config['route_id'] = [args.route_id]
		runner = ScenicRunner(agent_config, scenario_config)
		else:
		## id shift due to the test settings in safebench v1
		scenario_config['route_id'] = args.route_id + 4
		runner = CarlaRunner(agent_config, scenario_config)
		
		# start running
		try:
		runner.run(args.test_epoch)
		except:
		runner.close()
		traceback.print_exc()
		err_list.append([agent_cfg, scenario_cfg, traceback.format_exc()])
		
		for err in err_list:
		print(err[0], err[1], 'failed!')
		print(err[2])
		


     run\_train.py

	import sys
	print(sys.path)
	
	import setGPU
	import traceback
	import os
	import os.path as osp
	
	import torch 
	from safebench.util.run_util import load_config
	from safebench.util.torch_util import set_seed, set_torch_variable
	from safebench.carla_runner import CarlaRunner
	
	if __name__ == '__main__':
	import argparse
	parser = argparse.ArgumentParser()
	parser.add_argument('--exp_name', type=str, default='exp')
	parser.add_argument('--output_dir', type=str, default='log')
	parser.add_argument('--ROOT_DIR', type=str, default=osp.abspath(osp.dirname(osp.dirname(osp.realpath(__file__)))))
	
	parser.add_argument('--max_episode_step', type=int, default=300)
	parser.add_argument('--auto_ego', action='store_true')
	parser.add_argument('--mode', '-m', type=str, default='eval', choices=['train_scenario', 'train_agent', 'eval'])
	parser.add_argument('--agent_cfg', nargs='*', type=str, default=['adv_scenic.yaml'])
	parser.add_argument('--scenario_cfg', nargs='*', type=str, default=['train_scenario_scenic.yaml'])
	parser.add_argument('--continue_agent_training', '-cat', type=bool, default=False)
	parser.add_argument('--continue_scenario_training', '-cst', type=bool, default=False)
	
	parser.add_argument('--seed', '-s', type=int, default=0)
	parser.add_argument('--threads', type=int, default=4)
	parser.add_argument('--device', type=str, default='cuda:0' if torch.cuda.is_available() else 'cpu')   
	
	parser.add_argument('--num_scenario', '-ns', type=int, default=1, help='num of scenarios we run in one episode')
	parser.add_argument('--save_video', action='store_true')
	parser.add_argument('--render', type=bool, default=True)
	parser.add_argument('--frame_skip', '-fs', type=int, default=1, help='skip of frame in each step')
	parser.add_argument('--port', type=int, default=2002, help='port to communicate with carla')
	parser.add_argument('--tm_port', type=int, default=8002, help='traffic manager port')
	parser.add_argument('--fixed_delta_seconds', type=float, default=0.1)
	parser.add_argument('--scenario_id', type=int, default=None)
	args = parser.parse_args()
	args_dict = vars(args)
	
	err_list = []
	for agent_cfg in args.agent_cfg:
	for scenario_cfg in args.scenario_cfg:
	# set global parameters
	set_torch_variable(args.device)
	torch.set_num_threads(args.threads)
	set_seed(args.seed)
	
	# load agent config
	agent_config_path = osp.join(args.ROOT_DIR, 'safebench/agent/config', agent_cfg)
	agent_config = load_config(agent_config_path)
	
	# load scenario config
	scenario_config_path = osp.join(args.ROOT_DIR, 'safebench/scenario/config', scenario_cfg)
	scenario_config = load_config(scenario_config_path)
	
	## modification
	if args.scenario_id:
	scenario_config['scenario_id'] = args.scenario_id
	
	agent_config['load_dir'] = osp.join(agent_config['load_dir'], f"scenario_{scenario_config['scenario_id']}")
	# Check if the directory exists; if not, create it
	if not osp.exists(agent_config['load_dir']):
	os.makedirs(agent_config['load_dir'])        
	
	# main entry with a selected mode
	agent_config.update(args_dict)
	args_dict['output_dir'] = osp.join('log', 'adv_train', args.mode, agent_config['policy_name'], f"{agent_cfg.split('.')[0]}", f"scenario_{scenario_config['scenario_id']}")
	scenario_config.update(args_dict)
	if scenario_config['policy_type'] == 'scenic':
	from safebench.scenic_runner import ScenicRunner
	scenario_config['num_scenario'] = 1 ### 'the num_scenario can only be one for scenic'
	runner = ScenicRunner(agent_config, scenario_config)
	else:
	runner = CarlaRunner(agent_config, scenario_config)
	
	# start running
	runner.run()
	
	for err in err_list:
	print(err[0], err[1], 'failed!')
	print(err[2])
\end{lstlisting}


\subsection{其他}
\begin{lstlisting}
config.py:配置文件



		import carla
		
		import argparse
		import datetime
		import re
		import socket
		import textwrap
		
		
		def get_ip(host):
		if host in ['localhost', '127.0.0.1']:
		sock = socket.socket(socket.AF_INET, socket.SOCK_DGRAM)
		try:
		sock.connect(('10.255.255.255', 1))
		host = sock.getsockname()[0]
		except RuntimeError:
		pass
		finally:
		sock.close()
		return host
		
		
		def find_weather_presets():
		presets = [x for x in dir(carla.WeatherParameters) if re.match('[A-Z].+', x)]
		return [(getattr(carla.WeatherParameters, x), x) for x in presets]
		
		
		def list_options(client):
		maps = [m.replace('/Game/Carla/Maps/', '') for m in client.get_available_maps()]
		indent = 4 * ' '
		def wrap(text):
		return '\n'.join(textwrap.wrap(text, initial_indent=indent, subsequent_indent=indent))
		print('weather presets:\n')
		print(wrap(', '.join(x for _, x in find_weather_presets())) + '.\n')
		print('available maps:\n')
		print(wrap(', '.join(sorted(maps))) + '.\n')
		
		
		def list_blueprints(world, bp_filter):
		blueprint_library = world.get_blueprint_library()
		blueprints = [bp.id for bp in blueprint_library.filter(bp_filter)]
		print('available blueprints (filter %r):\n' % bp_filter)
		for bp in sorted(blueprints):
		print('    ' + bp)
		print('')
		
		
		def inspect(args, client):
		address = '%s:%d' % (get_ip(args.host), args.port)
		
		world = client.get_world()
		elapsed_time = world.get_snapshot().timestamp.elapsed_seconds
		elapsed_time = datetime.timedelta(seconds=int(elapsed_time))
		
		actors = world.get_actors()
		s = world.get_settings()
		
		weather = 'Custom'
		current_weather = world.get_weather()
		for preset, name in find_weather_presets():
		if current_weather == preset:
		weather = name
		
		if s.fixed_delta_seconds is None:
		frame_rate = 'variable'
		else:
		frame_rate = '%.2f ms (%d FPS)' % (
		1000.0 * s.fixed_delta_seconds,
		1.0 / s.fixed_delta_seconds)
		
		print('-' * 34)
		print('address:% 26s' % address)
		print('version:% 26s\n' % client.get_server_version())
		print('map:        % 22s' % world.get_map().name)
		print('weather:    % 22s\n' % weather)
		print('time:       % 22s\n' % elapsed_time)
		print('frame rate: % 22s' % frame_rate)
		print('rendering:  % 22s' % ('disabled' if s.no_rendering_mode else 'enabled'))
		print('sync mode:  % 22s\n' % ('disabled' if not s.synchronous_mode else 'enabled'))
		print('actors:     % 22d' % len(actors))
		print('  * spectator:% 20d' % len(actors.filter('spectator')))
		print('  * static:   % 20d' % len(actors.filter('static.*')))
		print('  * traffic:  % 20d' % len(actors.filter('traffic.*')))
		print('  * vehicles: % 20d' % len(actors.filter('vehicle.*')))
		print('  * walkers:  % 20d' % len(actors.filter('walker.*')))
		print('-' * 34)
		
		
		def main():
		argparser = argparse.ArgumentParser(
		description=__doc__)
		argparser.add_argument(
		'--host',
		metavar='H',
		default='localhost',
		help='IP of the host CARLA Simulator (default: localhost)')
		argparser.add_argument(
		'-p', '--port',
		metavar='P',
		default=2000,
		type=int,
		help='TCP port of CARLA Simulator (default: 2000)')
		argparser.add_argument(
		'-d', '--default',
		action='store_true',
		help='set default settings')
		argparser.add_argument(
		'-m', '--map',
		help='load a new map, use --list to see available maps')
		argparser.add_argument(
		'-r', '--reload-map',
		action='store_true',
		help='reload current map')
		argparser.add_argument(
		'--delta-seconds',
		metavar='S',
		type=float,
		help='set fixed delta seconds, zero for variable frame rate')
		argparser.add_argument(
		'--fps',
		metavar='N',
		type=float,
		help='set fixed FPS, zero for variable FPS (similar to --delta-seconds)')
		argparser.add_argument(
		'--rendering',
		action='store_true',
		help='enable rendering')
		argparser.add_argument(
		'--no-rendering',
		action='store_true',
		help='disable rendering')
		argparser.add_argument(
		'--no-sync',
		action='store_true',
		help='disable synchronous mode')
		argparser.add_argument(
		'--weather',
		help='set weather preset, use --list to see available presets')
		argparser.add_argument(
		'-i', '--inspect',
		action='store_true',
		help='inspect simulation')
		argparser.add_argument(
		'-l', '--list',
		action='store_true',
		help='list available options')
		argparser.add_argument(
		'-b', '--list-blueprints',
		metavar='FILTER',
		help='list available blueprints matching FILTER (use \'*\' to list them all)')
		argparser.add_argument(
		'-x', '--xodr-path',
		metavar='XODR_FILE_PATH',
		help='load a new map with a minimum physical road representation of the provided OpenDRIVE')
		argparser.add_argument(
		'--osm-path',
		metavar='OSM_FILE_PATH',
		help='load a new map with a minimum physical road representation of the provided OpenStreetMaps')
		argparser.add_argument(
		'--tile-stream-distance',
		metavar='N',
		type=float,
		help='Set tile streaming distance (large maps only)')
		argparser.add_argument(
		'--actor-active-distance',
		metavar='N',
		type=float,
		help='Set actor active distance (large maps only)')
		if len(sys.argv) < 2:
		argparser.print_help()
		return
		
		args = argparser.parse_args()
		
		client = carla.Client(args.host, args.port, worker_threads=1)
		client.set_timeout(10.0)
		
		if args.default:
		args.rendering = True
		args.delta_seconds = 0.0
		args.weather = 'Default'
		args.no_sync = True
		
		if args.map is not None:
		print('load map %r.' % args.map)
		world = client.load_world(args.map)
		elif args.reload_map:
		print('reload map.')
		world = client.reload_world()
		elif args.xodr_path is not None:
		if os.path.exists(args.xodr_path):
		with open(args.xodr_path, encoding='utf-8') as od_file:
		try:
		data = od_file.read()
		except OSError:
		print('file could not be readed.')
		sys.exit()
		print('load opendrive map %r.' % os.path.basename(args.xodr_path))
		vertex_distance = 2.0  # in meters
		max_road_length = 500.0 # in meters
		wall_height = 1.0      # in meters
		extra_width = 0.6      # in meters
		world = client.generate_opendrive_world(
		data, carla.OpendriveGenerationParameters(
		vertex_distance=vertex_distance,
		max_road_length=max_road_length,
		wall_height=wall_height,
		additional_width=extra_width,
		smooth_junctions=True,
		enable_mesh_visibility=True))
		else:
		print('file not found.')
		elif args.osm_path is not None:
		if os.path.exists(args.osm_path):
		with open(args.osm_path, encoding='utf-8') as od_file:
		try:
		data = od_file.read()
		except OSError:
		print('file could not be readed.')
		sys.exit()
		print('Converting OSM data to opendrive')
		xodr_data = carla.Osm2Odr.convert(data)
		print('load opendrive map.')
		vertex_distance = 2.0  # in meters
		max_road_length = 500.0 # in meters
		wall_height = 0.0      # in meters
		extra_width = 0.6      # in meters
		world = client.generate_opendrive_world(
		xodr_data, carla.OpendriveGenerationParameters(
		vertex_distance=vertex_distance,
		max_road_length=max_road_length,
		wall_height=wall_height,
		additional_width=extra_width,
		smooth_junctions=True,
		enable_mesh_visibility=True))
		else:
		print('file not found.')
		
		else:
		world = client.get_world()
		
		settings = world.get_settings()
		
		if args.no_rendering:
		print('disable rendering.')
		settings.no_rendering_mode = True
		elif args.rendering:
		print('enable rendering.')
		settings.no_rendering_mode = False
		
		if args.no_sync:
		print('disable synchronous mode.')
		settings.synchronous_mode = False
		
		if args.delta_seconds is not None:
		settings.fixed_delta_seconds = args.delta_seconds
		elif args.fps is not None:
		settings.fixed_delta_seconds = (1.0 / args.fps) if args.fps > 0.0 else 0.0
		
		if args.delta_seconds is not None or args.fps is not None:
		if settings.fixed_delta_seconds > 0.0:
		print('set fixed frame rate %.2f milliseconds (%d FPS)' % (
		1000.0 * settings.fixed_delta_seconds,
		1.0 / settings.fixed_delta_seconds))
		else:
		print('set variable frame rate.')
		settings.fixed_delta_seconds = None
		
		if args.tile_stream_distance is not None:
		settings.tile_stream_distance = args.tile_stream_distance
		if args.actor_active_distance is not None:
		settings.actor_active_distance = args.actor_active_distance
		
		world.apply_settings(settings)
		
		if args.weather is not None:
		if not hasattr(carla.WeatherParameters, args.weather):
		print('ERROR: weather preset %r not found.' % args.weather)
		else:
		print('set weather preset %r.' % args.weather)
		world.set_weather(getattr(carla.WeatherParameters, args.weather))
		
		if args.inspect:
		inspect(args, client)
		if args.list:
		list_options(client)
		if args.list_blueprints:
		list_blueprints(world, args.list_blueprints)
		
		
		if __name__ == '__main__':
		
		try:
		
		main()
		
		except KeyboardInterrupt:
		print('\nCancelled by user. Bye!')
		except RuntimeError as e:
		print(e)
		
\end{lstlisting}