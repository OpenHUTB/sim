\chapter{附录:系统模块源代码}

\subsection*{自然语言理解与场景生成模块}
\begin{enumerate}
	\item \texttt{retrieve.py}
	\begin{itemize}
		\item 该文件是系统的主文件,负责从自然语言描述中生成场景代码。它集成了自然语言处理和检索增强模块,通过调用大语言模型生成符合输入语义的Scenic场景脚本。
	\end{itemize}
	\begin{verbatim}
		
		
	\end{verbatim}
\end{enumerate}

\subsection*{场景合成与仿真模块}
\begin{enumerate}
	\item \texttt{run\_train\_dynamic.py}
	\begin{itemize}
		\item 作用:用于在动态生成的场景上训练代理。该文件使用 \texttt{dynamic\_scenic.yaml} 进行配置,并运行训练过程,优化代理的行为。
	\end{itemize}
	\begin{verbatim}
		
		
		
	\end{verbatim}
	
	\item \texttt{dynamic\_scenic.yaml}
	\begin{itemize}
		\item 作用:该文件是代理的配置文件,包含了代理的训练设置,包括其行为模型、对抗性行为、场景属性等。它与其他 YAML 文件结合使用来指定代理在不同场景中的行为。
	\end{itemize}
	\begin{verbatim}
		
		
	\end{verbatim}
\end{enumerate}

\subsection*{评估与展示模块}
\begin{enumerate}
	\item \texttt{evaluate\_scene\_quality.py}
	\begin{itemize}
		\item 该文件负责对生成的场景进行量化评估,输出语义保真度、多样性与驾驶性能相关指标。
	\end{itemize}
	\begin{verbatim}
		
		
	\end{verbatim}
	
	\item \texttt{run\_eval\_dynamic.py}
	\begin{itemize}
		\item 该文件用于运行评估流程,调用 \texttt{evaluate\_scene\_quality.py} 对生成的场景进行评估。
	\end{itemize}
	\begin{verbatim}
		
		
	\end{verbatim}
\end{enumerate}