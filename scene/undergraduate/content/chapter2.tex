\chapter{理论基础}

\subsection{预训练语言模型与代码生成机制}

预训练语言模型也就是 Pretrained Language Models(PLMs),它是基于大规模语料来学习语言分布规律的通用建模方法,其核心要点是通过自监督学习掌握词序列的统计规律,并且在这个基础上实现对下游任务的泛化能力,在生成结构化代码的任务当中,预训练语言模型会先理解自然语言的语义意图,然后将其映射为具有严格语法和语义约束的目标语言(像 Scenic)的代码表示,此过程涉及对语言结构、上下文关系以及目标语言语法规则的综合建模能力。

\subsubsection{Transformer 与大语言模型的理论机制}

Transformer是现在主流语言模型所采用的基础结构,其核心是自注意力机制也就是Self - Attention,主要用来捕捉输入序列里任意两个位置之间的依赖关系,在编码过程当中,输入序列首先会通过嵌入也就是Embedding层转化成向量表示,接着会通过多层的自注意力网络开展上下文建模,在每一层里面,模型会依据Query - Key - Value结构来计算注意力权重:

\begin{equation}
	\text{Attention}(Q, K, V) = \text{softmax}\left( \frac{QK^\top}{\sqrt{d_k}} \right)V
\end{equation}

在这其中 $Q$、$K$、$V$ 分别表示查询(Query)、键(Key)和值(Value)矩阵,$d_k$ 为维度归一化因子。这个机制可以在任意位置之间建立联系,从而有效捕捉长距离依赖。

在生成模型当中像GPT系列这类,Transformer被当作自回归结构来使用,当前token的表示要依赖之前所有token的上下文,目标是去预测下一个token出现的概率,经过大规模语料的训练后模型能够学习到语言里的统计规律、语法模式以及推理结构

\subsubsection{自然语言生成结构化代码的建模逻辑}

把自然语言变成结构化代码的这个过程,本质上是把输入的非形式化且语义丰富自然语言映射到有严格语法和执行逻辑的目标代码语言里,这个映射包含着三个核心理论组件:

\begin{enumerate}
	\item \textbf{语义建模(Semantic Modeling)}:语言模型要先理解自然语言背后的意图,也就是用户描述里蕴含的操作对象、属性约束、空间关系和时间先后等语义信息,这部分依靠语言模型对上下文的长距离建模能力以及对词汇的语义表示能力 。
	
	\item \textbf{语法约束建模(Syntactic Constraint Modeling)}:目标语言像Scenic这类有着明确的语法规则以及结构限制,语言模型在预训练过程里学习了多种语言模式,还能内隐地掌握特定语言的文法结构来生成合法的结构化代码。
	
	\item \textbf{结构对齐与表示变换(Structure Alignment)}:自然语言可以让人自由地进行表达,而结构化代码属于高度约束的形式语言。语言模型借助自回归生成的过程,把自然语言里的结构信息逐步转化成代码片段,达成从开放式表述到约束式语言的对齐映射。
\end{enumerate}

需要注意的是语言模型并非通过硬编码规则做语义转换,而是凭借大规模预训练获得的统计语言知识,在解码阶段依靠概率最大化来实现语言到代码的结构生成,这种方法具备通用性强和适应性高的优势,适合用于多领域的自然语言编程任务。
另外在结构生成的过程当中,可以引入显式语义约束机制或者模板结构,像代码前缀提示这类,以此进一步提升生成代码的可控性与可解释性。
\subsection{自然语言到交通场景的语义建模}

把自然语言描述转变为可执行的交通仿真场景,关键之处在于对语义信息做结构化建模,交通场景一般有着复杂的空间布局、多主体互动且时序性强等特征,所以要从自然语言里精准提取出参与实体、行为属性、空间关系以及时间逻辑,并且构建能被场景引擎识别和执行的语义表示结构,本节围绕场景语义结构分析与关键元素抽取、时序关系与交通意图建模这两个方面展开论述。

\subsubsection{场景语义结构分析与关键元素抽取}

交通场景里的自然语言一般会包含多个核心组成部分,像交通参与者(例如车辆、行人、自行车等)、空间约束(比如道路类型、位置关系)、行为动作(像是“行驶”“等待”“穿越”)以及环境条件(例如天气、时间、信号灯状态),语义结构分析的目标是从原始语句当中建立出如下形式的结构化语义表示。

\begin{equation}
	\text{Scene} = \{ \text{Agents}, \text{Actions}, \text{Locations}, \text{Relations}, \text{Conditions} \}
\end{equation}

其中:
\begin{itemize}
	\item \textbf{Agents}:参与主体及其类型(如一辆蓝色轿车、自行车、行人等);
	\item \textbf{Actions}:行为描述,如“驶入交叉口”、“等待信号”、“加速通过”等;
	\item \textbf{Locations}:空间位置或背景信息,如“在红绿灯前”、“十字路口中央”、“左侧车道”;
	\item \textbf{Relations}:主体之间的空间和逻辑关系,如“位于...左侧”、“跟随...行驶”;
	\item \textbf{Conditions}:上下文条件,如“在夜间”、“红灯状态”、“雨天”等。
\end{itemize}

抽取过程是依靠语言模型对上下文依存关系的建模能力来开展的,同时配合命名实体识别(NER)、依存句法分析(Dependency Parsing)等语言学工具,能够形成初步的语义结构,在本系统当中,语义结构会进一步映射为Scenic或Carla里支持的场景构造要素,以此实现语义向结构化语言的桥接。

\subsubsection{时序关系与交通意图建模}

交通场景可不只是静态配置这么简单,还涉及高度动态化的行为序列以及意图表达,自然语言里常常包含清晰或者隐含的时间逻辑,像“接着”“在……之后”“同时”这类信息,对于还原真实交通过程起着关键作用,时序关系建模的目标是把事件按照逻辑顺序组织成一个有向图或者序列,每个事件节点包含主体、动作以及发生时间,具体形式如下:
\begin{equation}
	\text{Timeline} = \left[ e_1 \rightarrow e_2 \rightarrow \cdots \rightarrow e_n \right], \quad e_i = (\text{Agent}, \text{Action}, \text{Time})
\end{equation}

借助语言模型对事件触发词以及像“当……时”“随后”这类连接词的建模能力,能够自动推理出事件之间的因果和先后关系并形成场景执行逻辑路径,另外交通意图建模所关注的是参与主体行为背后的目标导向性,比如“车辆试图避让行人”这种表述不仅描述了一个行为,还隐含着“行人优先”的交通规则与主车的决策意图,这类意图信息对于像 Carla 的行为树或触发机制这样的场景控制逻辑有着重要价值,可通过对谓词、助动词以及上下文环境进行深度理解来实现建模。

\subsection{Scenic语言的形式化语义与表达能力}

Scenic语言是专门面向智能驾驶场景的描述性语言,它主要目标是借助简洁且强大的语法结构,把复杂交通场景转化成可执行的仿真代码,为让语言具备高效表达能力和高度可扩展性,Scenic语言将形式化的语法设计和概率建模结合起来,通过对交通场景进行准确描述,能够生成包含多主体且行为多样化的复杂仿真环境,本节会从Scenic的语法设计与概率建模、Scenic在智能驾驶场景中的优势分析这两个方面展开详细讨论。
\subsubsection{Scenic的语法设计与概率建模}

Scenic语言语法设计是有效表达交通场景的基础,Scenic核心在于用简单面向对象描述方式,支持对交通参与者及空间位置等灵活建模,其语法结构支持层次化描述能精准构建场景元素,Scenic语法设计遵循以下几个基本原则:

\begin{itemize}
	\item \textbf{面向对象建模:} Scenic通过对“对象”进行定义,像车辆、行人这类,同时也对“对象属性”加以明确,比如位置、速度、尺寸这些,以此来构建一个高层次的交通场景。
	\item \textbf{行为定义:} 系统支持对交通参与者的动作(像“加速”“刹车”这类动作)以及事件(例如“碰撞”这样的事件)进行定义。
	\item \textbf{空间约束与条件:} Scenic能够对空间约束进行定义,像“在道路上”这种,还能定义基于条件的约束,例如“当交通信号为红色时”。
\end{itemize}

Scenic的关键特性之一是支持概率建模,允许在场景生成时引入随机性和不确定性。通过概率分布的引入,Scenic能够模拟复杂的交通情境,生成多样化且高度不确定的场景。例如,车辆的初始位置、速度、甚至行为都可以通过概率分布进行建模,以更好地贴近真实世界中交通流的动态性。

\begin{equation}
	P(\text{Scene}) = \int_{\Omega} P(\text{Elements} | \text{Conditions}) d\Omega
\end{equation}

在此公式中,$P(\text{Scene})$ 表示生成场景的概率,$\text{Elements}$ 代表场景中各种元素(如车辆、行人等),而 $\Omega$ 表示所有可能的场景配置空间。通过这种概率建模,Scenic可以在多种不确定性下生成具有高度真实性和多样性的场景。

\subsubsection{Scenic在智能驾驶场景中的优势分析}

Scenic语言的设计给智能驾驶领域的场景生成带来明显优势,下面从多个维度分析Scenic在智能驾驶仿真里的独特优势:

\begin{itemize}
	\item \textbf{高效的场景描述与复用:} Scenic凭借高度抽象化的语法以及简洁的模型表达,极大程度提高了交通场景的描述效率,场景具备很强的重用性,不同场景能够通过微小的语法变更来进行扩展和组合,可满足智能驾驶测试里对多样化场景的需求。
	
	\item \textbf{动态行为与时序控制:} Scenic可以在场景里嵌入复杂的时序关系,能准确模拟多个主体之间的互动情况,就像车辆与行人之间的避让行为,又或是车辆在不同时间点的动作序列,都能够依靠简洁的代码来实现,这让Scenic成了模拟交通意图和决策过程的重要工具。
	
	\item \textbf{概率建模提升真实性:} 通过引入随机性与不确定性,Scenic能够模拟出更贴近真实世界的交通场景,在生成多个交通参与者行为的时候,场景生成不局限于确定性条件,而是可包含行为模式的随机变化,这对测试自动驾驶系统的适应性十分重要。
	
	\item \textbf{与仿真平台的兼容性:} Scenic语言和多种智能驾驶仿真平台像CARLA有良好兼容性,借助Scenic语言用户能生成符合平台要求的场景代码,可直接和仿真环境对接开展测试工作,避免传统场景构建里繁琐的人工编码操作。
	
	\item \textbf{可扩展性与灵活性:} Scenic语言能够生成基于规则的固定场景,还能借助引入外部因素、环境变化以及多主体交互等元素扩展出更复杂场景,不管是常规的城市街道、环路,还是极端天气条件、突发事件等情况都可通过Scenic进行灵活建模。
\end{itemize}
\subsection{智能驾驶仿真平台CARLA与集成机制}

智能驾驶仿真平台CARLA也就是Car Learning to Act ,它是一款开源的高保真自动驾驶模拟器,在自动驾驶系统的开发和测试方面有着广泛应用。CARLA具备丰富的环境建模功能,能够模拟各种天气、时间、交通和驾驶情况。正因为如此,它成了自动驾驶仿真里不可或缺的工具之一。本节会介绍CARLA的场景接入机制,还会阐述其与Scenic的联合仿真流程设计。
\subsubsection{CARLA场景接入机制}

CARLA的场景接入机制主要是和外部代码或者仿真系统做对接,用户借助CARLA的API能够自定义并且加载复杂的交通场景,还可在这个基础上开展仿真测试,具体的接入方式包含如下内容:

\begin{itemize}
	\item \textbf{场景数据导入:}CARLA能够支持借助Python API和外部程序开展交互,用户可通过Python脚本动态生成相应场景,还能控制交通参与者像车辆行人的位置行为和状态,除此之外CARLA也提供了实时数据流接口,支持获取如摄像头雷达等传感器数据并反馈到控制系统 。
	
	\item \textbf{API与脚本接口:} CARLA给用户提供了一套较为完善的API和脚本接口,其支持通过编程来生成、控制以及操作场景元素,这些接口涵盖控制车辆运动、改变天气状况、管理交通信号等功能,可依据不同的仿真需求对场景进行实时调整。
	
	\item \textbf{物理引擎与碰撞检测:}CARLA的物理引擎能提供高度仿真的车辆动力学以及碰撞检测机制,可精确模拟车辆行为、道路条件和交通情况,这些物理特性保证了仿真环境具备真实性和高效性,为智能驾驶系统的测试提供了坚实的基础。
\end{itemize}

借助上面所提到的机制,CARLA可以和其他系统像Scenic语言进行紧密集成,以此实现更为复杂的智能驾驶场景仿真。

\subsubsection{与Scenic的联合仿真流程设计}

为了提升自动驾驶测试的自动化程度以及场景多样性,本文给出了CARLA与Scenic的联合仿真具体流程如下:

\begin{enumerate}
	\item \textbf{场景描述生成:} 用户使用自然语言对智能驾驶场景进行描述,比如“自车在红绿灯前等待信号”或者“行人横穿街道”,这些自然语言描述会被传递到Scenic语言模块,进而生成对应的场景代码。
	
	\item \textbf{Scenic场景转换:} Scenic能够把自然语言描述转化成可执行的场景代码,这些代码涵盖交通参与者的位置、行为、交互以及时间序列等方面内容。
	
	\item \textbf{场景加载到CARLA:} 转换之后的Scenic场景代码借助CARLA的API接口加载到仿真环境里,此时CARLA会依据场景代码对相应的道路、交通标志以及行人等元素进行配置并初始化仿真环境。
	
	\item \textbf{仿真执行与数据采集:} CARLA开启场景仿真的执行工作,同时借助摄像头、激光雷达等传感器数据反馈仿真结果,这些数据会被用于后续的分析与评估工作。
	
	\item \textbf{仿真结果评估:}通过对仿真过程当中的数据进行分析,来评估智能驾驶系统在特定场景之下的表现,并且进一步优化场景生成以及测试的流程。
\end{enumerate}

这一联合仿真流程借助无缝集成Scenic语言和CARLA平台,让智能驾驶场景生成变得更高效、准确且多样化,可更好地支持自动驾驶系统的验证与优化工作。

\subsection{本文方法的创新点与理论意义}

本文提出基于预训练大语言模型和Scenic语言的高保真智能驾驶场景生成系统,其目的是解决传统交通场景构建方法效率瓶颈问题,推动自动驾驶系统智能化和自动化测试进程。本文方法具备如下创新点与理论意义:

\subsubsection{创新点}

\begin{itemize}
	\item \textbf{自然语言到场景代码的自动化转换:} 本文第一次把自然语言描述和Scenic场景生成系统结合起来,提出基于大语言模型的自然语言到交通场景转换方法,利用这种方法,用户只需提供简洁的场景描述内容,系统就能自动生成符合交通规则和驾驶逻辑仿真场景代码,大大降低人工干预和时间方面的成本。
	
	\item \textbf{基于概率建模的场景多样性生成:} 这篇文章在Scenic语言里引入了概率建模的方法,对不同交通情境所具有的随机性和不确定性进行模拟,让生成出来的交通场景变得更加多样化丰富化,能够全方位评估自动驾驶系统在各种各样不同情况下的实际表现。
	
	\item \textbf{CARLA与Scenic的高效集成:} 本文专门设计出CARLA与Scenic的联合仿真流程让两个系统高效协同工作,借助这一集成机制用户可在CARLA平台上快速执行复杂交通场景仿真且收集反馈数据做进一步分析。
	
	\item \textbf{智能驾驶系统的量化评估:} 本文提出来一种综合性的评估框架,这个框架能够从语义保真度、准确性与效率等多个维度,对生成出来的智能驾驶场景进行量化评估工作,为自动驾驶系统的性能分析提供更科学客观的评估标准。
\end{itemize}
