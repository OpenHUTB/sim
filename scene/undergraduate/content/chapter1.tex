% 第一章 绪论
\chapter{绪论}
\section{研究背景}
\subsection{智能驾驶测试对海量长尾场景的需求矛盾}
随着智能驾驶技术的飞速发展,其测试需求也日益复杂和多样化。智能驾驶系统需要在各种复杂多变的交通场景下表现出可靠的性能,以确保驾驶的安全性和舒适性。然而,现实交通环境中存在着海量的长尾场景,这些场景虽然出现频率较低,但一旦发生,往往会对智能驾驶系统构成严峻挑战。
长尾场景的复杂性主要体现在以下几个方面。首先,交通环境的动态性极强,车辆、行人、非机动车等多种交通参与者的行为模式千变万化。例如,在城市道路中,行人可能会突然横穿马路,非机动车可能会随意变道或逆行,这些行为都可能导致潜在的碰撞风险。智能驾驶系统必须能够准确感知和预测这些行为,并做出合理的决策。其次,道路条件的多样性也增加了测试场景的复杂性。从城市快速路到乡村小道,从高速公路到山区道路,不同道路的几何形状、路面状况、交通标志和信号灯等都有所不同。智能驾驶系统需要在这些不同的道路条件下都能正常运行。此外,天气条件和光照条件的变化也会对智能驾驶系统的感知和决策产生影响。雨、雪、雾等恶劣天气会降低传感器的性能,而不同的光照条件(如强光、弱光、逆光等)会影响视觉系统的识别效果。
为了确保智能驾驶系统的安全性和可靠性,测试过程中需要覆盖尽可能多的长尾场景。然而,这面临着巨大的挑战。一方面,长尾场景的数量庞大,几乎无法穷尽。要完全覆盖所有可能的场景几乎是不可能的任务。另一方面,这些场景的出现频率较低,难以在实际道路测试中频繁遇到。因此,传统的测试方法往往难以满足智能驾驶系统对海量长尾场景的测试需求,导致测试的充分性和有效性受到限制。

\subsection{传统场景构建方法的人力成本与效率瓶颈}
在智能驾驶测试领域,传统的场景构建方法主要依赖于人工设计和开发。这些方法虽然在一定程度上能够满足测试需求,但也存在着显著的局限性。
首先,人工设计场景需要大量的专业人员投入。这些人员需要具备深厚的交通工程、计算机科学和智能驾驶技术等多学科知识,能够准确理解和模拟各种复杂的交通场景。然而,这样的人才相对稀缺,培养成本高昂。而且,随着测试需求的不断增加,对专业人员的需求也在持续增长,这进一步加剧了人力成本的压力。
其次,人工设计场景的效率较低。设计一个复杂的交通场景需要经过需求分析、场景建模、代码编写和调试等多个步骤。每个步骤都需要耗费大量的时间和精力。例如,在场景建模阶段,需要精确地定义道路的几何形状、交通参与者的初始位置和行为模式等。在代码编写阶段,需要将这些模型转化为可执行的代码,这不仅需要专业的编程技能,还需要反复调试以确保代码的正确性和稳定性。因此,人工设计场景的速度远远无法满足智能驾驶系统快速迭代和测试的需求。
此外,人工设计场景的准确性和一致性也难以保证。由于不同设计人员的理解和经验不同,可能会导致设计出的场景存在差异。而且,在复杂的场景中,人工设计容易遗漏一些关键的细节,从而影响测试结果的准确性和可靠性。

\subsection{大语言模型在代码生成领域的突破性进展}
近年来,大语言模型(LLM)在自然语言处理领域取得了巨大的突破,并逐渐拓展到代码生成等领域。大语言模型通过在海量文本数据上进行预训练,学习到了语言的语法、语义和逻辑结构,能够生成自然流畅且符合逻辑的文本内容。这种能力为解决智能驾驶测试场景构建的难题提供了新的思路。
在代码生成领域,大语言模型已经展现出了强大的能力。通过对代码数据的学习,大语言模型能够理解代码的结构和逻辑,生成符合语法规范的代码片段。例如,一些基于大语言模型的代码生成工具可以根据用户输入的自然语言描述,自动生成相应的代码实现。这些工具已经在软件开发领域得到了广泛的应用,显著提高了代码开发的效率和质量。
大语言模型在代码生成中的优势主要体现在以下几个方面。首先,它能够快速生成代码,大大缩短了开发周期。传统的人工编写代码需要经过需求分析、设计、编码和调试等多个阶段,每个阶段都需要耗费大量的时间。而大语言模型可以根据输入的描述直接生成代码,减少了中间环节,提高了开发效率。其次,大语言模型生成的代码质量较高。它能够学习到代码的最佳实践和规范,生成的代码不仅符合语法规范,还具有良好的可读性和可维护性。此外,大语言模型还能够根据不同的需求生成多样化的代码实现,为开发者提供了更多的选择。
将大语言模型应用于智能驾驶测试场景构建,可以充分利用其在代码生成领域的优势,解决传统方法面临的困境。通过将自然语言描述的测试场景需求转化为代码实现,大语言模型可以快速生成高保真的测试场景,提高场景构建的效率和质量。同时,结合智能驾驶领域的专业知识,还可以进一步优化大语言模型的性能,使其更好地适应智能驾驶测试场景构建的需求。


\section{研究现状}
\subsection{自然语言到形式化描述转换技术(NL2Formal)}
自然语言到形式化描述转换技术(NL2Formal)是自然语言处理(NLP)领域的一个重要研究方向,其目标是将自然语言描述的意图或需求转化为精确的形式化语言,以便计算机能够理解和执行。在自动驾驶测试场景生成中,NL2Formal技术尤为重要,因为它可以将测试工程师的自然语言描述转化为精确的测试场景代码,从而提高测试效率和准确性。早期的NL2Formal技术主要基于规则,通过预定义的规则或语义解析器来理解自然语言查询并将其转换为形式化语言。这种方法的优点是规则明确,易于理解和实现,但缺点是灵活性差,难以处理复杂的自然语言现象。随着神经网络技术的发展,基于神经网络的方法逐渐成为主流。序列到序列模型和图神经网络被广泛应用于自然语言到SQL查询的转换任务中,取得了显著的进展。这些方法通过学习自然语言和形式化语言之间的映射关系,能够生成更准确的形式化代码。近年来,预训练语言模型(如BERT和T5)的出现进一步推动了NL2Formal技术的发展。这些模型通过在大规模文本数据上进行预训练,学习到了语言的语法、语义和逻辑结构,能够生成高质量的形式化语言。最新的研究则进入了大型语言模型(LLMs)时代,LLMs具有卓越的语言理解和生成能力,可以通过提示(prompting)技术执行NL2Formal任务,生成高质量的形式化代码。这些技术的不断发展和应用,为自动驾驶测试场景的高效生成提供了新的可能性。

\subsection{Scenic语言在自动驾驶仿真中的应用}
Scenic是一种用于自动驾驶仿真测试的形式化场景描述语言,它允许测试工程师以简洁、直观的方式描述复杂的交通场景。Scenic语言的核心优势在于其强大的表达能力和灵活性,能够精确地定义交通参与者的初始位置、行为模式以及环境条件。在自动驾驶仿真中,Scenic语言的应用主要体现在场景建模、代码生成和场景合成等方面。通过Scenic语言,测试工程师可以定义车辆的初始位置、速度、加速度以及行人横穿马路的行为模式,从而构建出高度复杂的交通场景。Scenic语言的代码生成机制能够将场景描述转化为可执行的仿真代码,这使得测试工程师可以专注于场景的逻辑设计,而无需关心底层的代码实现。此外,Scenic语言支持场景的合成和组合,可以将多个简单的场景组合成复杂的场景,大大提高了场景生成的效率,同时也保证了场景的多样性和复杂性。Scenic语言的这些特性使其在自动驾驶仿真测试中得到了广泛应用,成为自动驾驶测试领域的重要工具之一。

\subsection{场景生成质量评估方法研究进展}
场景生成质量评估是自动驾驶测试中的一个重要环节,它直接影响到测试结果的可靠性和有效性。近年来,随着自动驾驶技术的发展,场景生成质量评估方法也取得了显著进展。目前,场景生成质量评估方法主要集中在功能质量、美学质量和人因质量等方面。功能质量评估从场景物件的使用性与功能关联出发,通过可达性、可见性、开阔度、关联关系和功能比例等指标来衡量场景的功能质量。美学质量评估则基于场景平面构图和人视点视野,使用平衡性、齐整度与和谐性等指标来度量场景的美学质量。人因质量评估根据人类活动模拟,采用流线合理性和活动舒适度等指标评估人因质量。这些评估方法从不同角度对场景生成质量进行了全面的评估,为自动驾驶测试场景的优化提供了重要的参考。除了上述评估指标,还有一些研究提出了基于深度学习的评估方法。通过构建深度神经网络模型,自动学习场景生成质量的评估标准,这些方法能够自动识别场景中的问题,并提供优化建议。这些技术的不断发展和应用,为自动驾驶测试场景的高效生成和质量评估提供了重要的技术支持,未来有望进一步提高自动驾驶测试的效率和可靠性。


\section{研究内容}
\subsection{面向场景描述的领域知识图谱构建}
在自动驾驶测试场景生成中,构建面向场景描述的领域知识图谱是实现高效、准确场景生成的基础。领域知识图谱通过整合自动驾驶领域的专业知识,包括交通规则、道路类型、车辆行为模式、传感器特性等,为自然语言描述的解析和形式化代码的生成提供丰富的上下文信息和语义支持。
领域知识图谱的构建涉及多个关键步骤。首先,需要对自动驾驶领域的知识进行系统梳理和分类,明确知识的层次结构和关联关系。这包括对交通场景中的实体(如车辆、行人、道路、交通标志等)及其属性(如位置、速度、类型等)的定义,以及这些实体之间的关系(如车辆与道路的交互、车辆与行人的避让等)。通过构建知识图谱,可以将这些复杂的知识结构化地表示出来,便于后续的查询和推理。
在知识图谱的构建过程中,还需要考虑知识的动态更新和扩展。自动驾驶技术不断发展,新的交通规则、车辆类型和传感器技术等不断涌现,因此知识图谱需要具备良好的可扩展性和可更新性。通过持续的知识更新,可以确保知识图谱始终保持最新状态,从而为场景生成提供准确的知识支持。
此外,领域知识图谱的构建还需要考虑知识的表达和存储方式。知识图谱通常以图的形式存储,其中节点表示实体,边表示实体之间的关系。这种结构化的存储方式不仅便于知识的查询和推理,还能够支持复杂的知识融合和关联分析。通过知识图谱的构建,可以实现对自动驾驶场景描述的深度理解和语义解析,为后续的代码生成和场景合成提供坚实的基础。

\subsection{基于LLM的语义约束代码生成方法}
基于大型语言模型(LLM)的语义约束代码生成方法是实现自动驾驶测试场景高效生成的关键技术之一。LLM具有强大的语言理解和生成能力,能够根据自然语言描述生成高质量的形式化代码。然而,为了确保生成代码的准确性和可靠性,需要在代码生成过程中引入语义约束机制。
语义约束代码生成方法的核心在于将自然语言描述的语义信息转化为代码生成的约束条件。这些约束条件可以包括交通规则、道路类型、车辆行为模式等,确保生成的代码不仅符合语法规范,还满足实际场景的语义要求。通过语义约束机制,可以有效避免生成代码中的逻辑错误和不符合实际场景的情况,提高代码的质量和可用性。
在实现语义约束代码生成时,需要充分利用LLM的语言理解和生成能力。LLM可以通过对自然语言描述的深度理解,提取出关键的语义信息,并将其转化为代码生成的约束条件。同时,还需要开发相应的算法和工具,将这些约束条件嵌入到代码生成过程中,确保生成的代码能够准确地反映自然语言描述的语义内容。
此外,语义约束代码生成方法还需要考虑代码的可读性和可维护性。生成的代码不仅需要符合语法和语义规范,还需要具有良好的结构和注释,便于后续的修改和扩展。通过基于LLM的语义约束代码生成方法,可以实现从自然语言描述到形式化代码的高效转换,为自动驾驶测试场景的生成提供强大的技术支持。

\subsection{场景物理合理性的多模态验证机制}
场景物理合理性的验证是自动驾驶测试场景生成中的一个重要环节,它直接影响到测试场景的真实性和可靠性。为了确保生成场景的物理合理性,需要建立一种多模态验证机制,通过多种方式对场景进行综合验证。
多模态验证机制的核心在于结合多种验证手段,从不同角度对场景的物理合理性进行评估。这些验证手段可以包括基于物理规则的验证、基于仿真数据的验证以及基于专家知识的验证等。通过多种验证手段的结合,可以全面评估场景的物理合理性,确保生成的场景符合实际交通环境的物理规律。
基于物理规则的验证是多模态验证机制的重要组成部分。通过定义和应用物理规则,如牛顿运动定律、能量守恒定律等,可以对场景中的物体运动和交互进行验证。例如,可以验证车辆的加速度是否符合物理规律,车辆与行人之间的碰撞是否符合能量守恒等。这种基于物理规则的验证方法能够从理论上确保场景的物理合理性。
基于仿真数据的验证则是通过在仿真环境中运行生成的场景,收集仿真数据,并对数据进行分析和评估。通过对比仿真结果与实际交通数据,可以评估场景的物理合理性。例如,可以对比仿真中车辆的行驶速度、加速度等数据与实际交通数据的一致性。这种基于仿真数据的验证方法能够从实际运行的角度评估场景的物理合理性。
基于专家知识的验证则依赖于领域专家的经验和知识。通过邀请专家对生成的场景进行评估和审核,可以发现场景中可能存在的问题和不合理之处。专家可以根据自己的经验和知识,对场景中的车辆行为、交通规则遵守情况等进行评估,提出改进意见。这种基于专家知识的验证方法能够从实践经验和知识的角度评估场景的物理合理性。
通过建立场景物理合理性的多模态验证机制,可以全面评估生成场景的物理合理性,确保生成的场景符合实际交通环境的物理规律。这种多模态验证机制的建立,为自动驾驶测试场景的高质量生成提供了重要的保障,有助于提高自动驾驶测试的可靠性和有效性。
