% 第一章 绪论
\chapter{绪论}
\section{研究背景}
\subsection{智能驾驶测试对海量长尾场景的需求矛盾}
随着智能驾驶技术快速发展起来,它的测试需求也变得日益复杂且多样化,智能驾驶系统需要在各类复杂多变交通场景下\cite{abeysirigoonawardena2019adversarial},展现出可靠性能来确保驾驶安全与舒适,然而现实交通环境里存在海量长尾场景,这些场景虽出现频率低但发生时会对系统构成严峻挑战。

长尾场景的复杂性主要体现在如下几个方面,首先交通环境动态性极强,车辆\ref{fig:vehicle_sample}、行人以及非机动车等交通参与者行为模式千变万化,比如在城市道路当中行人可能突然横穿马路,非机动车可能随意变道或者逆行,\cite{bagschik2018ontology}这些行为都可能引发潜在的碰撞风险,智能驾驶系统必须能够准确感知并预测这些行为,然后做出合理决策。其次道路条件的多样性增加了测试场景的复杂性\cite{biggio2013evasion},从城市快速路到乡村小道,再从高速公路到山区道路,不同道路几何形状、路面状况、交通标志和信号灯等都不一样,智能驾驶系统需要在这些不同道路条件下都能正常运行\cite{brown2020language}。此外天气条件和光照条件的变化,也会对智能驾驶系统的感知和决策造成影响,雨、雪、雾等恶劣天气会降低传感器的性能,不同光照条件像强光、弱光、逆光等会影响视觉系统识别效果。

为了保证智能驾驶系统具备安全性和可靠性,测试过程得覆盖尽可能多的长尾场景,然而这面临着巨大挑战,一方面长尾场景数量庞大几乎无法穷尽,要完全覆盖所有可能场景几乎是不可能任务,另一方面这些场景出现频率较低难以在实际道路测试中频繁遇到,所以传统测试方法往往难以满足智能驾驶系统对海量长尾场景的测试需求,使得测试的充分性和有效性受到限制\cite{cai2020summit}。
\begin{figure}[h]
	\centering
	\includegraphics[width=0.7\textwidth]{"images/车辆样式图.pdf"}
	\caption{carla场景样图}
	\label{fig:vehicle_sample}
\end{figure}
\subsection{传统场景构建方法的人力成本与效率瓶颈}
在智能驾驶测试这个领域当中,传统的场景构建方法主要靠人工设计和开发\cite{cui2023chatlaw},这些方法虽说在一定程度上能够满足测试需求,不过也存在着比较显著的局限性,人工设计场景需要投入大量专业人员,这些人员要具备交通工程、计算机科学和智能驾驶技术等多学科深厚知识,能准确理解并模拟各种复杂交通场景,这样的专业人才相对比较稀缺,培养成本也十分高昂,随着测试需求不断增加,对专业人员的需求也持续增长,这进一步加剧了人力成本方面的压力。

其次人工设计场景的效率相对比较低,设计一个复杂交通场景需经多步骤,像需求分析、场景建模、代码编写和调试等,每一个步骤都要耗费大量时间精力,例如在场景建模阶段要精确去定义道路几何形状、交通参与者初始位置和行为模式等,在代码编写阶段需把这些模型转化成可执行代码,这不仅要求具备专业编程技能,还需反复调试来确保代码正确性和稳定性,所以人工设计场景速度远不能满足智能驾驶系统快速迭代和测试的需求,此外人工设计场景的准确性和一致性难保证,因为不同设计人员理解和经验存在差异,可能致使设计出的场景出现一定的差异,并且在复杂场景中人工设计易遗漏关键细节,进而影响测试结果准确性和可靠性。

\subsection{大语言模型在代码生成领域的突破性进展}
近年来大语言模型在自然语言处理领域取得巨大突破且逐渐拓展到代码生成等领域,大语言模型通过在海量文本数据上预训练学习到语言语法语义和逻辑结构能生成自然流畅且符合逻辑的文本内容,这种能力为解决智能驾驶测试场景构建难题提供新的思路。在代码生成领域大语言模型已经展现出强大能力\cite{chowdhery2022palm},通过对代码数据学习大语言模型能理解代码结构和逻辑生成符合语法规范的代码片段,例如一些基于大语言模型的代码生成工具可根据用户输入自然语言描述自动生成相应代码实现,这些工具在软件开发领域得到广泛应用显著提高代码开发效率和质量。大语言模型在代码生成中的优势主要体现在以下几个方面:

首先它能够快速生成代码大大缩短开发周期,传统人工编写代码需经过需求分析设计编码和调试等多个阶段且每个阶段都耗费大量时间,而大语言模型可根据输入描述直接生成代码减少中间环节提高开发效率\cite{feng2021intelligent},其次大语言模型生成的代码质量较高,它能学习到代码最佳实践和规范生成的代码不仅符合语法规范还具有良好可读性和可维护性,

此外大语言模型还能根据不同需求生成多样化代码实现为开发者提供更多选择。把大语言模型运用到智能驾驶测试场景构建当中\cite{kong2020physgan},能够充分发挥它在代码生成领域的优势来解决传统方法面临的困境,借助将自然语言描述的测试场景需求转变为代码实现,大语言模型能够快速生成高保真的测试场景进而提高场景构建的效率和质量,结合智能驾驶领域的专业知识还可以进一步优化大语言模型的性能让它更好适应智能驾驶测试场景构建的需求。




\section{国内外研究现状}
\subsection{基于大语言模型能够根据人的自然语言指令生成Scenic场景代码}
随着大语言模型像GPT -3、T5等的广泛应用起来,基于自然语言描述生成仿真场景代码成为自动驾驶仿真研究重要方向,自然语言描述能够以简单且直观的方式传递场景信息\cite{klischat2020scenario},传统手工编写场景代码的方式存在效率低且灵活性差的问题,所以如何利用自然语言生成交通仿真场景脚本成为该领域核心问题。

大语言模型像GPT - 3和T5这类\cite{Xu2023DriveGPT4},已经成功应用于自然语言到场景代码的转换工作,经过训练之后,这些模型能够理解复杂的自然语言输入内容,依据相关描述,它们可以生成结构化的Scenic脚本文件\cite{scenario_runner_contributors2019carla},这些模型通过解析用户给出的自然语言指令,来生成符合自动驾驶仿真要求的场景代码,比如,提出一种基于GPT的框架,能根据自然语言描述生成含车辆行人交通灯等元素的完整交通场景配置。

在挑战与进展方面大语言模型生成场景代码有一定进展但依旧面临挑战,首先处理自然语言里的歧义和模糊性属于一个难题,复杂场景描述下生成的场景可能无法完全符合预期,其次现有模型生成能力在处理复杂或特殊交通场景时存在局限性,所以增强语言模型语义理解和场景生成准确性仍是活跃研究领域。

\subsection{将场景代码合成合理的智能驾驶场景}
把生成的Scenic场景代码转化成合理的智能驾驶仿真场景\cite{wu2025complexscenes},这是实现自动驾驶测试与验证的关键步骤,生成的场景不仅要符合交通方面的规则,还得能够模拟真实的驾驶环境来进行有效测试。
场景代码转化到仿真环境方面\cite{gu2025reinforcementlearning},当前的研究重点主要集中在怎样把大语言模型所生成的Scenic脚本转化成仿真平台能够执行的场景配置内容。例如,开发出基于Scenic的编译器用于把自然语言生成场景描述转化为CARLA仿真平台所需配置文件,通过这种方式生成的场景能够在高保真的仿真环境里进行验证

智能场景合成跟动态行为建模这方面,智能驾驶场景不只是包含像道路、交通标志这类静态元素,还涵盖如车辆行驶、变道、停车等动态行为,研究者们提出了多种动态行为建模的方法,让生成的交通场景能更真实地反映驾驶行为与交通流情况\cite{wu2025taxi},例如,提出一种基于行为模型的动态场景合成方法来通过模拟交通参与者行为生成交互式智能驾驶环境。

复杂交通场景面临的挑战是,虽说场景合成方法在持续发展,但生成复杂真实且符合交通规则的场景依旧是技术难题,现有的场景生成方法在应对复杂交通场景时,可能会出现车辆行为不自然以及交通规则不严谨等状况,所以提高场景合成的灵活性和复杂性仍是研究重点。

\subsection{对生成的交通场景进行展示和效果的量化衡量}
在生成交通场景之后对其进行展示以及量化评估是保证场景质量和仿真结果准确性的关键步骤\cite{peng2025data},通过对场景开展可视化以及量化评估研究人员能够验证场景的有效性并为后续自动驾驶算法优化提供数据支撑。

可视化展示就是把生成的交通场景借助可视化工具进行展示,从而方便进行人工验证和审阅。我这边提出一种基于关键帧截取的可视化方法\cite{fang2025yolo},通过在仿真过程当中截取关键帧图像,以此帮助研究人员快速了解仿真过程里的交通场景,结合深度学习相关技术,自动驾驶系统还可以从这些图像当中提取重要信息,进而对场景开展进一步分析与优化。

提出量化评估指标的情况是,为了能系统地对生成场景的质量进行评估,许多研究都提出了量化评估框架。例如,提出一种涵盖场景语义保真度等多方面的多维度评估方法\cite{zhu2024tsgan},这些评估指标既能反映生成场景的真实性又能助研究人员评估仿真结果有效性。

安全性与性能评估:自动驾驶系统在仿真环境中的表现也需要量化评估。提出一种基于自动驾驶系统安全性的评估框架,通过对车辆在生成场景里碰撞率与通过率等指标进行分析,以此评估自动驾驶系统于不同场景之下的安全性和稳定性,借助这样的量化评估手段,研究人员可识别出潜在的风险和问题并进一步优化自动驾驶算法。

\section{研究内容}
\subsection{面向场景描述的领域知识图谱构建}
在自动驾驶测试场景生成这个事情当中,构建面向场景描述的领域知识图谱是实现高效且准确场景生成的基础,领域知识图谱通过整合自动驾驶领域的专业知识\cite{Wang2021advSim},像交通规则、道路类型、车辆行为模式以及传感器特性等内容,为自然语言描述的解析和形式化代码的生成提供丰富上下文信息和语义支持。

领域知识图谱构建包含多个关键步骤\cite{Xu2022Safebench},首先要对自动驾驶领域知识做系统梳理与分类,明确知识的层次结构和关联关系,这涵盖对交通场景里实体(像车辆、行人、道路、交通标志等)及其属性(例如位置、速度、类型等)的定义,还有这些实体之间关系(比如车辆与道路的交互、车辆与行人的避让等),构建知识图谱可将这些复杂知识结构化表示出来,方便后续进行查询和推理。

在知识图谱的构建过程当中需要考虑知识动态更新与扩展\cite{Yang2020SurfelGAN},自动驾驶技术处于不断发展中新交通规则车辆类型等不断涌现,所以知识图谱得具备良好可扩展性以及可更新性,通过持续开展知识更新能确保知识图谱始终保持最新状态,进而为场景生成提供准确无误的知识方面支持。

除此之外领域知识图谱的构建还得考虑知识表达与存储方式,知识图谱一般是以图的形式进行存储的\cite{Yao2022React},其中节点所表示的是实体而边表示实体间的关系,这种结构化的存储方式不仅方便知识的查询和推理,还能够支持复杂的知识融合与关联分析,借助知识图谱的构建能够实现对自动驾驶场景描述的深度理解和语义解析,为后续的代码生成以及场景合成提供坚实可靠的基础。

\subsection{基于LLM的语义约束代码生成方法}
基于大型语言模型(LLM)的语义约束代码生成方法是达成自动驾驶测试场景高效生成的一项关键技术,大型语言模型(LLM)具备强大的语言理解和生成能力可依据自然语言描述生成高质量形式化代码,不过为保证生成代码的准确性与可靠性需在代码生成过程中引入语义约束机制\cite{Zhang2023CAT}。

语义约束代码生成方法的关键是把自然语言描述的语义信息转成代码生成约束条件,这些约束条件涵盖交通规则、道路类型以及车辆行为模式等内容,目的是保证生成的代码既符合语法规范又能满足实际场景语义要求\cite{Zhang2022AdversarialRobustness},借助语义约束机制可有效避免生成代码里的逻辑错误与不符合实际场景的状况,以此提高代码的质量和可用性。

实现语义约束代码生成的时候要充分利用LLM语言理解与生成能力\\ \cite{Zheng2023JudgingLLM},LLM可通过对自然语言描述深度理解提取关键语义信息,再将这些关键语义信息转化为代码生成约束条件,同时还得开发对应算法和工具把约束条件嵌入代码生成过程,以此确保生成代码能准确反映自然语言描述语义内容,此外语义约束代码生成方法也需考虑代码可读性和可维护性,生成代码不仅要符合语法和语义规范,还得具备良好结构和注释以方便后续修改和扩展,借助基于LLM的语义约束代码生成方法能实现从自然语言描述到形式化代码高效转换\cite{zhong2023language},可为自动驾驶测试场景生成提供强大技术支持。

\subsection{场景物理合理性的多模态验证机制}
场景物理合理性验证在自动驾驶测试场景生成里是重要环节\cite{zhao2025key},它会直接影响测试场景的真实性和可靠性,为确保生成场景具备物理合理性,需要建立一种多模态验证机制来通过多种方式综合验证场景,多模态验证机制核心在于结合多种验证手段,从不同角度对场景的物理合理性展开评估,这些验证手段包括基于物理规则的验证、基于仿真数据的验证以及基于专家知识的验证等,通过多种验证手段结合能够全面评估场景物理合理性,确保生成场景符合实际交通环境物理规律\cite{Yao2022React},基于物理规则的验证是多模态验证机制的重要组成部分,通过定义和应用牛顿运动定律、能量守恒定律等物理规则,可对场景中物体运动和交互进行验证,比如验证车辆加速度是否符合物理规律、车辆与行人之间碰撞是否符合能量守恒等。

\section{本文研究结构}

为了达成基于自然语言输入自动生成高保真三维交通场景并做量化评估这一目标,本文构建起一个把自然语言处理、交通场景建模\cite{du2025scene}、三维仿真以及量化评估整合在一起的综合研究体系,本文对当前国内外在自然语言驱动的场景生成、智能仿真系统集成和自动驾驶场景评估这些方面的研究进展进行分析,从而明确研究问题以及技术挑战。

第一章绪论这部分内容阐述了研究背景以及问题提出逻辑起点,首先点明当前智能驾驶技术面临关键挑战是对低频高风险“长尾场景”识别与处理能力不足,接着分析长尾场景典型特征包括交通参与者行为不可预测性、道路布局复杂性以及自然条件变化对系统的干扰等情况,本章还进一步指出现有测试方法难以全面覆盖这些多样化场景从而导致系统评估存在盲区,所以为实现更高保真的测试需求提出通过生成逼真三维测试场景来填补这一空白,本章节为后续系统设计与实现奠定问题基础和研究动因并明确构建智能驾驶场景生成系统的必要性与紧迫性。

第二章相关工作:这一章节围绕自然语言生成智能驾驶场景相关技术进行论述,系统介绍预训练语言模型在场景生成里的核心机制,尤其是基于Sentence - T5模型对自然语言场景描述做语义向量编码,再通过向量匹配去检索结构化模板,实现从语言理解到结构生成的语义桥接过程,同时深入探讨自然语言到结构化代码转换的建模逻辑,强调语义建模、语法约束建模和结构对齐的协同作用,在交通场景语义建模方面,本文提出把自然语言转化成结构化语义表示,提取参与实体、行为属性、空间关系和时序逻辑,以此生成具备真实执行逻辑的交通场景脚本,Scenic语言作为场景描述核心工具,凭借简洁语法和概率建模能力,为多样化、高保真的场景生成提供支持,最后章节简要介绍本系统与CARLA智能驾驶仿真平台的集成方式,借助Scenic语言生成的脚本自动构建可执行三维交通场景,实现自然语言到可交互场景的完整闭环。

第三章系统需求分析与总体架构设计:主要是系统地对基于自然语言的三维智能驾驶场景生成系统做分析和设计,首先明确系统需要满足的核心功能需求,涵盖高保真自动驾驶测试场景构建需求、自然语言输入解析、语义理解与结构映射、可执行脚本生成以及仿真平台验证等关键能力,旨在以自动化方式提高智能驾驶测试效率与多样性,接着详细说明系统总体架构,采用模块化分层设计将系统划分为自然语言输入模块、结构生成模块、CARLA仿真模块和结果评估分析模块这四大部分,其中自然语言输入模块借助图形界面与语义模型实现对用户场景描述的高质量编码与检索,结构生成模块把检索结果和预设模板相融合输出符合Scenic语法的结构化脚本,仿真模块基于CARLA平台加载脚本并自动调度执行以完成交通参与体的行为仿真与场景复现,结果评估模块对仿真过程进行图像、视频和行为数据的采集与分析并依据多维指标输出仿真效果与驾驶行为质量评估,如此便构建出一个完整的自然语言驱动智能驾驶场景生成与验证系统架构,为后续模块实现提供明确蓝图和技术支撑。

第四章系统设计与实现部分介绍系统开发环境及核心功能实现情况,硬件方面采用具备高性能的本地计算机包含Intel i9、RTX 3060以及32GB内存,软件环境是基于Windows 11和Python 3.8且集成自然语言处理与自动驾驶仿真相关依赖库,系统实现划分为三大核心模块,自然语言理解模块借助句向量检索和大模型来生成Scenic脚本,场景合成与仿真模块负责将脚本转化成CARLA里的动态三维场景,评估与展示模块对仿真效果进行可视化呈现并在语义保真度、多样性、驾驶性能等方面开展定量分析为系统验证和优化提供相应依据。

第五章系统案例测试分析:这章通过还原十组典型自然语言输入场景来展示系统在语义解析、结构化生成以及三维仿真中的完整流程和实际效果,每组案例都从文本输入开始,系统会自动完成语义理解、Scenic 脚本构建以及 Carla 场景生成,直观呈现不同语义指令下生成结果的可视化表现,在此基础上结合语义保真度、场景多样性和系统响应效率这三大维度开展系统性能评估,实验结果显示系统在保真性和语义一致性方面最终得分能保持在 0.94 - 0.95 之间,多样性与准确性指标都达到较高水准,和规则方法以及 GPT - 4.0 方法相比展现出更好的综合生成质量和语义还原能力。

第六章总结与展望:这篇文章以自然语言处理技术为基础,设计并且实现出一种自动驾驶仿真场景生成方法,还结合CARLA平台达成动态场景的语义驱动生成和感知系统的集成,系统利用深度学习模型对自然语言描述做语义解析,能够自动构建出符合语义要求的复杂交通场景,同时借助集成的目标检测与行为预测模块提高自动驾驶系统的智能决策能力,实验结果证实了该方法具备有效性和实用性,为提升自动驾驶场景多样性和感知智能化给予了有力支持。