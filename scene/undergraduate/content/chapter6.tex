%!TEX root = ../../csuthesis_main.tex
\chapter{总结与展望}

\section{工作总结}

随着自动驾驶技术快速地不断发展,在复杂交通环境中借助高效场景生成与感知系统来提升自动驾驶系统智能性和安全性成研究关键,本文设计并实现基于自然语言处理的自动驾驶仿真场景生成方法,结合Carla仿真平台探讨利用自然语言输入生成动态仿真场景并结合感知系统提高自动驾驶决策能力,本研究围绕“自然语言驱动的自动驾驶场景生成与感知”问题,采用基于预训练大模型的检索与生成方法利用深度学习技术提升场景生成灵活性和多样性,通过对输入自然语言描述进行语义分析,系统可自动生成符合语义要求场景并在Carla平台上进行仿真验证,此外本文还设计集成感知系统,通过深度学习与图像处理技术实现交通目标检测、跟踪与行为预测为后续决策与控制提供支持。

实验结果显示基于自然语言生成的仿真场景能准确反映输入描述语义,还可通过感知系统实时跟踪和预测交通目标行为意图,系统能够在复杂交通环境里稳定运行,这验证了自然语言描述和自动驾驶场景生成的可行性与有效性。本文的贡献是提出了一种创新性的自动驾驶场景生成与感知方法,通过自然语言生成仿真场景且结合感知与决策支持,提升了自动驾驶系统的适应性与智能化水平,研究中采用的Carla仿真平台为系统验证与优化提供了丰富测试数据,未来研究成果有望为智能驾驶系统多样化场景生成和感知能力提升提供更多技术支持。

\section{研究不足}
虽然本文在自然语言生成场景和感知系统研究里取得了一定进展,不过在某些方面还是存在不足与局限,当前的场景生成方法虽能生成基本交通场景,但处理更复杂动态环境如极端天气和特殊道路条件时显得力不从心,现有的自然语言理解模型对复杂描述理解能力和场景生成精确度有待提升,面对不常见交通情境生成场景缺乏丰富性和真实感,尽管本文的感知系统在多数情况下能准确跟踪和预测目标行为,但在高密度交通或目标交错情况下会出现目标跟踪失误影响行为预测准确性,在多目标竞态和复杂背景下现有的目标跟踪算法仍可能受影响导致系统长期跟踪出现问题。


虽说本研究把深度学习和物理建模结合起来做行为意图分析,可现有的方法依靠物理模型简单规则,或许难以捕捉更复杂多变的驾驶行为,特别是多个目标之间存在复杂交互的时候,系统可能没办法准确预测其行为,最后,系统在处理高分辨率图像和复杂场景时,其实时性和计算性能还存在一定瓶颈,虽说现有系统能满足基本实时性要求,但在高负载条件下,响应时间和处理速度可能会受影响,这在实际应用当中也许会带来挑战。

\section{后续优化方向}

在未来的研究当中系统的优化会集中在下面这几个方面 首先提升场景生成的多样性和复杂度会是未来研究的重点内容 当前系统生成的交通场景主要是以基本场景作为主体 面对复杂交通情况的时候依然存在着一定的不足 未来的研究将会采用更先进自然语言处理和场景生成模型 结合强化学习等相关方法提高系统在极端天气事故等复杂情况中的应对能力,进而生成更加多样化且真实的场景 其次目标跟踪与行为意图预测的精度会得到进一步的提升 虽然现有的跟踪算法在大多数情况之下能够保持目标稳定跟踪 但是在高密度或者复杂背景的状况下仍然存在一定精度问题。

未来的工作会进一步探索更加鲁棒的目标跟踪算法 结合深度学习和多传感器融合技术提高系统的目标识别和跟踪能力 此外意图预测会结合更多像交通规则和社会行为这样的情境因素 以此来提升对复杂驾驶行为的预测精度。


再者系统实时性与计算性能会持续不断优化,随着场景复杂度逐渐增加系统实时性和处理能力成关键问题,未来研究将重点聚焦高效图像处理和计算方法,借助硬件加速以及算法优化提高系统计算效率,确保在高负载和复杂场景下依旧能保持流畅运行。最后未来研究将结合真实道路测试与仿真验证,把研究成果转化成为实际可应用的内容,通过在真实环境中开展测试进一步评估系统,确保系统在多样化且复杂交通环境中的稳定性。通过以上这些方面的优化未来系统能更好应对,动态复杂交通场景提高自动驾驶系统智能感知,为实现更高安全性和智能化水平自动驾驶技术奠定基础。
