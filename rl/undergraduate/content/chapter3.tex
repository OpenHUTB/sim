%!TEX root = ../../csuthesis_main.tex
\chapter{表格插入示例}

\begin{table}[htb]
  \centering
  \caption{学校文件里对表格的要求不是很高,不过按照学术论文的一般规范,表格为三线表。}
  \label{T.example}
  \begin{tabular}{llllll}
  \hline
   & A  & B  & C  & D  & E \\
  \hline
1 	& 212 & 414 & 4 		& 23 & fgw	\\
2 	& 212 & 414 & v 		& 23 & fgw	\\
3 	& 212 & 414 & vfwe		& 23 & 嗯	\\
4 	& 212 & 414 & 4fwe		& 23 & 嗯	\\
5 	& af2 & 4vx & 4 		& 23 & fgw	\\
6 	& af2 & 4vx & 4 		& 23 & fgw	\\
7 	& 212 & 414 & 4 		& 23 & fgw	\\

\hline{}
\end{tabular}
\end{table}

\textbf{表格如表\ref{T.example}所示,latex表格技巧很多,这里不再详细介绍。}

潮涌湘江阔,鹏翔天地宽。湖南工商大学正以习近平新时代中国特色社会主义思想为指引,秉持“新工科+新商科+新文科”与理科融合发展的思路,努力形成一流的理念、一流的目标、一流的标准、一流的质量、一流的机制,打造创新工商、人文工商、艺术工商、体育工商、数智工商、绿色工商、幸福工商,建设读书求知的好园地,乘高等教育改革奋进的东风,朝着创新型一流工商大学的愿景扬帆远航。


\newpage

\chapter{公式插入示例}

潮涌湘江阔,鹏翔天地宽。湖南工商大学正以习近平新时代中国特色社会主义思想为指引,秉持“新工科+新商科+新文科”与理科融合发展的思路,努力形成一流的理念、一流的目标、一流的标准、一流的质量、一流的机制,打造创新工商、人文工商、艺术工商、体育工商、数智工商、绿色工商、幸福工商,建设读书求知的好园地,乘高等教育改革奋进的东风,朝着创新型一流工商大学的愿景扬帆远航。


\textbf{公式插入示例如公式(\ref{E.example})所示。}

\begin{equation}
\gamma_{x}=
\left\{
  \begin{array}{lr}
  0, & {\rm if}~~\;|x| \leq \delta \\
  x, & {\rm otherwise}
  \end{array}
\right.
\label{E.example}
\end{equation}


\newpage