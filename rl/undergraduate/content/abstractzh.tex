%!TEX root = ../csuthesis_main.tex
% 设置中文摘要
\keywordscn{数字孪生\quad 自动驾驶\quad 强化学习\quad 仿真系统\quad 决策优化}
%\categorycn{TP391}
\begin{abstractzh}
随着当今世界科学技术的行驶发展,人们对于自动驾驶的需求也越来越高,因此,现如今的汽车已无法满足人们对现如今复杂的驾驶环境的需要。而数字孪生技术作为一种新型的模拟仿真技术得到了人们的关注。数字孪生技术通过构建与现实环境相对于的虚拟环境来为人们提供一个高效、安全的自动驾驶测试平台。数字孪生通过构建一个虚拟环境来为人们提供实时反馈的环境信息。数字孪生通过提供的实时信息来为人们提供对自动驾驶模型训练的支持。而强化学习作为一种能够在不断变化的环境中进行自我学习的算法。其能够为人们对自动驾驶系统的完成提供帮助。通过将数字孪生技术和强化学习算法相结合,不仅能够帮助我们更好的进行自动驾驶系统的开发与验证,还能够提高自动驾驶系统的适应能力与安全性。

本篇文章通过讨论基于数字孪生的自动驾驶强化学习仿真系统。来展示将数字孪生技术与强化学习算法相结合之后在自动驾驶方面所展现的巨大潜力与力量。通过本篇文章,我们可以清晰的看到数字孪生技术能够为人们提供一个与现实环境相对应的虚拟环境,来帮助人们更加安全和准确的训练自动驾驶模型。而强化学习算法则向我们展示了其是如何帮助自动驾驶系统更加快速和准确的训练自动驾驶模型。

本篇文章通过先介绍什么是数字孪生和强化学习算法,再展示了数字孪生和强化学习的基本算法和原理。并展示了数字孪生和强化学习算法的具体公式。最后展示了我本人是怎样在实际的环境中应用数字孪生技术和强化学习算法来对自动驾驶模型进行训练的。我本人通过将数字孪生技术和强化学习算法相结合,设计了利用不同的典型强化学习算法如PPO、DQN来训练一个自动驾驶模型。该自动驾驶模型能够在Carla仿真环境中实现自动驾驶。实现我对该模型的具体需求。


\end{abstractzh}