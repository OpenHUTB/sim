%!TEX root = ../csuthesis_main.tex
% \begin{appendixs} % 无章节编号
\chapter{附录代码}
\section{PPO算法Python描述}
\begin{lstlisting}
# 测试模式
if train is False: 
# 创建PPO智能体实例(使用初始动作标准差参数)
agent = PPOAgent(town, action_std_init)
# 加载预训练模型参数
agent.load()
# 冻结旧策略网络参数(测试时不需要梯度更新)
for params in agent.old_policy.actor.parameters():
params.requires_grad = False

# 训练模式
else:  
# 恢复训练逻辑
if checkpoint_load:
print("resume:checkpoint_load")
# 获取检查点目录下文件数量(减1得到最新编号)
chkt_file_nums = len(next(os.walk(f'checkpoints/PPO/{town}'))[2]) - 1
# 构建检查点文件路径
chkpt_file = f'checkpoints/PPO/{town}/checkpoint_ppo_' + str(chkt_file_nums) + '.pickle'

# 加载检查点数据
with open(chkpt_file, 'rb') as f:
data = pickle.load(f)
episode = data['episode']          # 恢复训练回合数
timestep = data['timestep']        # 恢复时间步计数
cumulative_score = data['cumulative_score']  # 恢复累计得分
action_std_init = data['action_std_init']    # 恢复当前动作标准差

# 使用恢复的参数创建新智能体
agent = PPOAgent(town, action_std_init)
agent.load()  # 加载模型参数
else:
# 全新训练初始化
agent = PPOAgent(town, action_std_init)

# 训练主循环
if train:
print("-----------------Carla_PPO Train-------------------")
# 总时间步控制循环
while timestep < total_timesteps:
# 环境重置并获取初始观察值
observation = env.reset()
# 对原始观察进行编码处理
observation = encode.process(observation)

current_ep_reward = 0  # 当前回合累计奖励
t1 = datetime.now()    # 记录回合开始时间

# 单回合循环(最大长度由参数控制)
for t in range(args.episode_length):
# 通过策略网络选择动作(训练模式启用探索)
action = agent.get_action(observation, train=True)

# 执行动作,获取环境反馈
observation, reward, done, info = env.step(action)
if observation is None:  # 异常处理
break
# 处理新观察值
observation = encode.process(observation)
# 存储奖励和终止标志到经验池
agent.memory.rewards.append(reward)
agent.memory.dones.append(done)

timestep += 1  # 全局时间步计数
current_ep_reward += reward  # 回合奖励累计

# 定期衰减动作标准差(降低探索率)
if timestep % action_std_decay_freq == 0:
action_std_init = agent.decay_action_std(action_std_decay_rate, min_action_std)

# 训练结束时保存最终模型
if timestep == total_timesteps - 1:
agent.save()

# 回合提前终止处理
if done:
episode += 1  # 全局回合计数
# 计算回合持续时间
t2 = datetime.now()
t3 = t2 - t1
episodic_length.append(abs(t3.total_seconds()))
break
\end{lstlisting}

\section{DQN算法Python描述}
\begin{lstlisting}
	
# ============================== 智能体初始化部分 ==============================
if train is False:  # 测试模式
# 创建DQN智能体实例(指定城镇和动作空间维度)
agent = DQNAgent(town, n_actions)
# 加载预训练模型参数
agent.load_model()
# 冻结评估网络参数(测试不需要梯度更新)
for params in agent.q_network_eval.parameters():
params.requires_grad = False
# 冻结目标网络参数
for params in agent.q_network_target.parameters():
params.requires_grad = False
else:  # 训练模式
if checkpoint_load:  # 从检查点恢复
agent = DQNAgent(town, n_actions)
agent.load_model()  # 加载模型参数
if exp_name == 'ddqn':  # 如果是双DQN算法
# 加载训练状态数据
with open(f'checkpoints/DDQN/{town}/checkpoint_ddqn.pickle', 'rb') as f:
data = pickle.load(f)
epoch = data['epoch']  # 恢复训练轮次
cumulative_score = data['cumulative_score']  # 恢复累计得分
agent.epsilon = data['epsilon']  # 恢复当前探索率
else:  # 全新训练初始化
agent = DQNAgent(town, n_actions)

# =========================== 经验回放预填充(仅DDQN) ===========================
if exp_name == 'ddqn' and checkpoint_load:
# 用随机策略填充经验回放缓冲区
while agent.replay_buffer.counter < agent.replay_buffer.buffer_size:
observation = env.reset()
observation = encode.process(observation)  # 编码初始状态
done = False
while not done:
# 生成随机动作(0到n_actions-1之间的整数)
action = random.randint(0, n_actions - 1)
# 执行随机动作
new_observation, reward, done, _ = env.step(action)
new_observation = encode.process(new_observation)
# 保存转移样本到经验池
agent.save_transition(observation, action, reward, new_observation, int(done))
observation = new_observation  # 更新当前状态

# ================================ 训练主循环 ================================
if args.train:
print("-----------------Carla_DQN Train-------------------")
# 从恢复的epoch开始继续训练
for step in range(epoch + 1, EPISODES + 1):
done = False  # 回合终止标志
observation = env.reset()  # 重置环境
observation = encode.process(observation)  # 状态编码
current_ep_reward = 0  # 当前回合奖励累计

t1 = datetime.now()  # 记录回合开始时间

while not done:  # 单回合循环
# 通过ε-greedy策略选择动作(训练模式启用探索)
action = agent.get_action(args.train, observation)
# 执行动作并获取环境反馈
new_observation, reward, done, info = env.step(action)

if new_observation is None:  # 异常处理
break
new_observation = encode.process(new_observation)  # 编码新状态
current_ep_reward += reward  # 累计即时奖励

# 保存经验元组(当前状态,动作,奖励,新状态,终止标志)
agent.save_transition(observation, action, reward, new_observation, int(done))

# 当经验池达到预热阈值后开始学习
if agent.get_len_buffer() > WARMING_UP:
agent.learn()  # 执行Q网络更新(每一步都学习)

observation = new_observation  # 状态转移
timestep += 1  # 全局时间步计数
if done:
episode += 1  # 全局回合计数

# ============================ 回合结束处理 ============================
t2 = datetime.now()
t3 = t2 - t1
episodic_length.append(abs(t3.total_seconds()))  # 记录回合时长

# 收集性能指标
deviation_from_center += info[1]  # 累计道路中心偏离量
distance_covered += info[0]       # 累计行驶距离

scores.append(current_ep_reward)  # 记录回合奖励

# 计算移动平均奖励
if checkpoint_load:
cumulative_score = ((cumulative_score * (episode - 1)) + current_ep_reward) / (episode)
else:
cumulative_score = np.mean(scores)

# 输出训练信息
print('Starting Episode: ', episode, 
', Epsilon Now:  {:.3f}'.format(agent.epsilon),  # 当前探索率
'Reward:  {:.2f}'.format(current_ep_reward), 
', Average Reward:  {:.2f}'.format(cumulative_score))

# 保存当前最佳模型(根据回合奖励)
agent.save_model(current_ep_reward, step)

# ============================ 定期保存与日志 ============================
if episode % 100 == 0:
# 保存训练状态检查点
data_obj = {
	'cumulative_score': cumulative_score,
	'epsilon': agent.epsilon,  # 当前探索率
	'epoch': step  # 当前训练轮次
}
with open(f'checkpoints/DDQN/{town}/checkpoint_ddqn.pickle', 'wb') as handle:
pickle.dump(data_obj, handle)

if episode % 10 == 0:
# 记录TensorBoard指标
writer.add_scalar("Cumulative Reward/info", cumulative_score, episode)
writer.add_scalar("Epsilon/info", agent.epsilon, episode)  # 探索率衰减曲线
writer.add_scalar("Episodic Reward/episode", scores[-1], episode)
writer.add_scalar("Reward/(t)", current_ep_reward, timestep)
writer.add_scalar("Episode Length (s)/info", np.mean(episodic_length), episode)
writer.add_scalar("Average Deviation from Center/episode", deviation_from_center / 10, episode)
writer.add_scalar("Average Distance Covered (m)/episode", distance_covered / 10, episode)

# 重置性能统计量
episodic_length = list()
deviation_from_center = 0
distance_covered = 0

print("Terminating the run.")
sys.exit()

\end{lstlisting}

\section{SAC算法Python描述}
\begin{lstlisting}
	
def main():
# ============================== 参数解析与环境初始化 ==============================
args = parse_args()  # 解析命令行参数
utils.set_seed_everywhere(args.seed)  # 设置全局随机种子

# 创建Carla环境实例
env = CarlaEnv(
render_display=args.render,        # 是否渲染显示(本地调试用)
display_text=args.render,          # 是否显示调试文本
changing_weather_speed=0.1,        # 天气变化速度(0表示静态天气)
rl_image_size=args.image_size,      # 输入图像的尺寸
max_episode_steps=2000,            # 单回合最大步数(对于导航任务可调整)
frame_skip=args.action_repeat,     # 动作重复帧数(类似Atari的帧跳过)
port=args.port,                    # Carla服务器端口
trafficManagerPort=args.trafficManagerPort,  # 交通管理器端口
scenarios=args.scenarios,          # 训练场景配置
algorithm=args.agent               # 使用的算法类型
)
eval_env = env  # 评估环境副本(此处暂未实现评估逻辑)

# ============================== 帧堆叠处理(像素输入时) ==============================
if args.encoder_type.startswith('pixel'):
# 将连续多帧堆叠为输入(通常4帧)
env = utils.FrameStack(env, k=args.frame_stack)
eval_env = utils.FrameStack(eval_env, k=args.frame_stack)

# ============================== 工作目录与路径管理 ==============================
# 创建带时间戳的工作目录
work_dir = args.work_dir + '/' + args.scenarios + "_" + args.agent + "_seed_{}_".format(args.seed) + time.strftime("%Y-%m-%d-%H-%M-%S")
utils.make_dir(work_dir)  # 创建主目录

# 创建子目录
video_dir = utils.make_dir(os.path.join(work_dir, 'video'))  # 视频保存路径
model_dir = utils.make_dir(os.path.join(work_dir, 'model'))  # 模型保存路径
buffer_dir = utils.make_dir(os.path.join(work_dir, 'buffer'))  # 经验回放缓存路径
res_dir = utils.make_dir(os.path.join(work_dir, 'res_dir'))  # 结果文件路径

video = VideoRecorder(video_dir if args.save_video else None)  # 视频记录器

# 保存参数配置到JSON文件
with open(os.path.join(work_dir, 'args.json'), 'w') as f:
json.dump(vars(args), f, sort_keys=True, indent=4)

# ============================== 设备初始化与动作空间验证 ==============================
device = torch.device('cuda' if torch.cuda.is_available() else 'cpu')  # 自动选择设备

# 验证动作空间标准化(确保动作值在[-1,1]范围内)
assert env.action_space.low.min() >= -1
assert env.action_space.high.max() <= 1

# ============================== 经验回放缓冲区配置 ==============================
if args.agent == "ddqn":  # 双深度Q网络特殊处理
replay_buffer = utils.ReplayBuffer_DDQN(
obs_shape=env.observation_space.shape,  # 观测空间形状
action_shape=env.action_space.shape,    # 动作空间形状
capacity=args.replay_buffer_capacity,   # 缓冲区容量(通常1e6)
batch_size=args.batch_size,             # 采样批次大小
device=device                          # 存储设备
)
else:  # 其他算法的标准缓冲区
replay_buffer = utils.ReplayBuffer(
obs_shape=env.observation_space.shape,
action_shape=env.action_space.shape,
capacity=args.replay_buffer_capacity,
batch_size=args.batch_size,
device=device
)

# ============================== 智能体初始化 ==============================
agent = make_agent(
obs_shape=env.observation_space.shape,  # 观测空间维度
action_shape=env.action_space.shape,    # 动作空间维度
args=args,                             # 配置参数
device=device                          # 计算设备
)

# ============================== 日志系统初始化 ==============================
L = Logger(work_dir, use_tb=args.save_tb)  # 创建日志记录器(可选TensorBoard)

# ============================== 训练主循环 ==============================
episode, episode_reward, done = 0, 0, True  # 初始化训练指标
expl_noise = 0.1  # 探索噪声系数(用于某些算法的动作扰动)
start_time = time.time()  # 训练计时开始

# 开始训练迭代
for step in range(args.num_train_steps):
if done:  # 回合结束处理
# 逆模型解码器的特殊处理(清空历史观测)
if args.decoder_type == 'inverse':
for i in range(1, args.k):
replay_buffer.k_obses[replay_buffer.idx - i] = 0

# 日志记录与模型保存
if step > 0:
L.log('train/duration', time.time() - start_time, step)  # 记录回合耗时
start_time = time.time()  # 重置计时器
L.dump(step)  # 写入日志文件

# 模型保存逻辑
if args.save_model:
agent.save_best(model_dir, episode_reward)  # 保存当前最佳模型

# 记录当前回合奖励
L.log('train/episode_reward', episode_reward, step)

# 将训练日志写入文本文件
train_log_filepath = os.path.join(res_dir, "train_log.txt")
train_log_txt_formatter = "{step}:{episode_reward}\n"
with open(train_log_filepath, "a") as f:
f.write(f"{step}:{episode_reward}\n")

# 环境重置与新回合初始化
obs = env.reset()
done = False
episode_reward = 0
episode_step = 0
episode += 1
reward = 0

L.log('train/episode', episode, step)  # 记录当前回合数
\end{lstlisting}
	
% \end{appendixs}

