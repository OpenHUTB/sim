%!TEX root = ../../csuthesis_main.tex

\chapter{总结与展望}

\section{总结}

近年来,自动驾驶技术成为科技界日益关注的热点,各大高校、汽车厂商以及众多互联网公司都在开展自动驾驶算法的研究。据相关数据显示,自动驾驶汽车已在全球多个国家开展测试,并在无人驾驶公交车、无人驾驶出租车、无人驾驶卡车等领域进入商业化阶段。人工智能也正在快速发展,深度学习和强化学习在各自领域展现出强大的威力,取得了令人瞩目的成果。目前市面上的自动驾驶算法研究主要集中在感知系统的开发,在决策和控制方面的进展较为缓慢。而深度强化学习的研究则展现了其在自动驾驶场景中强大的决策能力。因此,本研究希望将深度学习强大的感知能力与传统的车辆控制算法相结合,利用深度强化学习技术进行精准控制,从而改进现有的自动驾驶控制策略。

本文首先介绍了该算法的相关理论框架。深度强化学习基于强化学习的思想,并融合了深度神经网络强大的表征能力,使得强化学习算法能够随着时间推移解决复杂的问题。本文选取与所开发算法相关的部分,例如DQN、PPO算法,进行详细讲解,以帮助读者更好地理解算法设计。

其次,本研究提出基于车路云协同的高精度地图纯电动汽车感知作为自动驾驶汽车的感知层。该方法在高精度静态地图的基础上,通过车云协同、路云协同两种策略,创建能够为车辆提供精准实时道路使用者状况信息的动态图层。目前,大多数基于强化学习的自动驾驶算法使用来自多个摄像头的图像或激光雷达点云作为输入。但由于环境状态空间较大、模型收敛速度较慢,这些方法一般效果较差。本研究选取BEV作为高精度地图算法模型的输入数据,大大提高了认知层的信息密度。

接下来,在本研究中,我们提出了一种基于改进的 DQN 的端到端自主控制算法。该模型利用深度强化学习,可以使用高精度地图学习纯电动汽车的决策指南,并将其转化为驾驶指令。这些算法使车辆能够在复杂的道路上安全行驶,并遵守特定地点的交通规则。在本文中,我们首先详细介绍了算法的状态空间、动作空间和控制层的设计。然后,参考分层强化学习理论,针对自主控制任务的各个阶段设计不同的任务目标和相应的奖励函数。接下来提出了一种改进的DQN算法,并给出了改进的DQN算法的结构详细设计。

最后,本文基于CARLA模拟器搭建了CarlaEnv强化学习环境,并针对算法的学习阶段和测试阶段设计了两个对比实验,以证明本文所开发算法的有效性和优良性。本文开发了几种控制场景,并使用本文提出的自主控制算法和对照组的其他算法同时进行训练和测试。实验结果表明,本文提出的算法在学习阶段能够取得更好的学习效果和更快的收敛速度。同时,在测试阶段,本文提出的算法在给定的道路上能够很好地工作,并且量化性能可以接近甚至超越人类的手动控制,体现了算法的智能性。

\section{展望}

本文所提出的自动驾驶算法虽然在结构化道路测试中展现出显著的智能决策能力,但在系统完备性与工程适用性层面仍存在若干亟待突破的技术瓶颈。基于当前研究局限,本课题的未来演进路径可系统性地规划为以下三个核心方向:

(1)多模态交通场景的泛化能力提升

目前的算法大多基于理想化的道路交通地形结构进行训练,在场景概括能力方面存在明显的不足。未来的研究将致力于构建多层次的场景扩展系统。在空间维度上,我们计划融合LiDAR点云、视觉语义分割等多模态传感器数据,构建包含极端天气(暴雨/大雪)、突发事故(车辆受损/建筑物倒塌)、特殊路况(道路不平整/施工区域)等在内的异构场景数据库。在时间维度上,构建配送流动态生成机制,通过车、路、人、物的时空互联建模,实现早晚高峰配送、重大活动疏散等复杂配送模式的数字孪生。具体来说,在交通标志语义分析任务中,我们专注于使用图上的合成神经网络对高精度地图进行拓扑增强,创建动态再生的虚拟交通标志系统,并在具有物理约束的游戏中开发多智能体奖励机制,以便算法能够自适应地确定速度限制和交通规则等监管要素。

(2)奖励函数的自主进化机制研究

现有奖励函数的设计仍然依赖于基于专家经验的特征工程,存在表征不完整和环境偏差的问题。下一步,我们将考虑基于深度生成模型的奖励函数自优化结构。首先,构建分层奖励系统,将初始控制行为分解为轨迹遵守、碰撞风险值、能量效率等量化子目标,并利用逆强化学习(IRL)从人类控制数据中学习隐式奖励规则。其次,设计动态权重分配算法,利用元学习实现根据学习过程的进度自适应调整奖励系数,解决多目标优化中的梯度碰撞问题。同时,我们引入因果机制,建立一系列行动与长期利益之间的关联模式,避免局部最优导致的风险策略。此外,我们计划集成联邦学习系统,通过多场景并行训练实现奖励函数的迁移学习,提升不同地域算法适配的效率。

(3)智能化算法开发平台的生态构建

虽然目前的测试方法已经具备了大规模研究的能力,但是硬件和软件之间仍然存在差异。该计划旨在打造“一站式”智能研发生态:在架构最底层,利用容器化技术打造去中心化的学习课堂,配合ONNX架构改造和TensorRT加速,实现应用性能的大幅提升;在中层,开发了低代码可视化设计框架,提供场景生成(如道路分割、产品分割)、通过“拖拽”的方式组装算法模块(强化学习/模拟学习三步流程/热初始化流程)等基础任务。在网络应用的顶层,自主访问控制(SANB)系统设计为支持按照GB/T 38186-2021标准进行通信,包括基于ISO 26262的主动安全分析模块,以及与RARLA等关键架构进行通信的开放API。通过开发完整的“数据训练-验证-处理”工具,将算法的计算时间降低到目前的百分之三十以下。

