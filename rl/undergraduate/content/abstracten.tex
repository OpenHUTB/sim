%!TEX root = ../csuthesis_main.tex
\keywordsen{Digital Twin\ \ Autonomous Driving\ \ Reinforcement Learning\ \ Simulation System; \ \ Decision Optimization}
\begin{abstracten}

With the rapid development of artificial intelligence and autonomous driving technology, traditional 
testing and validation methods can no longer meet the demands of complex and variable driving 
environments. Digital twin technology, as an emerging simulation method, provides an efficient and safe 
testing platform for autonomous driving systems by constructing virtual models that correspond to the 
real world. Digital twins can reflect the state of physical entities in real-time, supporting precise 
simulations of various aspects such as vehicle dynamics and environmental changes, thereby providing 
rich data support for the training and optimization of autonomous driving algorithms. Reinforcement 
learning, as a self-learning intelligent algorithm, can optimize decision-making processes in constantly 
changing environments. Combining digital twins with reinforcement learning can not only accelerate the 
development and validation of autonomous driving systems but also enhance their adaptability and safety 
in complex scenarios. Researching a digital twin-based reinforcement learning simulation system for 
autonomous driving has significant theoretical significance and practical application value.

This article aims to explore the design and implementation of a digital twin-based reinforcement 
learning simulation system for autonomous driving. With the rapid development of autonomous driving 
technology, traditional testing and validation methods can no longer meet the increasingly complex 
driving environments and diverse driving scenarios. Digital twin technology, as an emerging simulation 
method, provides an efficient and safe testing platform for autonomous driving systems by constructing 
virtual models corresponding to the real world. This article first reviews the development history of 
autonomous driving technology, analyzes the basic concepts of digital twins and their application 
potential in the field of autonomous driving. It delves into the basic principles of reinforcement learning 
and its importance in autonomous driving, emphasizing the necessity of optimizing autonomous driving 
decisions through reinforcement learning algorithms.

On this basis, the article proposes a simulation system framework that combines digital twins and 
reinforcement learning, detailing the system's architectural design, functional modules, and 
implementation process. By constructing a digital twin model of a real environment, the system can 
simulate a large number of driving scenarios in a virtual environment, thereby accelerating the 
reinforcement learning training process. Experimental results show that the system has significant 
advantages in improving the accuracy and stability of autonomous driving decisions. The article also 
discusses the challenges and future development directions of the system in practical applications, 
pointing out solutions to issues such as data scarcity and convergence, providing references for 
subsequent research. The digital twin-based reinforcement learning simulation system for autonomous 
driving not only offers new ideas for the validation and optimization of autonomous driving technology 
but also provides valuable practical experience for researchers in related fields.

\end{abstracten}