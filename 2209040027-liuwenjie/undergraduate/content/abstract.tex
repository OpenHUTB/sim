\begin{abstractcn}
随着无人机集群技术的发展,高保真仿真平台成为验证集群算法的重要手段。AirSim作为主流仿真平台,在多无人机场景下存在性能瓶颈。本文深入分析AirSim在多机仿真中的性能问题,定位其根源在于低效的线程调度机制。通过设计并实现五种线程优化方案,显著提升了仿真规模上限。实验结果表明,优化后的系统支持无人机数量从18架提升至210架,为大规模集群算法验证提供了高效仿真环境。

\keywords{无人机集群;仿真平台;性能优化;线程调度;AirSim}
\end{abstractcn}

\begin{abstracten}
With the development of UAV swarm technology, high-fidelity simulation platforms have become important for validating swarm algorithms. AirSim, as a mainstream simulation platform, exhibits performance bottlenecks in multi-UAV scenarios. This paper deeply analyzes the performance issues of AirSim in multi-drone simulation, identifying the root cause as inefficient thread scheduling mechanisms. By designing and implementing five thread optimization schemes, the simulation scale limit is significantly improved. Experimental results show that the optimized system supports an increase in UAV count from 18 to 210, providing an efficient simulation environment for large-scale swarm algorithm validation.

\keywords{UAV Swarm; Simulation Platform; Performance Optimization; Thread Scheduling; AirSim}
\end{abstracten}
