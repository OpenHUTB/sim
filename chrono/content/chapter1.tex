\documentclass{hutbthesis}
\usepackage{ctex}
\usepackage{graphicx}
\usepackage{amsmath}
\usepackage{amsfonts}
\usepackage{amssymb}
\usepackage{enumitem} % 优化列表格式

% 论文基本信息(请根据实际情况替换占位符)
\title{基于Chrono与虚幻引擎的多体物理车辆仿真系统开发}
\author{你的姓名}
\advisor{你的指导教师 教授/副教授}
\major{车辆工程(或对应专业名称)}
\school{湖南工商大学}
\date{2025年5月} % 按实际提交日期修改

\begin{document}

% 封面(模板自动生成,无需手动编写)
\maketitle

% 摘要(若需补充,可在此处添加,示例如下)
\begin{abstract}
随着汽车工业向智能化、网联化转型,车辆研发对仿真测试的实时性、高精度和可视化需求日益迫切。本文聚焦于多体物理车辆仿真技术,提出基于Chrono多体动力学引擎与虚幻引擎的融合方案,解决传统仿真系统实时性差、可视化程度低等问题。首先阐述了研究背景与意义,梳理了国内外相关研究现状;其次明确了研究内容与技术路线;最后总结了研究创新点。本研究旨在构建高精度、高实时性的车辆仿真平台,为自动驾驶测试与车辆底盘性能验证提供技术支撑。
\end{abstract}

% 关键词(若需补充,可在此处添加)
\keywords{车辆仿真;Chrono引擎;虚幻引擎;多体动力学;实时可视化}

% 第一章 绪论
\chapter{绪论}
\section{研究背景与意义}
随着汽车工业向智能化、网联化深度转型,车辆研发流程对仿真测试的依赖度大幅提升。传统车辆多体物理仿真工具(如ADAMS、SIMPACK)虽具备高精度求解能力,但存在实时性差、可视化程度低、交互性弱等固有缺陷,难以满足自动驾驶海量场景测试、底盘性能动态验证等新兴研发需求。

虚幻引擎(Unreal Engine)作为主流的实时渲染引擎,凭借高保真场景渲染、灵活的人机交互设计能力,在工业可视化领域得到广泛应用;Chrono作为开源工程级多体动力学引擎,具备精准的车辆子系统建模、多物理场耦合求解能力,其精度可对标商业级仿真软件。将Chrono的高精度物理求解能力与虚幻引擎的高保真可视化优势相结合,构建“物理内核+可视化外壳”的车辆仿真系统,可实现“高精度动力学仿真-实时可视化交互-复杂场景适配”的协同,有效弥补传统仿真系统的短板,契合当前车辆研发数字化、实时化的发展趋势。

在此背景下,本研究聚焦于基于Chrono与虚幻引擎的多体物理车辆仿真系统开发,旨在解决多体动力学模型与可视化场景的实时同步、车辆核心子系统精准建模、复杂工况下仿真精度保障等关键技术问题,为车辆研发、自动驾驶测试等提供高效、可靠的仿真平台,具有重要的工程应用价值与学术研究意义。

\section{国内外研究现状}
\subsection{国外研究现状}
国外在车辆多体仿真与实时可视化融合领域起步较早,已形成较多阶段性成果。在多体动力学引擎应用方面,Chrono项目团队联合密歇根大学等科研机构,基于Chrono::Vehicle模块开发了多款车辆仿真模型,实现了悬架、轮胎等核心子系统的精准建模,并验证了其在越野车辆通过性仿真中的有效性;在虚实融合技术方面,通用、宝马等国际车企采用“专业物理引擎+实时渲染引擎”的技术路线,将Siemens Simcenter 3D与虚幻引擎结合,构建了车辆数字孪生仿真平台,成功应用于底盘性能标定与自动驾驶场景测试。

同时,国外学者在实时数据同步领域开展了深入研究,提出了基于UDP协议的轻量化数据传输方案,实现了物理仿真与可视化场景的毫秒级同步;在多物理场耦合方面,通过Chrono与OpenFOAM的协同,完成了车辆气动阻力与动力学性能的联合仿真,为复杂工况下的车辆性能分析提供了有力支撑。但国外研究多聚焦于高端车企的定制化方案,开源化、可复用的技术方案较少,难以满足通用化、低成本的研发需求。

\subsection{国内研究现状}
国内对车辆仿真技术的研究多集中于传统商业软件的应用与二次开发,例如基于ADAMS的车辆底盘性能仿真、基于MATLAB/Simulink的控制算法验证等。近年来,随着虚幻引擎、Chrono等开源工具的普及,国内高校与科研机构开始探索虚实融合的仿真技术路线:清华大学基于Chrono构建了电动车辆多体动力学模型,实现了动力传动系统的性能仿真;吉林大学将虚幻引擎与PhysX物理引擎结合,开发了自动驾驶场景仿真平台,完成了基础交通场景的可视化测试。

但国内研究仍存在明显不足:一是多数研究采用游戏级物理引擎(如PhysX),多体动力学求解精度不足,难以满足工程级应用需求;二是Chrono与虚幻引擎的协同多停留在简单数据传输层面,缺乏对同步延迟补偿、动力学精度校验的深入研究;三是场景建模与车辆物理行为的耦合度低,难以还原复杂路面、极端天气等真实环境对车辆性能的影响。因此,开发一款高精度、高实时性、高耦合度的基于Chrono与虚幻引擎的车辆仿真系统,具有明确的研究必要性。

\section{研究内容与方法}
\subsection{研究内容}
本研究旨在开发一套基于Chrono与虚幻引擎的多体物理车辆仿真系统,核心研究内容包括以下四个方面:
\begin{enumerate}[label=(\arabic{enumi})]
    \item 高精度车辆多体动力学模型构建:基于Chrono::Vehicle模块,完成车架、悬架、转向、传动、轮胎等核心子系统的建模与集成,确保模型动力学响应精度对标实车测试数据。
    \item 多体仿真与可视化场景的协同架构设计:设计“Chrono物理层-数据传输层-虚幻可视化层”的三层架构,基于UDP/TCP协议实现数据双向传输,引入延迟补偿算法降低同步延迟。
    \item 复杂仿真场景建模与工况测试设计:基于虚幻引擎构建多类型测试场景(铺装路面、湿滑路面、越野地形等),设计典型测试工况(匀速行驶、紧急制动、蛇形试验等),制定仿真数据与实车数据的对比指标。
    \item 系统功能集成与优化:开发交互界面,优化Chrono求解器参数与虚幻引擎渲染效率,完成系统稳定性测试,确保长时间仿真无数据丢失、无状态失真。
\end{enumerate}

\subsection{研究方法}
本研究采用的技术方法如下:
\begin{enumerate}[label=(\arabic{enumi})]
    \item 多体动力学建模方法:基于Chrono::Vehicle模块的标准化接口,采用参数化建模思想,实现各子系统的模块化构建与约束耦合。
    \item 实时同步技术:采用UDP协议实现轻量化数据传输,结合预测迭代法设计延迟补偿算法,保障物理仿真与可视化的实时一致性。
    \item 场景建模与渲染优化:利用虚幻引擎的地形编辑、材质系统与LOD(细节层次)技术,平衡场景保真度与渲染效率。
    \item 验证与优化方法:通过实车测试数据对标、典型工况仿真对比,验证系统精度;采用迭代优化策略,调整求解器参数与渲染设置,提升系统综合性能。
\end{enumerate}

\section{研究创新点}
本研究的创新点主要体现在以下三个方面:
\begin{enumerate}[label=(\arabic{enumi})]
    \item 提出“高精度物理内核+高保真可视化外壳”的协同架构:将Chrono的工程级多体动力学求解能力与虚幻引擎的实时渲染优势深度融合,弥补了传统仿真系统“精度与实时性不可兼得”的短板。
    \item 构建多子系统精准耦合的车辆动力学模型:针对悬架、轮胎等核心子系统采用工程级模型(如Pac2002轮胎模型),结合土壤力学模块适配复杂地形,提升了仿真的工程实用性。
    \item 设计低延迟数据同步机制:引入预测迭代法补偿传输延迟,实现物理仿真与可视化场景的毫秒级同步(≤10ms),保障了仿真过程的一致性与交互性。
\end{enumerate}

\end{document}