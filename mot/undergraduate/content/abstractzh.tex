%!TEX root = ../csuthesis_main.tex
% 设置中文摘要
\keywordscn{多目标多摄像头跟踪 (MTMCT)\quad 车辆轨迹复现\quad  检测跟踪模型优化\quad 多路口车辆轨迹拼接\quad CARLA仿真平台\quad  目标重识别(Re-ID)网络}
%\categorycn{TP391}
\begin{abstractzh}
	
本项目在 CarlaUE4仿真平台 、PyCharm 、Matlab2024b 这三个软件的基础上完成,通过整合 CARLA 仿真平台、多传感器融合(激光雷达、相机)与深度学习技术,完成面向智慧交通场景的多目标跟踪算法和评测。在CARLA仿真平台中结合相机与雷达传感器,进行多目标跟踪,获取车辆在多个路口的精确轨迹,并通过再识别技术整合整体车辆轨迹数据,实现对车辆轨迹的复现,从而完成面向智慧交通场景的多目标跟踪算法和评测。

本项目主要是在 CARLA 仿真平台中的town10 地图中对现有的检测跟踪模型进行 了优化。通过调整和改进模型的参数和结构,在 10 个关键性能指标中至少有 3 个实现 了超过 Baseline 5\% 的性能提升。这些指标包括但不限于跟踪精度、跟踪稳定性和计算效率。项目的优化策略不仅提高了模型的跟踪能力, 还增强了其在复杂交通场景下的鲁棒性。

为了更直观地展示本项目的设计和现有模型的性能,对于展示,在车辆轨迹跟踪性能方面项目采用了 Matlab2024b 中的可视化方法;在车辆轨迹复现性能方面项目采用了PyCharm 中的可视化方法。运行代码脚本后会出现视窗清楚地展示车辆跟踪与复现的性能效果。它们帮助在本项目中直观地展示了模型在跟踪复现任务中的表现,清晰地展示了我们模型跟踪、检测、复现轨迹功能的性能数据,尤其是在跟踪多个目标和在不同摄像头视角下进行车辆再识别时。

最后,本项目的研究不仅在理论上提供了对多目标多摄像头跟踪 (MTMCT) 和目标重 识别 (Re-ID) 特征提取的深入理解,而且在实践上提供了一种有效的模型优化和性能评估方法。我相信,这些成果将为未来的智能交通监控、自动驾驶测试、数字孪生城市等方面提供有力的技术支持。


\end{abstractzh}
