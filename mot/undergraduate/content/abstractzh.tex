%!TEX root = ../csuthesis_main.tex
% 设置中文摘要
\keywordscn{多目标多摄像头跟踪 (MTMCT)\quad 行人再识别 (Re-ID)\quad Ground Truth\quad 特征提取\quad 可视化展示\quad 检测跟踪模型优化}
%\categorycn{TP391}
\begin{abstractzh}
	
\indent 在此基础上,我们对现有的检测跟踪模型进行了优化。通过调整和改进模型的参数和结构,我们在10个关键性能指标中至少有3个实现了超过Baseline模型0.05的性能提升。这些指标包括但不限于跟踪精度、跟踪稳定性和计算效率。我们的优化策略不仅提高了模型的跟踪能力,还增强了其在复杂交通场景下的鲁棒性。

\indent 为了更直观地展示我们的设计和现有模型的性能,我们采用了多种可视化方法。这些方法包括热力图、轨迹图和混淆矩阵等,它们帮助我们直观地比较了不同模型在跟踪和识别任务中的表现。可视化结果清晰地展示了我们模型的优势,尤其是在跟踪多个目标和在不同摄像头视角下进行行人再识别时。

\indent 此外,我们的研究还涉及到了自适应加权三元组损失函数的开发,以及难点挖掘技术的应用。这些技术的引入进一步提高了模型在复杂场景下的特征提取能力,从而在DukeMTMC基准测试中实现了跟踪性能的超越,并在Market-1501和DukeMTMC-ReID基准测试中提升了Re-ID性能。

\indent 最后,我们的研究不仅在理论上提供了对MTMCT和Re-ID特征提取的深入理解,而且在实践上提供了一种有效的模型优化和性能评估方法。我们相信,这些成果将为未来的交通监控和安全系统设计提供有力的技术支持。



\end{abstractzh}