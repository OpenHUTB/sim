\chapter{面向未来的可能性尝试}
随着智慧交通的快速发展,多目标跟踪算法在交通管理、自动驾驶和智能监控等领域的重要性日益凸显。本文在现有研究基础上,利用CARLA仿真平台的Town10场景,优化了多目标跟踪模型,并在关键性能指标上实现了显著提升。未来的研究将着眼于进一步的技术创新和实际应用拓展。


\section{未来可能性尝试}
展望未来,我们计划从技术创新、实际应用拓展以及跨学科研究三个方向进一步推进研究\cite{voigtlaender2019mots}。这包括探索多模态数据融合、改进深度学习模型、引入强化学习,以及将算法应用于更复杂的实际场景和与其他交通管理系统进行深度集成。我们相信,通过这些努力,多目标跟踪算法将在智慧交通领域发挥更为关键的作用,为构建更加智能和高效的交通出行体系提供坚实的技术支撑。


\subsection{技术创新与模型优化}
多模态数据融合:未来可探索融合更多类型的传感器数据,如毫米波雷达和视觉数据的深度融合,以弥补单一传感器的不足,提升目标检测的准确性和鲁棒性。

深度学习模型的持续改进:研究更先进的深度学习架构,如Transformer,其在处理序列数据和捕捉长期依赖关系方面具有优势,有望进一步提高跟踪性能。

强化学习的应用:引入强化学习算法,使模型能够根据实时交通状况动态调整跟踪策略,以适应复杂多变的交通环境。

\subsection{实际应用拓展}

复杂场景下的应用:将优化后的模型应用于更具挑战性的实际交通场景,如恶劣天气条件下的交通跟踪,或高密度人流与车流混合区域的跟踪\cite{wang2021multiple}。

交通行为分析与预测:基于多目标跟踪结果,深入分析交通参与者的的行为模式,预测其未来行动,为交通管理和自动驾驶决策提供更丰富的信息。

与交通管理系统的集成:探索如何将多目标跟踪算法与现有的交通信号控制系统、交通流量管理系统等进行深度集成,以实现更智能的交通流量优化和事件响应。

\subsection{跨学科研究}

与心理学和行为学的交叉研究:结合心理学和行为学的研究成果,更准确地模拟和预测驾驶员及行人的行为,提升多目标跟踪算法在复杂交通场景中的表现\cite{fang2021咬}。

法律与伦理考量:随着自动驾驶和智能交通系统的发展,研究与多目标跟踪相关的法律和伦理问题,确保技术的应用符合社会规范和法律要求\cite{孙金颖 2024 基于虚拟仿真环境的端到端自动驾驶方法研究}。

\subsection{未来算法的可扩展性与适应性}

模块化设计:开发具有模块化设计的多目标跟踪系统,使得各个组件(如检测、关联、跟踪)能够独立更新和优化,以快速适应新的需求和技术进步。

云边协同计算:利用云计算的强大处理能力和边缘计算的低延迟优势,设计云边协同的多目标跟踪系统,以应对大规模交通场景中的数据处理需求\cite{辛泊言 2024 基于深度强化学习的自动驾驶决策算法研究}。




\section{项目拓展:项目尝试的实际应用}
本项目拓展,使用中电软件园的仿真场景来做车辆的数字孪生。轨迹信息并不是一个个相距很近的航点,而是车辆经过每个路口时的信息,其包括车牌、车牌种类、过车时间、设备编号、路口名称、车道编号以及方向编号。项目测试的数据选自2024年6~8月中的6月3日7:00-7:20的车辆数据。其如下图所示:


另外,再建立一个通用数据库,将路口、车道、方向映射到CARLA仿真场景中道路终点坐标位置,包括x, y, z和yaw。这样,便可以控制车辆从一个路口导航到另一个路口。

相对的,本项目获取到了中电软件园的仿真场景中小车轨迹坐标数据,便可以仿照上一节当中的步骤,将他们都放在Waypoints.txt文件夹中,方便后续代码的调用。

\subsection{规划原理}

\subsubsection{拓扑地图提取}

拓扑地图提取是从CARLA环境中提取道路网络结构的过程,旨在为后续的路径规划提供基础。首先,通过与CARLA的地图服务器连接,获取整个地图的道路信息。这个过程包括识别每条道路的起点和终点,并记录它们之间的重要路径节点。这些信息随后被整理成一个有向图,节点代表道路的入口和出口,而边则表示它们之间的连接关系。最终,这些数据被结构化存储,以便后续的导航和决策使用。

\subsubsection{搜索Graph 建立}

在拓扑地图构建完成后,下一步是进行图的搜索与路径规划。这个过程旨在找到从起点到终点的最佳行驶路线。

构建的图是一个有向图,其中节点代表道路网络中的特定位置(例如交叉口、路径的起点和终点),而边则代表连接这些位置的道路段。每条边不仅记录了连接的两个节点,还包含了道路的相关属性,如长度、方向、是否为交叉口等。



\subsubsection{寻找最优路径}

最终,经过路径搜索算法的计算,将返回一系列按顺序排列的节点。这些节点代表车辆应遵循的行驶路径,包括必要的转向和变道信息。返回的路径不仅考虑了行驶的顺畅性,还会在可能的情况下优化行驶时间,确保安全性和效率。


\subsubsection{车辆控制}

即使我们使用的是CARLA自带的全局路径规划模块,其底层还是使用的控制算法来控制车辆按指定航点行驶。规划路径的同时,在航点之间已经生成了一系列的驾驶决策。当然,还可以自己控制车辆的行为,如刹车、油门等。当一辆车从一个路口行驶到另一个路口后,设置车辆的下一个路口的目标位置,这样我们就完成了基于CARLA的智能交叉口导航与数字孪生系统。

\subsubsection{具体开发与使用}

完成上面提到的步骤之后,首先需要搭建路口车辆航点到CARLA坐标映射的数据库,安装mysql,搭建数据库其详细的执行,步骤在sql.txt文件中,选择代码块逐步执行。其次需要下载中电软件园的CARLA地图。再环境搭建,完成相应的环境配置。最后打开CARLA地图,此时CARLA地图已经是中电软件园的CARLA地图,运行 twin navigation.py脚本,便可以使得小车在真实的场景道路环境下,完成复现与自动驾驶。








