\chapter{}

\section{代码Drived.py关键部分}
\begin{lstlisting}
# 从文件加载路径点
waypoints_by_vehicle = {}
with open(WAYPOINTS_FILENAME, 'r') as f:
reader = csv.reader(f)
for row in reader:
vehicle_id = int(row[0])
x, y, t = row[1:4]
if vehicle_id not in waypoints_by_vehicle:
waypoints_by_vehicle[vehicle_id] = []
waypoints_by_vehicle[vehicle_id].append((x, y, t))

# 路径点插值平滑
smoothed_waypoints_by_vehicle = {}
for vehicle_id, waypoints in waypoints_by_vehicle.items():
waypoints_np = np.array(waypoints)
wp_interp = []
for i in range(len(waypoints_np) - 1):
wp_interp.append(waypoints_np[i])
distance = np.linalg.norm(waypoints_np[i + 1] - waypoints_np[i])
if distance > INTERP_DISTANCE_RES:
num_interp_points = int(distance / INTERP_DISTANCE_RES)
for j in range(1, num_interp_points):
interp_point = waypoints_np[i] + (waypoints_np[i + 1] - waypoints_np[i]) * (j / num_interp_points)
wp_interp.append(interp_point)
wp_interp.append(waypoints_np[-1])
smoothed_waypoints_by_vehicle[vehicle_id] = wp_interp

# 生成车辆并初始化控制器
vehicles = []
controllers = []
for vehicle_id, waypoints in waypoints_by_vehicle.items():
location = carla.Location(x=waypoints[0][0], y=waypoints[0][1], z=1)
vehicle = world.try_spawn_actor(random.choice(vehicle_bp), carla.Transform(location))
if vehicle:
controller = Controller.Controller(waypoints, args.lateral_controller, args.longitudinal_controller)
vehicles.append(vehicle)
controllers.append(controller)

# 仿真主循环
for frame in range(FRAME_NUM):
world.tick()  # 进行仿真步
for i, vehicle in enumerate(vehicles):
current_x, current_y, current_yaw = get_current_pose(vehicle)
current_speed = vehicle.get_velocity().length()

# 更新控制器状态
controllers[i].update_waypoints(smoothed_waypoints_by_vehicle[vehicle_ids[i]])
controllers[i].update_values(current_x, current_y, current_yaw, current_speed, frame)
controllers[i].update_controls()

# 获取控制指令
throttle, steer, brake = controllers[i].get_commands()

# 发送控制指令
send_control_command(vehicle, throttle, steer, brake)


# 仿真结束
cleanup_resources(world)
\end{lstlisting}

\section{代码collect lidar dataset.py关键部分}


\begin{lstlisting}
	

# 雷达参数
LIDAR_RANGE = 50
POINT_SAVE_TIME = 3000

# 保存点云数据和标签
def save_data(radar_data, world, location, lidar_to_world_inv, sensor_queue):
timestamp = world.get_snapshot().timestamp.elapsed_seconds
current_frame = radar_data.frame

# 获取雷达范围内的车辆
vehicle_list = world.get_actors().filter("*vehicle*")
vehicle_list = [v for v in vehicle_list if v.get_location().distance(location) < LIDAR_RANGE]

labels = []
for vehicle in vehicle_list:
bbox = vehicle.bounding_box
bbox_location = np.array([bbox.location.x, bbox.location.y, bbox.location.z, 1])
bbox_location_lidar = lidar_to_world_inv @ bbox_location
labels.append(bbox_location_lidar[:3])

# 保存点云数据
points = np.frombuffer(radar_data.raw_data, dtype=np.dtype('f4'))
points = np.reshape(points, (len(points) // 4, 4))
location = points[:, :3].astype(np.float64)
intensity = points[:, 3].reshape(-1, 1).astype(np.float64)

datalog = {
	'PointCloud': {
		'Location': location,
		'Intensity': intensity
	},
	'Labels': labels,
	'Timestamp': timestamp
}
sensor_queue.put(datalog)

# 设置雷达传感器
def setup_sensors(world, transform, sensor_queue):
lidar_bp = world.get_blueprint_library().find('sensor.lidar.ray_cast')
lidar_bp.set_attribute('range', '200')
lidar_bp.set_attribute('points_per_second', '2200000')
lidar_bp.set_attribute('rotation_frequency', '20')
lidar = world.spawn_actor(lidar_bp, transform)
lidar.listen(lambda data: save_data(data, world, transform.location, np.linalg.inv(np.array(transform.get_matrix())), sensor_queue))
return lidar

# 主函数
def main():
client = carla.Client('localhost', 2000)
client.set_timeout(10.0)
world = client.get_world()
settings = world.get_settings()
settings.fixed_delta_seconds = 0.05
settings.synchronous_mode = True
world.apply_settings(settings)

try:
sensor_queue = Queue()
lidar_transform = carla.Transform(carla.Location(x=-46, y=21, z=1.8), carla.Rotation(pitch=0, yaw=90, roll=0))
lidar = setup_sensors(world, lidar_transform, sensor_queue)

for _ in range(POINT_SAVE_TIME):
world.tick()
datalog = sensor_queue.get(True, 1.0)
scipy.io.savemat(f"data/{_}.mat", {'datalog': datalog})

finally:
settings.synchronous_mode = False
world.apply_settings(settings)
if lidar:
lidar.stop()
lidar.destroy()


	
\end{lstlisting}


\section{代码collect intersection camera lidar.py关键部分}

\begin{lstlisting}


# 雷达参数
DATA_MUN = 500
FUSION_DETECTION_ACTUAL_DIS = 25  # 多目标跟踪的实际检测距离

# 创建文件夹
def create_folder(folder_name):
if not os.path.exists(folder_name):
os.makedirs(folder_name)
return folder_name

# 保存车辆标签
def save_point_label(world, location, lidar_to_world_inv, timestamp, all_vehicle_labels):
vehicle_list = world.get_actors().filter("*vehicle*")
vehicle_list = [v for v in vehicle_list if v.get_location().distance(location) < FUSION_DETECTION_ACTUAL_DIS]
vehicle_labels = []
for vehicle in vehicle_list:
bbox = vehicle.bounding_box
bbox_location = np.array([bbox.location.x, bbox.location.y, bbox.location.z, 1])
bbox_location_lidar = lidar_to_world_inv @ bbox_location
label = [
bbox_location_lidar[0],
bbox_location_lidar[1],
bbox_location_lidar[2] + 0.3,
2 * bbox.extent.x,
2 * bbox.extent.y,
2 * bbox.extent.z
]
vehicle_labels.append((timestamp, vehicle.id, label))
all_vehicle_labels.append(vehicle_labels)

# 保存雷达数据
def save_radar_data(radar_data, world, ego_vehicle_transform, lidar_to_world_inv, all_vehicle_labels, junc, town_folder):
timestamp = world.get_snapshot().timestamp.elapsed_seconds
location = ego_vehicle_transform.location
save_point_label(world, location, lidar_to_world_inv, timestamp, all_vehicle_labels)

points = np.frombuffer(radar_data.raw_data, dtype=np.dtype('f4'))
points = np.reshape(points, (len(points) // 4, 4))
location = points[:, :3].astype(np.float64)
intensity = points[:, 3].reshape(-1, 1).astype(np.float64)

radar_folder = create_folder(f"{town_folder}/{junc}")
file_name = os.path.join(radar_folder, f"{radar_data.frame}.mat")
LidarData = {
	'PointCloud': {
		'Location': location,
		'Intensity': intensity
	},
	'Timestamp': timestamp
}
scipy.io.savemat(file_name, {'LidarData': LidarData})

# 设置传感器
def setup_sensors(world, transform, sensor_queue):
lidar_bp = world.get_blueprint_library().find('sensor.lidar.ray_cast')
lidar_bp.set_attribute('channels', '64')
lidar_bp.set_attribute('range', '200')
lidar_bp.set_attribute('points_per_second', '2200000')
lidar_bp.set_attribute('rotation_frequency', '20')
lidar = world.spawn_actor(lidar_bp, transform)
lidar.listen(lambda data: sensor_queue.put((data, "lidar")))
return lidar

# 生成自动驾驶车辆
def spawn_autonomous_vehicles(world, tm, num_vehicles=70, random_seed=42):
random.seed(random_seed)
vehicle_list = []
blueprint_library = world.get_blueprint_library()
vehicle_blueprints = blueprint_library.filter('vehicle.*')
vehicle_blueprints = [bp for bp in vehicle_blueprints if 'bike' not in bp.id]
for _ in range(num_vehicles):
transform = random.choice(world.get_map().get_spawn_points())
vehicle_bp = random.choice(vehicle_blueprints)
vehicle = world.try_spawn_actor(vehicle_bp, transform)
if vehicle:
vehicle.set_autopilot(True)
tm.ignore_lights_percentage(vehicle, 100)
vehicle_list.append(vehicle)
return vehicle_list

# 主函数
def main():
client = carla.Client('localhost', 2000)
client.set_timeout(10.0)
world = client.get_world()
settings = world.get_settings()
settings.fixed_delta_seconds = 0.05
settings.synchronous_mode = True
world.apply_settings(settings)
tm = client.get_trafficmanager(8000)
tm.set_synchronous_mode(True)

try:
town_folder = create_folder("Town10HD_Opt")
junc = "test_data_junc1"
ego_transform = carla.Transform(carla.Location(x=-46, y=21, z=1), carla.Rotation(pitch=0, yaw=90, roll=0))
lidar_transform = carla.Transform(carla.Location(x=ego_transform.location.x, y=ego_transform.location.y, z=ego_transform.location.z + 0.82), ego_transform.rotation)
lidar_to_world = np.array(lidar_transform.get_matrix())
lidar_to_world_inv = np.linalg.inv(lidar_to_world)

vehicles = spawn_autonomous_vehicles(world, tm, num_vehicles=50, random_seed=20)
sensor_queue = Queue()
lidar = setup_sensors(world, lidar_transform, sensor_queue)
all_vehicle_labels = []

for _ in range(DATA_MUN):
world.tick()
data, sensor_name = sensor_queue.get(True, 1.0)
if sensor_name == "lidar":
save_radar_data(data, world, ego_transform, lidar_to_world_inv, all_vehicle_labels, junc, town_folder)

finally:
settings.synchronous_mode = False
world.apply_settings(settings)
if lidar:
lidar.stop()
lidar.destroy()
for vehicle in vehicles:
vehicle.destroy()

	

\end{lstlisting}

\section{代码collect reid dataset.py关键部分}

\begin{lstlisting}
	
import carla
import os
import cv2
import numpy as np
import scipy.io
from queue import Queue

# 设置参数
DROP_BUFFER_TIME = 60
IMAGES_SAVE_TIME = 20
OUTPUT_DIR = "./reid_data"
camera_location = [
carla.Transform(carla.Location(x=-111, y=-2, z=2.800176), carla.Rotation(yaw=-90)),
carla.Transform(carla.Location(x=-106, y=-25, z=2.800176), carla.Rotation(yaw=90))
]

# 图像保存函数
def save_image(image, folder_path):
image_array = np.array(image.raw_data)
image_array = image_array.reshape((image.height, image.width, 4))[:, :, :3]
current_frame = image.frame
img_path = os.path.join(folder_path, f"{current_frame}.jpeg")
cv2.imwrite(img_path, image_array)

# 创建相机传感器
def create_camera_sensor(world, transform):
camera_bp = world.get_blueprint_library().find('sensor.camera.rgb')
camera_bp.set_attribute('image_size_x', '1920')
camera_bp.set_attribute('image_size_y', '1080')
camera_bp.set_attribute('fov', '90')
camera = world.spawn_actor(camera_bp, transform)
return camera

# 生成车辆并拍摄图像
def generate_vehicle_images(vehicle_type, folder_name, world, spawn_points, tm):
folder_path = os.path.join(OUTPUT_DIR, folder_name)
if not os.path.exists(folder_path):
os.makedirs(folder_path)

spawn_point = spawn_points[13]
vehicle = world.try_spawn_actor(vehicle_type, spawn_point)
vehicle.set_autopilot(True)
tm.ignore_lights_percentage(vehicle, 100)

cameras = []
for cam_loc in camera_location:
camera = create_camera_sensor(world, cam_loc)
camera.listen(lambda image: save_image(image, folder_path))
cameras.append(camera)

for _ in range(DROP_BUFFER_TIME):
world.tick()
time.sleep(0.05)

for _ in range(IMAGES_SAVE_TIME):
world.tick()
time.sleep(0.05)

for cam in cameras:
cam.stop()
cam.destroy()
vehicle.destroy()

# 主函数
def main():
client = carla.Client('localhost', 2000)
client.set_timeout(10.0)
world = client.get_world()
settings = world.get_settings()
settings.fixed_delta_seconds = 0.05
settings.synchronous_mode = True
world.apply_settings(settings)
tm = client.get_trafficmanager(8000)
tm.set_synchronous_mode(True)

blueprint_library = world.get_blueprint_library()
vehicle_blueprints = blueprint_library.filter('vehicle.*')
vehicle_blueprints = [bp for bp in vehicle_blueprints if 'bike' not in bp.id]
spawn_points = world.get_map().get_spawn_points()

for folder_index, vehicle_bp in enumerate(vehicle_blueprints):
folder_name = f"{folder_index + 1}"
generate_vehicle_images(vehicle_bp, folder_name, world, spawn_points, tm)

settings.synchronous_mode = False
world.apply_settings(settings)

if __name__ == "__main__":
main()
	
	


\end{lstlisting}

\section{代码evaluator.py关键部分}


\begin{lstlisting}

import numpy as np
from fastdtw import fastdtw

# 计算轨迹重合度
def trajectory_overlap_dtw(truth_trajectory, track_trajectory, threshold=0.5):
truth_points = np.array([[point[0], point[1]] for point in truth_trajectory])
track_points = np.array([[point[0], point[1]] for point in track_trajectory])
min_length = min(len(truth_points), len(track_points))
truth_points = truth_points[:min_length]
track_points = track_points[:min_length]
distance, _ = fastdtw(truth_points, track_points)
max_distance = np.max(np.linalg.norm(truth_points - track_points, axis=1)) * min_length
overlap_ratio = 1 - (distance / max_distance)
return max(0, min(1, overlap_ratio))

# 计算轨迹指标
def trajectory_metrics(truth, track, threshold=0.5):
total_error = 0.0
total_max_error = 0.0
total_tor = 0.0
total_fpe = 0.0
num_vehicles = 0
total_points = 0

for (truth_id, truth_trajectory), (track_id, track_trajectory) in zip(truth.items(), track.items()):
min_length = min(len(truth_trajectory), len(track_trajectory))
if min_length == 0:
continue
vehicle_error = 0.0
vehicle_max_error = 0.0
for i in range(min_length):
truth_point = truth_trajectory[i]
track_point = track_trajectory[i]
error = np.sqrt((truth_point[0] - track_point[0]) ** 2 + (truth_point[1] - track_point[1]) ** 2)
vehicle_error += error
vehicle_max_error = max(vehicle_max_error, error)
total_points += 1
vehicle_tor = trajectory_overlap_dtw(truth_trajectory, track_trajectory, threshold)
truth_end_point = truth_trajectory[-1]
track_end_point = track_trajectory[-1]
fpe = np.sqrt((truth_end_point[0] - track_end_point[0]) ** 2 + (truth_end_point[1] - track_end_point[1]) ** 2)
total_error += vehicle_error
total_max_error += vehicle_max_error
total_tor += vehicle_tor
total_fpe += fpe
num_vehicles += 1

mean_error = total_error / total_points if total_points > 0 else 0.0
mean_max_error = total_max_error / num_vehicles if num_vehicles > 0 else 0.0
mean_tor = total_tor / num_vehicles if num_vehicles > 0 else 0.0
mean_fpe = total_fpe / num_vehicles if num_vehicles > 0 else 0.0
return mean_tor, mean_error, mean_max_error, mean_fpe

# 计算平均误差和延迟
def mean_metrics(lateral_errors, longitudinal_errors, delays):
total_lateral_error = 0.0
total_longitudinal_error = 0.0
total_delay = 0.0
total_frames = 0

for lateral_vehicle, longitudinal_vehicle, delay_vehicle in zip(lateral_errors, longitudinal_errors, delays):
min_length = min(len(lateral_vehicle), len(longitudinal_vehicle), len(delay_vehicle))
if min_length == 0:
continue
total_lateral_error += sum(abs(e) for e in lateral_vehicle[:min_length])
total_longitudinal_error += sum(abs(e) for e in longitudinal_vehicle[:min_length])
total_delay += sum(d for d in delay_vehicle[:min_length])
total_frames += min_length

if total_frames == 0:
return 0.0, 0.0, 0.0
mean_lateral_error = total_lateral_error / total_frames
mean_longitudinal_error = total_longitudinal_error / total_frames
mean_delay = total_delay / total_frames
return mean_lateral_error, mean_longitudinal_error, mean_delay



\end{lstlisting}


\section{代码evaluator.py关键部分}


\begin{lstlisting}
	
# 配置和初始化
# 设置CARLA世界参数

def setting_config(settings, world):
settings.synchronous_mode = True
settings.fixed_delta_seconds = 0.05
world.apply_settings(settings)
	
	
# 数据加载与处理
# 读取车辆路径数据
with open(waypoints_file, 'r', encoding='utf-8') as f:
reader = csv.reader(f, delimiter=';')
for row in reader:
vehicle_id = int(row[0])
t = row[2]
inter = row[4]
lane = int(row[5]) + 1
direct = row[6]
if vehicle_id in waypoints_by_vehicle:
waypoints_by_vehicle[vehicle_id].append([inter, lane, direct])
else:
waypoints_by_vehicle[vehicle_id] = [[inter, lane, direct]]
if vehicle_id in intersection_time:
intersection_time[vehicle_id].append(t)
else:
intersection_time[vehicle_id] = [t]
	
	
# 数据库交互
# 查询车辆生成位置
def query_location(waypoint, cursor):
intersection = waypoint[0]
lane = waypoint[1]
direction = waypoint[2]
intersection_id = query_id(intersection, cursor)
query = '''
SELECT x, y, z, yaw FROM zhongdianruanjianyuan
WHERE IntersectionID = %s AND Lane = %s AND DirectionID = %s
'''
cursor.execute(query, (intersection_id, lane, direction))
results = cursor.fetchall()
loc_x, loc_y, loc_z, ro_yam = results[0]
loc_x = float(loc_x)
loc_y = float(loc_y)
loc_z = float(loc_z) - RES_SPAWN
ro_yaw = float(ro_yam)
location = carla.Location(x=loc_x, y=loc_y, z=loc_z)
return location, ro_yaw
	
# 车辆生成

def spawn_vehicle(vehicle_id, waypoints_by_vehicle, vehicle_waypoint_offset, cursor, filtered_vehicle_blueprints,
world, vehicle_list, vehicle_plate_list, agent_list, share_global_planner):
first_waypoint = waypoints_by_vehicle[vehicle_id][0]
location, ro_yaw = query_location(first_waypoint, cursor)
transform = carla.Transform(location, carla.Rotation(yaw=ro_yaw))
vehicle_bp = random.choice(filtered_vehicle_blueprints)
vehicle = world.try_spawn_actor(vehicle_bp, transform)
while vehicle is None:
transform.location -= transform.get_forward_vector() * OFFSET_DISTANCE
vehicle = world.try_spawn_actor(vehicle_bp, transform)
vehicle_list[vehicle_id] = vehicle
vehicle_plate_list.append(vehicle_id)
agent = BehaviorAgent(vehicle, behavior='normal', grp_inst=share_global_planner)
agent_list[vehicle_id] = agent
agent.last_action_time = time.time()
second_waypoint = waypoints_by_vehicle[vehicle_id][1]
location, ro_yam = query_location(second_waypoint, cursor)
agent.set_destination(location)
vehicle_waypoint_offset[vehicle_id] = 2


# 批量控制车辆
def batch_control_vehicles(agent_list, world, client):
batch_size = 20
agents = list(agent_list.values())
commands = []
for i in range(0, len(agents), batch_size):
batch = agents[i:i + batch_size]
for agent in batch:
vehicle = agent._vehicle
if not agent.done():
control = agent.run_step()
commands.append(ApplyControl(vehicle, control))
client.apply_batch(commands)
commands.clear()
world.tick()


# 清理已销毁车辆的资源
def clear_source(need_to_destroy, waypoints_by_vehicle, intersection_time, vehicle_list,
vehicle_waypoint_offset, vehicle_plate_list, agent_list):
for vehicle_id in need_to_destroy:
del waypoints_by_vehicle[vehicle_id]
del intersection_time[vehicle_id]
del vehicle_list[vehicle_id]
del vehicle_waypoint_offset[vehicle_id]
del agent_list[vehicle_id]
vehicle_plate_list.remove(vehicle_id)

while len(vehicle_list) > 0:
# 按时间顺序生成车辆
if next_vehicle_num < len(start_time):
this_stamp = get_timestamp(start_time[next_vehicle_num])
during_to_next_vehicle = this_stamp - first_stamp
this_system_stamp = world.get_snapshot().timestamp.elapsed_seconds
during_system_time = this_system_stamp - first_system_stamp
if during_system_time >= during_to_next_vehicle:
num = start_time.count(start_time[next_vehicle_num])
for vehicle_id in vehicle_id_list[next_vehicle_num:]:
spawn_vehicle(vehicle_id, waypoints_by_vehicle, vehicle_waypoint_offset, cursor,
filtered_vehicle_blueprints, world, vehicle_list, vehicle_plate_list, agent_list, share_global_planner)
next_vehicle_num += 1
if next_vehicle_num >= len(start_time):
break
first_stamp = this_stamp
first_system_stamp = this_system_stamp

# 控制车辆行为
batch_control_vehicles(agent_list, world, client)

# 检查车辆是否到达终点或需要销毁
for vehicle_id, agent in agent_list.items():
vehicle = vehicle_list[vehicle_id]
if agent.done() and vehicle_waypoint_offset[vehicle_id] >= len(waypoints_by_vehicle[vehicle_id]):
vehicle.destroy()
need_to_destroy.append(vehicle_id)
destroyed_vehicle_id.append(vehicle_id)

# 清理已销毁车辆的资源
if need_to_destroy:
clear_source(need_to_destroy, waypoints_by_vehicle, intersection_time, vehicle_list,
vehicle_waypoint_offset, vehicle_plate_list, agent_list)
need_to_destroy.clear()





# 车辆生命周期差异计算
time_differences = calculate_vehicle_lifespan(vehicle_lifetimes, intersection_time)
save_lifespan_to_csv(time_differences, csv_filename='vehicle_lifespan_differences.csv')
plot_lifespan_differences_line(time_differences, image_filename='vehicle_lifespan_differences.png', limit=100)

# 路径误差评估
path_errors = evaluate_path_error(vehicle_middle_junctions, vehicle_actual_junctions)
plot_path_errors(path_errors)
	
\end{lstlisting}


\section{代码DEMO.m关键部分}


\begin{lstlisting}
	
	%% 将生成的轨迹与外观特征绑定
	%% 匹配轨迹之前,处理单个路口的轨迹id变化的情况
	config;
	townName = 'Town10';   % 可以修改城镇和;路口
	dataSets = {'Town10HD_Opt\test_data_junc1', 'Town10HD_Opt\test_data_junc2', 'Town10HD_Opt\test_data_junc3', 'Town10HD_Opt\test_data_junc4', 'Town10HD_Opt\test_data_junc5'};
	
	dirParts = strsplit(dataSets{1}, '\');
	townConfig = dataset.(townName);
	for i = 1:length(dataSets)
	juncField = sprintf('intersection_%d', i);
	juncConfig = townConfig.(juncField);
	transMatrix = juncConfig.TransformationMatrix;
	loadAllTraj(dataSets{i}, transMatrix);
	end
	%% 加载所有轨迹
	currentPath = fileparts(mfilename('fullpath'));
	juncTracksFolderPath = fullfile(currentPath, dirParts{1});
	% 获取所有轨迹文件
	matFiles = dir(fullfile(juncTracksFolderPath, "*.mat"));
	numMatFiles = length(matFiles);
	% 创建cell数组保存每个路口的轨迹
	juncTrajCell = cell(1,numMatFiles);
	for file = 1:numMatFiles
	fileName = fullfile(juncTracksFolderPath, matFiles(file).name);
	data = load(fileName);
	juncTrajCell{file} = data.juncVehicleTraj;
	end
	
	%% 轨迹匹配,链接全部路口的轨迹
	matchThreshold = 0.65;  % 车辆匹配阈值
	traj = linkIdentities(juncTrajCell, matchThreshold);
	
	% 获取当前目录
	currentFileDir = fileparts(mfilename('fullpath')); 
	
	% 保存 traj 到当前目录下的 traj.mat 文件
	save(fullfile(currentFileDir, 'traj.mat'), 'traj');
	
	
\end{lstlisting}
