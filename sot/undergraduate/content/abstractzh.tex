%!TEX root = ../csuthesis_main.tex
% 设置中文摘要
\keywordscn{数字孪生\quad 自动驾驶\quad 强化学习\quad 仿真系统\quad 决策优化}
%\categorycn{TP391}
\begin{abstractzh}

    随着人工智能和自动驾驶技术的迅猛发展,传统的测试与验证方法已无法满足复杂
    多变的驾驶环境需求。数字孪生技术作为一种新兴的仿真手段,通过构建与现实世界相
    对应的虚拟模型,为自动驾驶系统提供了高效、安全的测试平台。数字孪生能够实时反
    映物理实体的状态,支持对车辆动态、环境变化等多方面的精准模拟,从而为自动驾驶
    算法的训练与优化提供丰富的数据支持。强化学习作为一种自我学习的智能算法,能够
    在不断变化的环境中优化决策过程。将数字孪生与强化学习相结合,不仅可以加速自动
    驾驶系统的开发与验证,还能提升其在复杂场景下的适应能力和安全性。研究基于数字
    孪生的自动驾驶强化学习仿真系统具有重要的理论意义和实际应用价值。
    
    本文旨在探讨基于数字孪生的自动驾驶强化学习仿真系统的设计与实现。随着自动
    驾驶技术的快速发展,传统的测试与验证方法已无法满足日益复杂的驾驶环境和多样化
    的驾驶场景需求。数字孪生技术作为一种新兴的仿真手段,通过构建与现实世界相对应
    的虚拟模型,为自动驾驶系统提供了一个高效、安全的测试平台。本文首先回顾了自动
    驾驶技术的发展历程,分析了数字孪生的基本概念及其在自动驾驶领域的应用潜力。深
    入探讨了强化学习的基本原理及其在自动驾驶中的重要性,强调了通过强化学习算法优
    化自动驾驶决策的必要性。
    
    在此基础上,本文提出了一种结合数字孪生与强化学习的仿真系统框架,详细描述
    了系统的架构设计、功能模块及实现过程。通过构建一个真实环境的数字孪生模型,系
    统能够在虚拟环境中进行大量的驾驶场景仿真,进而加速强化学习的训练过程。实验结
    果表明,该系统在提高自动驾驶决策的准确性与稳定性方面具有显著优势。本文还讨论
    了系统在实际应用中的挑战与未来发展方向,指出了数据稀缺、收敛性等问题的解决方
    案,为后续研究提供了参考。基于数字孪生的自动驾驶强化学习仿真系统不仅为自动驾
    驶技术的验证与优化提供了新的思路,也为相关领域的研究者提供了有价值的实践经验。
    

\end{abstractzh}