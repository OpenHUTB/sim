%!TEX root = ../../csuthesis_main.tex
\chapter{研究理论基础}

本章旨在对本研究所依赖的关键算法与基础理论进行系统阐述,为后续视觉目标跟踪与意图识别系统的实现提供理论支撑。内容包括目标检测与视觉跟踪技术的基本概念,DeepSORT 跟踪算法的原理与工作流程,以及基于速度与空间信息的意图识别模型。同时,为便于理解,还将简要介绍在本研究中承担仿真任务的 Carla 平台的相关原理。

\section{目标检测与视觉跟踪概述}

自动驾驶系统当中,环境感知乃是达成安全行驶与智能决策的根基所在,通过计算机视觉技术来识别路上的车辆、行人、交通标志等重要物体,并持续追踪它们的动态状况,这属于完成环境创建以及行为预估的主要方法,目标检测和视觉跟随属于这里面非常关键的形成单元,会径直左右到自动驾驶系统对于周边环境的认识水平及其决策的精准度。

目标检测(Object Detection)即在输入图像当中找出全部存在的感兴趣目标,并精准地对这些目标在图像里的所在位置(一般用边界框来体现)以及所属类别实施回归,传统的目标检测算法大多依靠人工指定的特征加上滑动窗口这种机制,代表性的有Haar特征以及HOG + SVM检测器,此类方法虽然容易做到,但是在面对复杂背景的时候鲁棒性比较差,近些年来,深度学习不断发展起来,依靠卷积神经网络(CNN)的目标检测方法慢慢变成了主流,使得检测的准确率和速度均得到很大改善。 当下主流的检测模型大致可归为两类,其一为以FasterR - CNN作代表的两阶段检测器,该检测器先产生候选区域而后展开类别判断及边框回归操作,此类检测器精度较高,但速度偏慢。另一类是以YOLO(You Only Look Once),SSD(Single Shot MultiBoxDetector)等为典型的单阶段检测器,它们直接于特征图上实施回归预测,具备更快的即时性,适宜于像自动驾驶这般对时延较为敏感的场合当中。

与目标检测不同,目标跟踪(Object Tracking)是指在已知初始检测结果的前提下,持续对目标在视频序列中的位置进行估计。根据跟踪目标的数量,通常可分为单目标跟踪(Single Object Tracking, SOT)与多目标跟踪(Multiple Object Tracking, MOT)。单目标跟踪关注于对一个特定目标进行持续跟踪,其主要挑战在于遮挡、快速运动、目标消失与再出现等;而多目标跟踪则需要同时对多个目标进行识别与数据关联,面临着更高的关联复杂性与遮挡问题。在实际的自动驾驶场景中,由于交通参与者种类多、状态变化快、相互干扰强,因此往往需在较短时间内完成高准确率的多目标跟踪任务。

视觉目标跟踪往往把目标检测结果当作输入,依靠匹配机制达成目标的跨帧关联,按照是不是利用外部检测结果,跟踪算法可被划列为两类,其中一类是依托检测的跟踪(Tracking - by - Detection),此类方法先通过检测器得到每帧图像里的目标位置,再凭借轨迹预测和数据关联部件来做到目标编号的一致,这属于当下主流的工程化做法;另一类则是端到端的跟踪方法,它直接通过时序特征对目标的运动轨迹执行建模,适宜于复杂行为的建模任务,前面那种因为便于部署而且能和已有检测模型相适应,成了不少自动驾驶感知系统的优先选项。

本研究中,为加强系统的即时性与稳定性,采用“检测-跟踪”分离式结构,也就是先通过图像解析获取候选目标的边界框及状态信息,再借助外部跟踪器(DeepSORT)执行跨帧目标关联,这种结构可兼顾深度检测模型的精准度优势以及目标的运动学特点,达成即时,持续,稳定的目标轨迹追踪,而且还加入了单目标跟踪策略,专门关注当前可能与本车发生交互危险的目标,从而优化后面的意图分析环节的准确性与实用价值。

目标检测与视觉跟踪技术属于自动驾驶系统感知层的关键支撑部分,会给整个系统的安全,即时以及可靠程度带来直接影响,在这个课题当中,这两项技术处于算法链的起始位置,会给后面的意图识别和行为预测给予基本的信息支持。

\section{卡尔曼滤波}

卡尔曼滤波(KalmanFilter)\cite{kalman1960new}是由美国的工程师RudolphKalman 在1960 年提出的。作为一种最初为解决导航问题而提出的数学方法,卡尔曼滤波技术随后在系统辨识、状态估计与跟踪等多个领域展现出广泛的应用价值。该算法基于严格的数学理论框架,专门针对线性系统中的不确定性信息处理问题,通过最优估计方法实现对系统状态的精确描述。其核心思想在于将包含位置、速度及加速度等参数的线性系统模型与测量噪声的不确定性进行有机结合,通过迭代地融合实测数据与模型预测结果,持续优化状态估计过程,从而显著提升系统状态描述的准确性。

从理论架构来看,卡尔曼滤波的数学模型主要由预测模型和更新模型两个关键部分组成:
\begin{equation}
	X_t = \begin{bmatrix}
		cx, cy, w/h, h, \dot{c}x, \dot{c}y, \dot{w}/h, \dot{h}
	\end{bmatrix}^T
\label{eq:kalman1}
\end{equation}

\begin{equation}
	X_t = A X_{t-1} + B u_{t-1} + w_{t-1}
\label{eq:kalman2}
\end{equation}

\begin{equation}
	Y_t = [x, y, w/h, h]^T
\label{eq:kalman3}
\end{equation}

\begin{equation}
	Y_t = HX_t + V_t
\label{eq:kalman4}
\end{equation}

其中式~\eqref{eq:kalman1}是 t 时刻的状态向量,起到估计作用,其中x,y,w,h 分别为物体的中心位置坐标和长宽,剩余四个参数表示为对应四个参数的速度分量。式~\eqref{eq:kalman2}中,A,B,\(u_{t - 1}\)分别是状态转移矩阵、控制输入矩阵和系统控制量。\(w_{t - 1}\) 是服从协方差为Q的高斯过程噪声。式~\eqref{eq:kalman3}为t时刻的观测值Yt,也就是检测器检测出来的检测框的信息。式~\eqref{eq:kalman4}中,H 是观测矩阵,\(V_t\)是服从协方差为R的高斯观测噪声。

在t时刻,由状态转移矩阵A以及上一帧的最优估计\(\hat{x}_{t - 1}\)来预测t时刻的状态,并且t 时刻还可以获得观测值\(Z_t\)。通过卡尔曼增益K 将\(Z_t\) 和先验的t时刻的状态预测值\(\bar{\hat{x}}_t\)线性相加,就可以得到t时刻的最优估计\(\hat{x}_t\),\(\hat{x}_t\) 也会成为t+1 时刻先验状态预测值\(\bar{\hat{x}}_{t + 1}\)的输入。

以下是卡尔曼滤波的预测和更新过程:

\begin{table}[htbp]
	\caption{卡尔曼滤波}
	\label{tab:software_stack}
	\centering
	\begin{tabular}{l p{5cm} p{5cm}} % 第一列使用 l(左对齐),其他列使用 p{宽度}
		\toprule
		方程 & 说明 & 参数说明 \\
		\midrule
		\(\bar{\hat{x}}_t = A\bar{\hat{x}}_{t - 1} + Bu_{t - 1}\)   & 状态预测方程 & \(\bar{\hat{x}}_t\)是 t 时刻的状态预测值,\(\hat{x}_{t - 1}\)是 t-1 时刻的最优估计 \\
		\(P_t^- = AP_{t - 1}A^T + Q\)  & 协方差预测方程 & \(P_t^-\) 是 t 时刻的先验估计协方差,\(P_{t - 1}\)是 t-1 时刻的状态估计协方差 \\
		\(K_t = P_t^-H^T(HP_t^-H^T + R)^{-1}\)  & 卡尔曼增益更新方程 & \(K_t\)是卡尔曼增益,H 为状态观测矩阵 \\
		\(\hat{x}_t = \bar{\hat{x}}_t + K_t(Z_t - H\bar{\hat{x}}_t)\)   & 后验状态估计更新方程 & \(\hat{x}_t\) 为 t 时刻的最优估计 \\
		\(P_t = (I - K_tH)P_t^-\)  & 后验估计协方差方程 & I 为单位矩阵 \\
		\bottomrule
	\end{tabular}
\end{table}

\section{DeepSORT 目标跟踪算法}

DeepSORT作为多目标跟踪领域有代表性的算法,它的理论框架是基于“检测 - 跟踪”范式搭建起来的,凭借把目标运动学特征和深度语义信息相融合,为动态场景里目标身份的一致性维护给出了新的理论角度,此算法打破了传统跟踪方法对单一运动模型的依赖,提出了“双度量关联”理论。也就是借助卡尔曼滤波构建目标运动状态空间模型来捕捉短时轨迹的连续性,同时运用深度卷积网络提取目标的全局外观特征,以此提高跨帧身份辨识能力,这种运动与外观联合判据的引入,实际上解决了复杂交通场景中因目标轨迹交叉、短暂遮挡造成的身份歧义性问题,为多目标跟踪提供了更高维度的特征解空间。 

从理论层面来分析,DeepSORT的核心贡献在于构建了可微分的数据关联机制,依靠马氏距离与余弦相似度的加权融合,把目标运动不确定性纳入关联代价函数,让算法可自适应地调整不同场景下的特征权重配比,其级联匹配策略依靠时间衰减函数优先匹配近期活跃轨迹,本质上是模拟了人类视觉系统对目标存在性的认知惯性,有效地抑制了因长时间遮挡引发的虚警轨迹生成。在自动驾驶场景中,该算法依靠轻量化设计平衡了跟踪精度与实时性需求,其理论框架为动态目标行为建模提供了可扩展的接口,使后续研究中意图识别模块的嵌入有理论可行性,然而基于CNN的外观特征提取器对局部形变与光照变化比较敏感,以及运动模糊引起的特征退化现象,依然是该算法在高速动态场景下的理论瓶颈,这为引入尺度不变特征等改进方向提供了理论突破口。

\section{行为意图分析}

行为意图分析是自动驾驶感知系统从环境描述迈向行为理解的关键理论问题,其本质是凭借有限观测数据来解读交通参与者潜在的运动策略,传统的意图识别方法大多依据运动学参数阈值设定或者轨迹模式匹配,然而这类方法在动态交互场景里存在表征维度单一以及语义泛化能力较弱等理论方面的不足。本文从交通行为动力学角度重新构建意图分析的理论框架,提出“运动状态 - 场景语义 - 交互约束”三重耦合模型,以此来构建有较强可解释性且物理一致性较高的意图推理机制。

在理论基础这一层面,意图可被视作目标在未来时空域中运动轨迹的概率分布,其产生受到三方面因素的推动:一方面是本体运动学约束,囊括车辆的非完整运动特性以及动力学极限,另一方面是场景语义引导,涉及车道拓扑结构、交通规则以及道路边界条件,另一方面是社会交互博弈,也就是目标与周边车辆、行人间的协同或者竞争行为模式。基于此,本文构建的意图分析模型以贝叶斯推理作为数学框架,把意图识别问题转变为后验概率最大化问题,即:
\begin{equation}
	P(I_t | Z_{1:t}) \propto P(Z_t | I_t) \cdot \sum_{I_{t-1}} P(I_t | I_{t - 1}) P(I_{t - 1} | Z_{1:t - 1})
\end{equation}

其中, $I_t$表示t时刻的意图状态,\(Z_{1:t}\) 为历史观测序列。该模型运用隐马尔可夫链描绘意图状态的时序演变规律,突破了静态阈值法的瞬时决策局限,可捕捉像“渐进式变道”“试探性加速”这类长周期行为模式。

在特征工程方面,提出了多粒度运动表征理论:短期意图比如紧急制动依赖高帧率下的微分特征像Jerk值、横向加速度变化率,而长期意图比如路径规划则需要提取轨迹片段的方向一致性、曲率平滑度等宏观特征。借助引入冯·米塞斯分布对航向角偏差进行建模,解决了传统欧氏距离度量对方向敏感度不够的问题,另外设计了场景自适应权重分配机制,依据道路类型如高速公路或者城市道路动态调整运动特征与语义特征的贡献权重,例如在城市交叉口强化转向灯信号的置信度评估,而在高速场景中侧重相对速度的纵向威胁分析。

和纯数据驱动的黑箱模型相比,本文理论框架有优势:一是借助显式嵌入车辆运动学方程如自行车模型,保证预测轨迹符合物理规律,避免深度学习模型因数据偏差出现“幽灵变道”等非物理行为,二是基于信息熵理论设计意图不确定性量化指标,当目标行为同时匹配多个意图类别时,系统能触发高阶决策模块的冗余校验,提升复杂交互场景下的容错能力。不过现有模型对群体行为博弈如交叉路口多车冲突消解的建模深度还不够,未来要引入博弈论中的纳什均衡概念,构建基于多智能体强化学习的联合意图预测体系,以实现更高层次的场景认知智能化。


\begin{tabular}{l l}
%  \verb|\songti| & {\songti 宋体} \\
%  \verb|\heiti| & {\heiti 黑体} \\
%   \verb|\kaiti| & {\kaiti 楷体}
\end{tabular}
