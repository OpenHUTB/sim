%!TEX root = ../../csuthesis_main.tex
\chapter{总结与展望}

\section{工作总结}

伴随自动驾驶技术不断发展,怎样加强系统在复杂交通环境下的感知能力和决策智能性成了研究重点,本文通过设计并完成面向自动驾驶的视觉目标跟踪与意图分析系统,并结合Carla仿真平台,探究怎样借助目标检测,视觉跟踪以及意图分析等技术来助力自动驾驶系统变得更为安全、智能。

本研究针对“动态交通环境下的目标跟踪”这一问题,采取了DeepSORT算法,该算法在传统SORT算法之上增添了外观信息,从而加强了系统应对遮挡,快速移动等复杂情况时的鲁棒性,DeepSORT依靠卡尔曼滤波来预估目标位置,而且融合目标的外观特征实施数据关联,如此一来,即便目标被遮挡或者相互重叠,依然可以完成稳定跟踪,为便于后续展开意图分析与决策处理,本文在跟踪策略方面选取了单目标模式,着重关注当下正与本车交互最为密切的目标,缩减了计算量。

在意图分析上,本文给出了一种依靠物理模型的轻量化方法,通过剖析目标的速度,相对位置以及轨迹改变状况,规划出“靠近”,“远离”和“危险靠近”等行为预估准则,这样一种凭借物理量计算的做法,可以给予即时的行为预先判断,而且规避了深度学习模型也许会产生的计算压力,具有比较高的即时性,符合自动驾驶系统对于时效性的需求。

要保证系统稳定且具备可拓展性,研究依靠Carla仿真平台营造出很逼真的城市交通场景,用仿真数据集来做检测,Carla平台给予了包含道路,交通标志,行人等在内的完备交通环境,从而保障了实验数据的丰富性与真实性,在此基础之上,文章规划出一套数据收集和标识体系,通过RGB摄像头及时获取图像数据,而且得到目标物的二维边框,速度之类的信息,给后面的目标探测,追踪以及意图剖析赋予牢靠的数据支撑。

实验结果显示,DeepSORT算法在复杂交通环境下可维持目标稳定跟踪,免除了目标身份的混淆情况,依靠物理模型的意图分析方法,可以及时判定目标运动趋向,给后续的路线规划和决策给予精确的行为警报,而且,系统在仿真环境中的运行效率得到了证实,可以达成即时跟踪和警报,满足自动驾驶系统对于时效性的需求。

本文的贡献在于给出一种融合视觉感知和意图分析的全新自动驾驶方案,深度学习与物理模型相融合之后,既加强了系统的感知精准度,又改进了在复杂交通情况下的安全预测能力,研究所用到的Carla仿真平台给后续的算法改良和系统检测给予了关键的参照,将来,伴随自动驾驶技术不断发展,本文的研究成果有可能拓展至多目标追踪,群体行为分析等更为繁杂的任务当中,从而给智能驾驶系统的高效决策及其安全性赋予更强有力的技术支持。



\section{研究不足}

虽然本文就自动驾驶视觉目标跟踪及意图分析展开的研究收获了一些成果,可还是存在不少不够完善之处,其一,即便DeepSORT算法具备应对动态环境里目标被遮挡,高速运动之类状况的能力,但当遭遇极其繁杂的交通场景时,该算法仍旧有可能陷入目标身份错乱或者跟丢目标的困境之中,特别在高密度交通或者目标相互交织的情形之下,既有的算法也许很难在长时间的跟踪进程当中守住足够的鲁棒性,那么,怎样进一步优化DeepSORT在复杂场景中的性能,这依旧是个迫切必要解决的课题。

本文所提依靠物理模型的意图识别法可做到快速而精准的行为预估,但这种方法依靠预先指定好的规则及阈值,也许难以处理更为繁杂的驾驶行为,特别是存在大量目标物以及情况错综复杂的时候,伴随驾驶行为变得多种各类,情形也愈发繁杂,仅仅凭借简单的物理规则去做判断大概不能掌握全部有可能出现的意图,日后须要考量采用更多种高效的学习型模型,从而加强系统的智能决策水平。

而且,本文的实验只是局限于Carla仿真平台里的数据集,这个平台有着很强的可控性与仿真性,但是不能把真实交通环境里的各种复杂情况完全模仿出来,比如天气改变,光照情况,道路突然发生的状况等等,这也许会引发系统在实际环境当中碰上更多的难题和不确定因素,那么,怎样把仿真环境下得到的研究成果更好地应用到现实环境当中去,还是个必要深入探究的问题。


\section{后续优化方向}

即便本文给出的自动驾驶视觉目标跟踪与意图分析系统在仿真环境中有一些收获,可还是存有不少改良之处,其一,当下所用的DeepSORT算法在大部分情况下可做到有效的目标跟踪,但当处于高密度交通以及复杂目标交互情形的时候,却会发生目标身份转换和跟踪中断之类的状况,日后的研究不妨探寻更为优良的多目标跟踪算法,可以融合各类特征或者利用深度学习手段去提升目标跟踪的精准度及其鲁棒性,特别针对那些动态且繁杂的交通状况而言。

在意图分析模块当中,依靠物理规则的分析方法即便可以达成比较精确的行为预估,但它存在一定局限,也就是不能应对更繁杂的驾驶行为,特别当目标之间存在诸多交互的时候,以后的改良途径之一就是利用深度学习模型来做意图预测,格外是凭借深度加强学习之类技术的动态决策模型,可以通过学习改善自身的决策流程,进而更好地解决复杂场景下的不确定因素。

而且,本文的实验和验证大多依靠Carla仿真平台来做,这个平台虽然能给予高保真度的仿真环境,可是对比实际道路环境还是会有一些差别,真实环境里的光照改变,天气状况以及交通流的复杂程度,大概会给系统的稳定性和鲁棒性带来更高的需求,所以之后的工作应该把研究成果转为成实际应用,用真实世界的数据展开检测和验证,从而更好地考量系统在实际交通场景中的性能。

通过这些改良之后,将来的系统能够更好地应对动态且复杂的交通状况,进而加强自动驾驶系统的智能感知及决策能力,从而为达成更安全、更智能的自动驾驶技术打下根基。




\begin{tabular}{l l}
%  \verb|\songti| & {\songti 宋体} \\
%  \verb|\heiti| & {\heiti 黑体} \\
%   \verb|\kaiti| & {\kaiti 楷体}
\end{tabular}
