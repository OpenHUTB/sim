%!TEX root = ../csuthesis_main.tex
% \begin{appendixs} % 无章节编号
\chapter{附录代码}

\section{DeepSORT 多目标跟踪算法}

\begin{algorithm}[h]
    \caption{DeepSORT 多目标跟踪算法}\label{alg:ovf}
    \begin{algorithmic}[1]
		\STATE 初始化跟踪器集合 $\mathcal{T}$
		\FOR{每一帧图像}
		    \STATE 检测所有目标,生成检测集合 $\mathcal{D}$
		    \STATE 提取每个检测框的外观特征向量
		    \STATE 根据卡尔曼滤波器预测每个跟踪器的位置
		    \STATE 使用匈牙利算法(Hungarian Algorithm)匹配 $\mathcal{D}$ 与 $\mathcal{T}$,代价函数结合马氏距离和外观特征距离
		    \FOR{每个成功匹配的检测与跟踪器对}
		        \STATE 更新跟踪器状态(位置、速度、外观特征)
		    \ENDFOR
		    \FOR{每个未匹配到检测的跟踪器}
		        \STATE 标记为失配,增加失配计数
		    \ENDFOR
		    \FOR{每个未匹配到跟踪器的检测}
		        \STATE 初始化新的跟踪器
		    \ENDFOR
		    \STATE 移除失效跟踪器(如失配次数超过最大阈值)
		\ENDFOR
    \end{algorithmic}
\end{algorithm}

\section{面向自动驾驶的视觉目标跟踪和意图分析算法}
% \begin{minted}[linenos]{c}
\begin{lstlisting}
	#!/usr/bin/env python
	
	# Copyright (c) 2019 Aptiv
	#
	# This work is licensed under the terms of the MIT license.
	# For a copy, see <https://opensource.org/licenses/MIT>.
	
	"""
	An example of client-side bounding boxes with basic car controls.
	
	Controls:
	
	    W            : throttle
	    S            : brake
	    AD           : steer
	    Space        : hand-brake
	
	    ESC          : quit
	"""
	
	# ==============================================================================
	# -- find carla module ---------------------------------------------------------
	# ==============================================================================
	
	import json
	import cv2
	from datetime import datetime
	from deep_sort_realtime.deepsort_tracker import DeepSort
	
	import pathlib
	pathlib.Path("data/images").mkdir(parents=True, exist_ok=True)
	pathlib.Path("data/labels").mkdir(parents=True, exist_ok=True)
	
	import glob
	import os
	import sys
	
	try:
	    sys.path.append(glob.glob('../carla/dist/carla-*%d.%d-%s.egg' % (
	        sys.version_info.major,
	        sys.version_info.minor,
	        'win-amd64' if os.name == 'nt' else 'linux-x86_64'))[0])
	except IndexError:
	    pass
	
	
	# ==============================================================================
	# -- imports -------------------------------------------------------------------
	# ==============================================================================
	
	import carla
	
	import weakref
	import random
	
	try:
	    import pygame
	    from pygame.locals import K_ESCAPE
	    from pygame.locals import K_SPACE
	    from pygame.locals import K_a
	    from pygame.locals import K_d
	    from pygame.locals import K_s
	    from pygame.locals import K_w
	except ImportError:
	    raise RuntimeError('cannot import pygame, make sure pygame package is installed')
	
	try:
	    import numpy as np
	except ImportError:
	    raise RuntimeError('cannot import numpy, make sure numpy package is installed')
	
	VIEW_WIDTH = 1920//2
	VIEW_HEIGHT = 1080//2
	VIEW_FOV = 90
	
	BB_COLOR = (248, 64, 24)
	
	# ==============================================================================
	# -- ClientSideBoundingBoxes ---------------------------------------------------
	# ==============================================================================
	
	
	# 在 ClientSideBoundingBoxes 类中添加一个方法来获取车辆的速度并渲染速度文本
	
	class ClientSideBoundingBoxes(object):
	    @staticmethod
	    def get_bounding_boxes(vehicles, camera):
	        """
	        Creates 3D bounding boxes based on carla vehicle list and camera.
	        """
	        bounding_boxes = []
	        for vehicle in vehicles:
	            bbox = ClientSideBoundingBoxes.get_bounding_box(vehicle, camera)
	            speed = vehicle.get_velocity()
	            speed_magnitude = np.sqrt(speed.x**2 + speed.y**2 + speed.z**2)  # 计算车辆的速度大小
	            bounding_boxes.append((bbox, speed_magnitude))  # 返回边界框和速度
	        # filter objects behind camera
	        bounding_boxes = [bb for bb in bounding_boxes if all(bb[0][:, 2] > 0)]
	        return bounding_boxes
	
	    @staticmethod
	    def draw_bounding_boxes(display, bounding_boxes):
	        """
	        Draws bounding boxes on pygame display.
	        """
	        bb_surface = pygame.Surface((VIEW_WIDTH, VIEW_HEIGHT))
	        bb_surface.set_colorkey((0, 0, 0))
	        chinese_font = pygame.font.Font("C:/Windows/Fonts/simhei.ttf", 20)
	
	        for bbox, speed in bounding_boxes:
	            points = [(int(bbox[i, 0]), int(bbox[i, 1])) for i in range(8)]
	            # 绘制边界框
	            pygame.draw.line(bb_surface, BB_COLOR, points[0], points[1])
	            pygame.draw.line(bb_surface, BB_COLOR, points[0], points[1])
	            pygame.draw.line(bb_surface, BB_COLOR, points[1], points[2])
	            pygame.draw.line(bb_surface, BB_COLOR, points[2], points[3])
	            pygame.draw.line(bb_surface, BB_COLOR, points[3], points[0])
	            # top
	            pygame.draw.line(bb_surface, BB_COLOR, points[4], points[5])
	            pygame.draw.line(bb_surface, BB_COLOR, points[5], points[6])
	            pygame.draw.line(bb_surface, BB_COLOR, points[6], points[7])
	            pygame.draw.line(bb_surface, BB_COLOR, points[7], points[4])
	            # base-top
	            pygame.draw.line(bb_surface, BB_COLOR, points[0], points[4])
	            pygame.draw.line(bb_surface, BB_COLOR, points[1], points[5])
	            pygame.draw.line(bb_surface, BB_COLOR, points[2], points[6])
	            pygame.draw.line(bb_surface, BB_COLOR, points[3], points[7])
	            
	            # 绘制速度文本
	            speed_text = f"{speed:.2f} m/s"  # 显示速度,保留两位小数
	            text_surface = chinese_font.render(speed_text, True, (255, 255, 255))  # 白色文字
	            text_rect = text_surface.get_rect(center=(int(bbox[0, 0]), int(bbox[0, 1]) - 10))  # 在边界框上方显示
	            bb_surface.blit(text_surface, text_rect)
	        
	        display.blit(bb_surface, (0, 0))
	
	
	    @staticmethod
	    def get_bounding_box(vehicle, camera):
	        """
	        Returns 3D bounding box for a vehicle based on camera view.
	        """
	
	        bb_cords = ClientSideBoundingBoxes._create_bb_points(vehicle)
	        cords_x_y_z = ClientSideBoundingBoxes._vehicle_to_sensor(bb_cords, vehicle, camera)[:3, :]
	        cords_y_minus_z_x = np.concatenate([cords_x_y_z[1, :], -cords_x_y_z[2, :], cords_x_y_z[0, :]])
	        bbox = np.transpose(np.dot(camera.calibration, cords_y_minus_z_x))
	        camera_bbox = np.concatenate([bbox[:, 0] / bbox[:, 2], bbox[:, 1] / bbox[:, 2], bbox[:, 2]], axis=1)
	        return camera_bbox
	
	    @staticmethod
	    def _create_bb_points(vehicle):
	        """
	        Returns 3D bounding box for a vehicle.
	        """
	
	        cords = np.zeros((8, 4))
	        extent = vehicle.bounding_box.extent
	        cords[0, :] = np.array([extent.x, extent.y, -extent.z, 1])
	        cords[1, :] = np.array([-extent.x, extent.y, -extent.z, 1])
	        cords[2, :] = np.array([-extent.x, -extent.y, -extent.z, 1])
	        cords[3, :] = np.array([extent.x, -extent.y, -extent.z, 1])
	        cords[4, :] = np.array([extent.x, extent.y, extent.z, 1])
	        cords[5, :] = np.array([-extent.x, extent.y, extent.z, 1])
	        cords[6, :] = np.array([-extent.x, -extent.y, extent.z, 1])
	        cords[7, :] = np.array([extent.x, -extent.y, extent.z, 1])
	        return cords
	
	    @staticmethod
	    def _vehicle_to_sensor(cords, vehicle, sensor):
	        """
	        Transforms coordinates of a vehicle bounding box to sensor.
	        """
	
	        world_cord = ClientSideBoundingBoxes._vehicle_to_world(cords, vehicle)
	        sensor_cord = ClientSideBoundingBoxes._world_to_sensor(world_cord, sensor)
	        return sensor_cord
	
	    @staticmethod
	    def _vehicle_to_world(cords, vehicle):
	        """
	        Transforms coordinates of a vehicle bounding box to world.
	        """
	
	        bb_transform = carla.Transform(vehicle.bounding_box.location)
	        bb_vehicle_matrix = ClientSideBoundingBoxes.get_matrix(bb_transform)
	        vehicle_world_matrix = ClientSideBoundingBoxes.get_matrix(vehicle.get_transform())
	        bb_world_matrix = np.dot(vehicle_world_matrix, bb_vehicle_matrix)
	        world_cords = np.dot(bb_world_matrix, np.transpose(cords))
	        return world_cords
	
	    @staticmethod
	    def _world_to_sensor(cords, sensor):
	        """
	        Transforms world coordinates to sensor.
	        """
	
	        sensor_world_matrix = ClientSideBoundingBoxes.get_matrix(sensor.get_transform())
	        world_sensor_matrix = np.linalg.inv(sensor_world_matrix)
	        sensor_cords = np.dot(world_sensor_matrix, cords)
	        return sensor_cords
	
	    @staticmethod
	    def get_matrix(transform):
	        """
	        Creates matrix from carla transform.
	        """
	
	        rotation = transform.rotation
	        location = transform.location
	        c_y = np.cos(np.radians(rotation.yaw))
	        s_y = np.sin(np.radians(rotation.yaw))
	        c_r = np.cos(np.radians(rotation.roll))
	        s_r = np.sin(np.radians(rotation.roll))
	        c_p = np.cos(np.radians(rotation.pitch))
	        s_p = np.sin(np.radians(rotation.pitch))
	        matrix = np.matrix(np.identity(4))
	        matrix[0, 3] = location.x
	        matrix[1, 3] = location.y
	        matrix[2, 3] = location.z
	        matrix[0, 0] = c_p * c_y
	        matrix[0, 1] = c_y * s_p * s_r - s_y * c_r
	        matrix[0, 2] = -c_y * s_p * c_r - s_y * s_r
	        matrix[1, 0] = s_y * c_p
	        matrix[1, 1] = s_y * s_p * s_r + c_y * c_r
	        matrix[1, 2] = -s_y * s_p * c_r + c_y * s_r
	        matrix[2, 0] = s_p
	        matrix[2, 1] = -c_p * s_r
	        matrix[2, 2] = c_p * c_r
	        return matrix
	
	
	# ==============================================================================
	# -- BasicSynchronousClient ----------------------------------------------------
	# ==============================================================================
	
	
	class BasicSynchronousClient(object):
	    """
	    Basic implementation of a synchronous client.
	    """
	    def save_frame_and_labels(self, array, bounding_boxes, frame_idx):
	        """
	        保存当前帧图像和目标边界框 + 速度信息
	        """
	        timestamp = datetime.now().strftime("%Y%m%d_%H%M%S")
	        img_filename = f"frame_{frame_idx}_{timestamp}.jpg"
	        json_filename = img_filename.replace('.jpg', '.json')
	
	        # 保存图像
	        img_path = os.path.join("data/images", img_filename)
	        cv2.imwrite(img_path, array)
	
	        # 保存标注
	        label_data = []
	        for bbox, speed in bounding_boxes:
	            points = [(int(bbox[i, 0]), int(bbox[i, 1])) for i in range(4)]
	            is_tracked = False
	            if self.tracking_mode and self.target_id is not None:
	                x_min = min([p[0] for p in points])
	                y_min = min([p[1] for p in points])
	                x_max = max([p[0] for p in points])
	                y_max = max([p[1] for p in points])
	                box_area = (x_max - x_min) * (y_max - y_min)
	                # 简易判定是否与当前追踪目标相符(可改进为 IoU)
	                is_tracked = True if box_area > 1000 else False
	
	            label_data.append({
	                "bbox": points,
	                "speed_m_s": round(speed, 2),
	                "tracked_id": self.target_id if is_tracked else None
	            })
	
	
	        with open(os.path.join("data/labels", json_filename), 'w') as f:
	            json.dump(label_data, f, indent=2)
	
	    def __init__(self):
	        self.client = None
	        self.world = None
	        self.camera = None
	        self.car = None
	        self.display = None
	        self.image = None
	        self.capture = True
	        self.tracker = DeepSort(max_age=15)
	        self.target_id = None         # 当前跟踪的目标 ID
	        self.tracking_mode = True     # 是否启用追踪(便于后期控制开关)
	        self.prev_distance = None        # 上一帧距离
	        self.intent_text = ""           # 当前分析结果
	
	
	    def select_closest_target(self, bounding_boxes):
	        if not bounding_boxes:
	            print("[TRACK] 没有检测到目标")
	            return None
	        """
	        在所有检测目标中选择最近一个(速度最快+框最近)
	        """
	
	        # 简单使用左上角点距离中心来估算“接近程度”
	        center_x = VIEW_WIDTH / 2
	        center_y = VIEW_HEIGHT / 2
	
	        min_dist = float('inf')
	        closest_bbox = None
	        for bbox, speed in bounding_boxes:
	            x_coords = bbox[:, 0]
	            y_coords = bbox[:, 1]
	            x_min, y_min = np.min(x_coords), np.min(y_coords)
	            dist = np.sqrt((x_min - center_x) ** 2 + (y_min - center_y) ** 2)
	            if dist < min_dist:
	                min_dist = dist
	                closest_bbox = (bbox, speed)
	        return closest_bbox
	
	
	    def analyze_intention(self, bbox, speed):
	        """
	        基于边界框位置和速度分析当前意图
	        """
	        # 当前中心点
	        x_coords = bbox[:, 0]
	        y_coords = bbox[:, 1]
	        x_center = np.mean(x_coords)
	        y_center = np.mean(y_coords)
	        current_center = (x_center, y_center)
	
	        # 自车视图中心(屏幕正下方)
	        car_center = (VIEW_WIDTH / 2, VIEW_HEIGHT)
	
	        # 计算欧氏距离
	        distance = np.linalg.norm(np.array(current_center) - np.array(car_center))
	
	        if self.prev_distance is None:
	            self.prev_distance = distance
	            self.intent_text = "目标初始化中"
	            return
	
	        delta_d = distance - self.prev_distance
	        self.prev_distance = distance
	
	        if delta_d < -5 and speed > 1.5:
	            self.intent_text = "目标靠近中"
	        elif delta_d > 5:
	            self.intent_text = "目标远离中"
	        elif distance < 150 and speed > 3:
	            self.intent_text = "危险靠近"
	        else:
	            self.intent_text = "目标稳定"
	
	
	    def camera_blueprint(self):
	        """
	        Returns camera blueprint.
	        """
	
	        camera_bp = self.world.get_blueprint_library().find('sensor.camera.rgb')
	        camera_bp.set_attribute('image_size_x', str(VIEW_WIDTH))
	        camera_bp.set_attribute('image_size_y', str(VIEW_HEIGHT))
	        camera_bp.set_attribute('fov', str(VIEW_FOV))
	        return camera_bp
	
	    def set_synchronous_mode(self, synchronous_mode):
	        """
	        Sets synchronous mode.
	        """
	
	        settings = self.world.get_settings()
	        settings.synchronous_mode = synchronous_mode
	        self.world.apply_settings(settings)
	
	    def setup_car(self):
	        """
	        Spawns actor-vehicle to be controled.
	        """
	
	        car_bp = self.world.get_blueprint_library().filter('vehicle.*')[0]
	        location = random.choice(self.world.get_map().get_spawn_points())
	        self.car = self.world.spawn_actor(car_bp, location)
	
	    def setup_camera(self):
	        """
	        Spawns actor-camera to be used to render view.
	        Sets calibration for client-side boxes rendering.
	        """
	
	        camera_transform = carla.Transform(carla.Location(x=-5.5, z=2.8), carla.Rotation(pitch=-15))
	        self.camera = self.world.spawn_actor(self.camera_blueprint(), camera_transform, attach_to=self.car)
	        weak_self = weakref.ref(self)
	        self.camera.listen(lambda image: weak_self().set_image(weak_self, image))
	
	        calibration = np.identity(3)
	        calibration[0, 2] = VIEW_WIDTH / 2.0
	        calibration[1, 2] = VIEW_HEIGHT / 2.0
	        calibration[0, 0] = calibration[1, 1] = VIEW_WIDTH / (2.0 * np.tan(VIEW_FOV * np.pi / 360.0))
	        self.camera.calibration = calibration
	
	    def control(self, car):
	        """
	        Applies control to main car based on pygame pressed keys.
	        Will return True If ESCAPE is hit, otherwise False to end main loop.
	        """
	
	        keys = pygame.key.get_pressed()
	        if keys[K_ESCAPE]:
	            return True
	
	        control = car.get_control()
	        control.throttle = 0
	        if keys[K_w]:
	            control.throttle = 1
	            control.reverse = False
	        elif keys[K_s]:
	            control.throttle = 1
	            control.reverse = True
	        if keys[K_a]:
	            control.steer = max(-1., min(control.steer - 0.05, 0))
	        elif keys[K_d]:
	            control.steer = min(1., max(control.steer + 0.05, 0))
	        else:
	            control.steer = 0
	        control.hand_brake = keys[K_SPACE]
	
	        car.apply_control(control)
	        return False
	
	    @staticmethod
	    def set_image(weak_self, img):
	        """
	        Sets image coming from camera sensor.
	        The self.capture flag is a mean of synchronization - once the flag is
	        set, next coming image will be stored.
	        """
	
	        self = weak_self()
	        if self.capture:
	            self.image = img
	            self.capture = False
	
	    def render(self, display):
	        """
	        渲染图像并返回 RGB 数组(用于保存)
	        """
	        if self.image is not None:
	            array = np.frombuffer(self.image.raw_data, dtype=np.dtype("uint8"))
	            array = np.reshape(array, (self.image.height, self.image.width, 4))
	            rgb_array = array[:, :, :3]  # RGB,不做反转
	            bgr_array = rgb_array[:, :, ::-1]  # BGR for pygame
	
	            surface = pygame.surfarray.make_surface(bgr_array.swapaxes(0, 1))
	            display.blit(surface, (0, 0))
	            return rgb_array  # 返回原始 RGB 图像用于保存
	        return None
	
	
	
	    def game_loop(self):
	        """
	        Main program loop.
	        """
	
	        try:
	            pygame.init()
	
	            self.client = carla.Client('127.0.0.1', 2000)
	            self.client.set_timeout(2.0)
	            self.world = self.client.get_world()
	
	            self.setup_car()
	            self.setup_camera()
	
	            self.display = pygame.display.set_mode((VIEW_WIDTH, VIEW_HEIGHT), pygame.HWSURFACE | pygame.DOUBLEBUF)
	            pygame_clock = pygame.time.Clock()
	
	            self.set_synchronous_mode(True)
	            vehicles = [v for v in self.world.get_actors().filter('vehicle.*') if v.id != self.car.id]
	
	
	            frame_count = 0  # 添加在循环开始前的计数器
	            while True:
	                self.world.tick()
	                self.capture = True
	                pygame_clock.tick_busy_loop(20)
	
	                frame_count += 1
	                rgb_array = self.render(self.display)
	
	                bounding_boxes = ClientSideBoundingBoxes.get_bounding_boxes(vehicles, self.camera)
	                ClientSideBoundingBoxes.draw_bounding_boxes(self.display, bounding_boxes)
	                # ---- 单目标追踪逻辑 ----
	                target_input = None
	                if self.tracking_mode:
	                    closest = self.select_closest_target(bounding_boxes)
	                    if closest:
	                        bbox, speed = closest
	                        x_coords = bbox[:, 0]
	                        y_coords = bbox[:, 1]
	                        x_min, y_min = np.min(x_coords), np.min(y_coords)
	                        x_max, y_max = np.max(x_coords), np.max(y_coords)
	                        width = x_max - x_min
	                        height = y_max - y_min
	
	                        # 避免空框
	                        if width >= 10 and height >= 10:
	                            target_input = [([x_min, y_min, width, height], 0.9, "vehicle")]
	
	                # 用 DeepSort 追踪目标
	
	                track_id = None
	                if target_input:
	                    tracks = self.tracker.update_tracks(target_input, frame=rgb_array)
	
	                    bb_surface = pygame.Surface((VIEW_WIDTH, VIEW_HEIGHT))
	                    bb_surface.set_colorkey((0, 0, 0))
	                    font = pygame.font.SysFont("Arial", 20)
	
	                    for track in tracks:
	                        if not track.is_confirmed():
	                            continue
	                        track_id = track.track_id
	                        l, t, r, b = track.to_ltrb()
	                        pygame.draw.rect(bb_surface, (255, 255, 0), pygame.Rect(l, t, r - l, b - t), 2)
	                        text_surface = font.render(f"Tracked ID: {track_id}", True, (255, 255, 255))
	                        bb_surface.blit(text_surface, (int(l), int(t) - 20))
	                        # 分析意图(基于 closest)
	                        if closest:
	                            bbox, speed = closest
	                            self.analyze_intention(bbox, speed)
	
	                        chinese_font = pygame.font.Font("C:/Windows/Fonts/simhei.ttf", 20)
	                        intent_color = {
	                            "危险靠近": (255, 0, 0),
	                            "目标靠近中": (255, 128, 0),
	                            "目标远离中": (0, 255, 0),
	                            "目标稳定": (200, 200, 200),
	                            "目标初始化中": (150, 150, 150)
	                        }.get(self.intent_text, (255, 255, 255))
	                        intent_surface = chinese_font.render(self.intent_text, True, intent_color)
	                        bb_surface.blit(intent_surface, (int(l), int(t) - 40))
	
	
	
	                    self.display.blit(bb_surface, (0, 0))
	
	                self.target_id = track_id
	                # ---- end ----
	
	
	                # 每 5 帧保存一次
	                if frame_count % 5 == 0 and rgb_array is not None:
	                    self.save_frame_and_labels(rgb_array, bounding_boxes, frame_count)
	
	                pygame.display.flip()
	                pygame.event.pump()
	                if self.control(self.car):
	                    return
	
	        finally:
	            self.set_synchronous_mode(False)
	            self.camera.destroy()
	            self.car.destroy()
	            pygame.quit()
	
	
	# ==============================================================================
	# -- main() --------------------------------------------------------------------
	# ==============================================================================
	
	
	def main():
	    """
	    Initializes the client-side bounding box demo.
	    """
	
	    try:
	        client = BasicSynchronousClient()
	        client.game_loop()
	    finally:
	        print('EXIT')
	
	
	if __name__ == '__main__':
	    main()

\end{lstlisting}
% \end{minted}



% \end{appendixs}
